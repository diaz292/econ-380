\documentclass[pdf]{beamer}
\usetheme{Frankfurt}  
\usecolortheme{whale}
\usepackage{tikz} 
\usepackage{amsmath}
\usepackage{amsthm}
\usepackage{amssymb}              % used for \eqref{} in this document
\usepackage{dsfont}
\usepackage{hyperref}
\usepackage{threeparttable}
\usepackage{multirow}
\graphicspath{{Figures/}}
\usepackage{booktabs}
\usepackage{tikz}
\newtheorem{exmp}{Example}[section]
\usepackage{subcaption}
\usepackage{adjustbox}
\usepackage{graphicx}
\usepackage[mathscr]{euscript}
\usepackage{remreset}% tiny package containing just the \@removefromreset command
\makeatletter
\@removefromreset{subsection}{section}
\makeatother
\setcounter{subsection}{1}


\section{The Short-Run}

%% preamble
\title{Part IIC: Labor Demand}
\author[David A. D\'iaz]{David A. D\'iaz}
\institute{UNC Chapel Hill}
\date{}


\AtBeginSection[] %Section links on slides


\begin{document}
	
	\begin{frame}
		
		\titlepage
		
	\end{frame}
	
\begin{frame}{Labor Market Equilibrium}
	\begin{itemize}
		\item Earlier:  Developed theory of labor supply.  
		\item But, policy analysis only considered one side of market!
		\item Many questions remain:
		\begin{itemize}
		\item Where does the wage offered to workers come from in the first place?
		\item How can we explain many real-world properties of labor markets?
		\begin{itemize}
		\item Differences in wages due to (i) skills differences, (ii) irrelevant factors e.g. race
		\item Unemployment:  Why can’t workers always find a firm willing to hire at the market rate?
		Etc.
		\end{itemize}
		\item Need a model of firm behavior to consider these questions.
		\end{itemize}
	\end{itemize}
\end{frame}
	
\begin{frame}{The Firm's Objective}
\begin{itemize}
	\item What is the firm's goal?
	\begin{itemize}
		\item To maximize \textbf{profit}: $\Pi = TR - TC$
	\end{itemize}
	\item Assumption: The markets for inputs (e.g., labor \& capital) and output (the product the firm is producing) are competitive
	\begin{itemize}
		\item Firms are price takers
		\item Market price of output is $p$
		\item Market price of labor is $w$
		\item Market price of capital is $r$ 
	\end{itemize}
\end{itemize}
\end{frame}

\begin{frame}{The Firm's Constraint}
	\begin{itemize}
		\item Total costs: $TC = rK + wE$, where $K$ is the stock of capital used and $E$ is the number of employee-hours hired
	\item Total revenue: $TR = pq$, where $q$ is the output quantity
		\item Firms produce output using labor ($E$) and capital ($K$) according to the firm's production function,
		\[q = f(K,E)\]
	
	\end{itemize}
\end{frame}

\begin{frame}{The Firm's Constraint}
\begin{itemize}

	\item Assumptions about $E$:
	\begin{itemize}
		\item $E$ = We will generally simplify $E$ to just be the number of workers hired by the firm
		\item $E$ aggregates all different types of workers into one measure. Ignores potential heterogeneity in productivity across workers.
	\end{itemize}
\end{itemize}
\end{frame}

\begin{frame}{The Production Function}
	\begin{itemize}
		\item From the production function, we can define 
		\begin{enumerate}
			\item \textbf{The marginal product of labor ($MP_E$)}: The change in output resulting from hiring an additional worker (holding $K$ constant)
			\item \textbf{The marginal product of capital ($MP_K$)}: The change in output resulting from a one-unit increase in the capital stock (holding $E$ constant)
			\item \textbf{The average product of labor ($AP_E$)}: The amount of output produced by the typical worker. $AP_E = q/E$
		\end{enumerate}
	\end{itemize}
\end{frame}


\begin{frame}{The Production Function}
\begin{itemize}
		\item Assumptions: 
\begin{itemize}
	\item $f(K,E)$ is increasing in $K$ and $E$, so $MP_K$ and $MP_E$ are both positive.
	\item $MP_K$ and $MP_E$ \textit{eventually} decline (i.e., follow the \textbf{law of diminishing returns})
\end{itemize}		
\item What is the relationship between $MP_E$ and $AP_E$?
\end{itemize}
\end{frame}

\begin{frame}{Labor Demand in the SR}
	\begin{itemize}
		\item \textbf{The short run:} A time span sufficiently short such that the firm cannot adjust its capital stock. $K=K_0$. The firm can adjust $E$.
		\item \textbf{The long run:} Capital and labor can both be adjusted by the firm.
		\item Implicitly, labor is more mobile than capital.
	\end{itemize}
\end{frame}

\begin{frame}{Labor Demand in the SR}
	\begin{itemize}
		\item Short-run decision: Given a fixed level of capital $K_0$, how many units of labor should the firm hire in order to maximize profits?
		\item Marginal cost of hiring an additional worker: $MC_E = w$
		\item Marginal benefit of hiring: The increase in revenue brought about from hiring an additional worker
		\item Per the usual, should hire until $MB = MC$ 
	\end{itemize}
\end{frame}

\begin{frame}{Labor Demand in the SR}
	\begin{itemize}
		\item How do we define the marginal benefit?
		\item Because the firm is a price taker, $\Delta TR = p\times \Delta q$
		\item The marginal product of labor is given by
		\[MP_E = \frac{\Delta q}{\Delta E}\]
		\item After a one unit increase in $E$, 
		\[MP_E = \Delta q \Rightarrow \Delta TR = p\times MP_E\]
	\end{itemize}
\end{frame}

\begin{frame}{Labor Demand in the SR}
	\begin{itemize}
			\item \textbf{Value of the Marginal Product of Labor ($VMP_E$):} The dollar value of what each additional worker produces (holding $K$ constant).
			\[VMP_E = p \times MP_E\]
			\item $VMP_E$ is the change in revenue that results from a one unit increase in labor (holding $K$ constant)
			\item \textbf{Value of the average product of labor ($VAP_E$):} The dollar value of output per worker.
			\[VAP_E = p\times AP_E\]
	\end{itemize}
\end{frame}


\begin{frame}{Labor Demand in the SR}
	\begin{itemize}
		\item Optimal hiring rule: Hire until $MB = MC \Rightarrow VMP_E = w$
		\item Additional condition: $VMP_E$ must be declining.
		\item Moreover, the firm will only hire along points of the $VMP_E$ curve that lie \textit{below} the point where $VMP_E$ and $VAP_E$ meet.
	\end{itemize}
\end{frame}

\begin{frame}{Labor Demand in the SR}
\begin{itemize}
		\item This is referred to as the \textbf{marginal productivity condition}
	\item This condition is identical to the condition you have seen before where the firm produces until $MR = MC$
	\begin{itemize}
		\item The condition telling firms when to stop producing output is the same as the condition telling firms to stop hiring workers.
	\end{itemize}
\end{itemize}
\end{frame}

\begin{frame}{Labor Demand in the SR}
\begin{exmp}
Suppose the hourly wage is \$10 and the price of each unit of capital is \$25. The price of output is constant at \$50. The production function is given by 
\[f(E,K) = E^{1/2}K^{1/2},\]

which implies that $MP_E = (1/2)(K/E)^{1/2}$. If the current capital stock is fixed at 1,600 units, how much labor should the firm hire in the short run? How much profit will the firm earn?
\end{exmp}
\end{frame}

\begin{frame}{Short-Run Labor Demand}
	\begin{itemize}
		\item Recall that a labor demand curve gives the firm’s chosen level of employment as a function of the wage rate.
		\item Graphically, the short-run labor demand curve ($E_{SR}$) is just the downward-sloping portion of the $VMP_E$ curve
		\item The ``height'' of the labor demand curve depends on the price ($p$) of the output
		\begin{itemize}
			\item Positive relationship between short-run employment and $p$
		\end{itemize}
	\end{itemize}
\end{frame}

\begin{frame}{Short-Run Labor Demand}
	\begin{itemize}
		\item An important implication of the marginal productivity condition:
		\begin{itemize}
			\item At the margin, workers are paid a wage equivalent to what they earn for the firm.
			\item Workers are paid precisely what they're worth to the firm!
		\end{itemize}
		\item This condition does not hold if labor markets are not competitive.
		\begin{itemize}
			\item Competitive markets provide workers with high level of surplus.
			\item Do we believe labor markets are competitive?
			\item How can we reinterpret this condition in a non-competitive market?
			\item More on this later.
		\end{itemize}
	\end{itemize}
\end{frame}

\begin{frame}{Short-Run Labor Demand Elasticity}
	\begin{itemize}
		\item Elasticity of labor demand: A measure of the responsiveness of employment to changes in the wage rate
		\[\varepsilon_{w}^{D,SR} = \frac{\%\Delta E_{SR}}{\%\Delta w} = \frac{\Delta E_{SR}}{\Delta w} \times \frac{w_0}{E_{SR0}}\]
	\end{itemize}
\end{frame}

\begin{frame}{Readings}
\begin{itemize}
	\item Borjas 3.1-3.2
\end{itemize}
\end{frame}


\section{The Long-Run}

	
\begin{frame}{Labor Demand: Long versus Short-Run}
	\begin{itemize}
		\item Recall that in the short-run, the firm's capital stock is fixed at $K = K_0$
		\item In the short-run, the firm chooses $E$ to maximize profits since it takes $K_0$ as given:
		\[\max TR - TC = p q - wE - rK_0 \hspace{2mm}\text{s.t} \hspace{2mm} q = f(K,E)\]
		\item Long-run: Firm is free to vary capital, hence it must choose \textbf{both} the optimal labor level and the optimal capital level:
		\[\max TR - TC = p q - wE - rK \hspace{2mm}\text{s.t} \hspace{2mm} q = f(K,E)\]
	\end{itemize}
\end{frame}

\begin{frame}{Long-Run Optimality Conditions: Labor}
	\begin{itemize}
		\item Should hire labor until $VMP_E = w$
		\item Same intuition as the short-run case: Hire an additional unit of labor if, and only if, the marginal benefit ($VMP$) outweighs the marginal cost ($w$)
	\end{itemize}
\end{frame}

\begin{frame}{Long-Run Optimality Conditions: Capital}
	\begin{itemize} 
		\item Optimal capital hiring must satisfy a similar condition:
		\[VMP_K = p\times MP_K = r\]
		\item Intuition: Hire the next unit of capital if, and only if, the marginal benefit ($VMP$) outweighs the marginal cost ($r$)
	\end{itemize}
\end{frame}

\begin{frame}{Long-Run Optimality Conditions: MRTS Condition}
	\begin{itemize}
		\item Combining the optimality conditions for labor and capital, we get that optimal bundles of labor must satisfy:
		\[\frac{MP_E}{MP_K} = \frac{w}{r}\]
		\item The ratio of $MP_E$ and $MP_K$ is called the \textbf{marginal rate of technical substitution}.
	\end{itemize}
\end{frame}


\begin{frame}{Interpreting the MRTS Condition}
	\begin{itemize}
			\item Thus, in the long-run optimal bundles of labor and capital must satisfy $MRTS_{E,K} = w/r$ 
	\item This condition is largely the reason long-run labor demand behaves differently than short-run labor demand
		\item The intuition behind the $MRTS$ condition can be seen by re-writing it as
		\[\frac{MP_E}{w} = \frac{MP_K}{r}\]
	\end{itemize}
\end{frame}

\begin{frame}{Interpreting the MRTS Condition}
\begin{itemize}
		\item The condition essentially states that the last dollar spent on labor must yield as much output as the last dollar spent on capital
	\item How should we reallocate resources if $MRTS < w/r$?
	\item How should we reallocate resources if $MRTS > w/r$?
\end{itemize}
\end{frame}

\begin{frame}{Long-Run Labor Demand}
	\begin{itemize}
		\item Consider the dynamics of labor demand in response to  $\uparrow w$.
		\item Short-run: $VMP_E = w \Rightarrow \uparrow w \Rightarrow \downarrow E$
		\item Long-run: Free to vary capital. Because $\uparrow w$, now we have:
		\[\frac{MP_E}{MP_K} < \frac{\uparrow w}{r}\]
		so the firm must decrease $E$ \textit{even further} in order to re-balance the $MRTS$ condition
	\end{itemize}
\end{frame}

\begin{frame}{Long-Run Labor Demand Elasticity}
	\begin{itemize}
		\item Long-Run Labor Demand elasticity is the percentage change in the firm's labor level ($E$) which results from a one percent increase in the wage rate ($w$), while the firm is free to vary capital:
		\[\varepsilon_w^{D,LR} = \frac{\%\Delta E^*_{LR}}{\%\Delta w} = \frac{\Delta E^*_{LR}}{\Delta w} \times \frac{w_0}{E^*_{LR0}}\]
	\end{itemize}
\end{frame}

\begin{frame}{Long-Run versus Short-Run Labor Demand Elasticity}
	\begin{itemize}
		\item In the long-run, firm's have the option of substituting between labor and capital
		\item Hence, the firm can effectively replace some of its labor with capital if the wage rate rises. 
		\item Implication: $|\varepsilon_w^{D,LR}| > |\varepsilon_w^{D,SR}|$
		
	\end{itemize}
\end{frame}

\begin{frame}{Long-Run versus Short-Run Labor Demand Elasticity}
\begin{itemize}
	\item In words: Firms are more able to adjust their level of employment in response to changes in wages in the long-run
	\item Effect of policy changes may be different in long-run than short-run. E.g. The minimum wage
	\begin{itemize}
		\item Data indicate that firms do not significantly reduce E in short-run.
		\item Possible that firms do reduce E in the long-run.
	\end{itemize}
	
\end{itemize}
\end{frame}


\begin{frame}{Readings}
	\begin{itemize}
		\item Borjas 3.3-3.4
	\end{itemize}
\end{frame}


\section{Application: The Minimum Wage}

\begin{frame}{The Minimum Wage}
	\begin{itemize}
		\item Introduced by the Fair Labor Standards Act (FSLA) in 1938
		\item Coverage was initially small ($\sim$45\% of nonsupervisory work), but now most workers are covered by the legislation
		\item The minimum wage is updated at irregular intervals and is not indexed to inflation or productivity growth
		\begin{itemize}
			\item Implication: The \textit{real} minimum wage declines between the time it is set and the next time it is raised
		\end{itemize}
	\end{itemize}
\end{frame}

\begin{frame}
	\begin{figure}
		\centering
		\includegraphics[scale=.6]{03C_1.png}
	\end{figure}
\end{frame}

\begin{frame}
	\begin{figure}
		\centering
		\includegraphics[scale=.3]{03C_2.png}
	\end{figure}
\end{frame}

\begin{frame}{The Minimum Wage: Universal Coverage}
	\begin{itemize}
		\item Standard analysis under universal coverage: A minimum wage set above the equilibrium wage leads to a surplus of labor (i.e., unemployment)
		\begin{itemize}
			\item Firms demand less labor at the hire wage
			\item Some workers are displaced from their current job and become unemployed
			\item The higher wage increases the quantity of labor supplied
			\item These additional workers enter the labor market, but cannot find jobs and so increase unemployment 
		\end{itemize}
	\end{itemize}
\end{frame}

\begin{frame}{The Minimum Wage: Universal Coverage}
	\begin{itemize}
		\item For a given minimum wage, the level of employment depends on the elasticity of labor demand 
		\item The level of unemployment depends on the elasticity of labor demand and the elasticity of labor supply
		\item The greater the elasticities of each curve, the greater impact the minimum wage will have on the labor market
		\begin{itemize}
			\item Implication: The effect of the minimum wage will be greater in the long run
		\end{itemize}
	\end{itemize}
\end{frame}


\begin{frame}{The Minimum Wage: Evidence}
	\begin{itemize}
		\item Card \& Krueger, ``Minimum Wages and Employment: A Case Study of the Fast-Food Industry in NJ and PA,'' AER (1994)
		\item New Jersey increased their minimum wage in 1992
		\item Pennsylvania (a neighboring state) kept the minimum wage at the federally mandated level
	\end{itemize}
\end{frame}

\begin{frame}{The Minimum Wage: Evidence}
\begin{itemize}
	\item Control group: Fast food establishments in PA 
	\item Treatment group: Fast food establishments in NJ (same franchises)
	\item Strategy: Difference-in-differences
\end{itemize}
\end{frame}

\begin{frame}{The Minimum Wage: Evidence}
	\begin{figure}
		\centering
		\includegraphics[scale=.6]{03C_3.png}
	\end{figure}
\end{frame}

\begin{frame}{The Minimum Wage: Evidence}
	\begin{figure}
		\centering
		\includegraphics[scale=.6]{03C_4.png}
	\end{figure}
\end{frame}

\begin{frame}{The Minimum Wage: Evidence}
	\begin{itemize}
		\item Neumark \& Wascher (2000) performed the same exercise, but used (probably better) pay-roll record data
		\item Finding: Minimum wage increase lead to a small decrease in employment in NJ relative to PA 
	\end{itemize}
\end{frame}

\begin{frame}{The Minimum Wage: Evidence}
	\begin{itemize}
		\item Card \& Krueger (2000) performed the same exercise, again, but used even more detailed data from BLS
		\item Finding: ``The increase in New Jersey’s minimum wage probably had no effect on total employment in New Jersey's fast-food industry, and possibly had a small positive effect.''
		
	\end{itemize}
\end{frame}

\begin{frame}{The Minimum Wage: Evidence}
	\begin{itemize}
		\item Caveats:
		\begin{enumerate}
			\item Long-run vs. Short-run labor demand
			\begin{itemize}
				\item Long-run impacts may be larger if firms are more able to adjust inputs from labor to capital
			\end{itemize}
			\item Magnitude of minimum wage changes
			\begin{itemize}
				\item Most minimum wage changes are small, so it estimated effects on employment are likely small and difficult to estimate with precision 
			\end{itemize}
			\item Inconsistency of evidence
			\begin{itemize}
				\item Lots of ``noise'' in estimates. On average, estimates probably indicate that there is a fairly small disemployment effect
			\end{itemize}
		\end{enumerate}
	\end{itemize}
\end{frame}

\begin{frame}{The Minimum Wage: Evidence}
	\begin{figure}
		\centering
		\includegraphics[scale=.6]{03C_5.png}
	\end{figure}
\end{frame}

\begin{frame}{Readings}
	\begin{itemize}
		\item Borjas 3.10
	\end{itemize}
\end{frame}
	
\end{document}