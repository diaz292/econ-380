\documentclass[pdf]{beamer}
\usetheme{Frankfurt}  
\usecolortheme{whale}
\usepackage{tikz} 
\usepackage{amsmath}
\usepackage{amsthm}
\usepackage{amssymb}              % used for \eqref{} in this document
\usepackage{dsfont}
\usepackage{hyperref}
\usepackage{threeparttable}
\usepackage{multirow}
\graphicspath{{Figures/}}
\usepackage{booktabs}
\usepackage{tikz}
\newtheorem{exmp}{Example}[section]
\usepackage{subcaption}
\usepackage{adjustbox}
\usepackage{graphicx}
\usepackage[mathscr]{euscript}
\usepackage{remreset}% tiny package containing just the \@removefromreset command
\makeatletter
\@removefromreset{subsection}{section}
\makeatother
\setcounter{subsection}{1}


\section{The Wage Distribution}

%% preamble
\title{Part IV: Inequality}
\author[David A. D\'iaz]{David A. D\'iaz}
\institute{UNC Chapel Hill}
\date{}


\AtBeginSection[] %Section links on slides


\begin{document}
	
	\begin{frame}
		
		\titlepage
		
	\end{frame}
	
	\begin{frame}{Table of Contents}
		
		\tableofcontents
	\end{frame}




\begin{frame}{Introduction}
	\begin{itemize}
		\item Consider a couple of common metrics for economic well-being:
		\begin{itemize}
			\item GDP
			\item Per capita income adjusted for PPP
		\end{itemize}
		\item At best, these can only capture average economic status
		\item Aggregate economic well-being also depends on how GDP is distributed	
	\end{itemize}
\end{frame}

\begin{frame}{Introduction}
	\begin{itemize}
		\item Does per capita GDP Measure Economic well-being?
		\item Undoubtedly, it’s positively correlated with economic wellbeing.  But consider...	
		
		\begin{figure}
			\centering
			\includegraphics[scale=.35]{06A_6}
		\end{figure}
		
		\item Issue: Per Capita GDP does not account for distribution of income!
	\end{itemize}
	
\end{frame}



\begin{frame}{The US Wage Distribution}
	\begin{figure}
		\centering
		\includegraphics[scale=.45]{06A_1}
		\caption{US Wage Distribution, 2012}
	\end{figure}
\end{frame}

\begin{frame}{Measuring Inequality}
	\begin{itemize}
		\item Wage Percentile Differences: Compare wages between different percentiles of the distribution, usually expressed in percentage terms.
		\begin{itemize}
			\item E.g. How much more does a worker in the 90th percentile of the wage distribution earn than one in the 50th percentile of the wage distribution?
			\item E.g. How much more does a worker in the 90th percentile of the wage distribution earn than one in the 10th percentile of the wage distribution?
		\end{itemize} 
	\end{itemize}
\end{frame}

\begin{frame}{Measuring Inequality}
	\begin{itemize}
		\item Gini Coefficient 
		\item Lorenz Curve: Reports cumulative share of income accruing to percentiles of households.
	\end{itemize}
	\begin{figure}
		\centering
		\includegraphics[scale=.45]{06A_7}
	\end{figure}
\end{frame}

\begin{frame}{Measuring Inequality}
	\begin{itemize}
		\item Gini coefficient takes ratio of: Area between perfect equality Lorenz curve (red) and country's Lorenz curve (purple) to 
		area under perfect equality Lorenz curve
		\begin{figure}
			\centering
			\includegraphics[scale=.4]{06A_8}
		\end{figure}
		\item The Gini coefficient is thus defined as:
		\[\frac{\text{Area A}}{\text{Area A + B}}\]
	\end{itemize}
\end{frame}


\begin{frame}{Measuring Inequality}
	\begin{exmp}
		Suppose a simple economy is comprised of 50,000 individuals. Of these individuals, 60\% report an annual income of \$30,000, 30\% report an annual income of \$80,000, and 10\% report an annual income of \$200,000. What is the Gini coefficient associated with this economy?
	\end{exmp}
\end{frame}


\begin{frame}{US Trends}
	\begin{figure}
		\centering
		\includegraphics[scale=.45]{06A_2}
	\end{figure}
\end{frame}

\begin{frame}{US Trends}
	\begin{figure}
		\centering
		\includegraphics[scale=.55]{06A_3}
	\end{figure}
\end{frame}

\begin{frame}{US Trends}
	\begin{figure}
		\centering
		\includegraphics[scale=.55]{06A_4}
	\end{figure}
\end{frame}

\begin{frame}{US Trends}
	\begin{figure}
		\centering
		\includegraphics[scale=.45]{06A_9}
		\caption{US Gini Coefficient, 1970-2005 (Heathcote et al., 2009)}
	\end{figure}
\end{frame}

\begin{frame}{World Comparison}
	\begin{figure}
		\centering
		\includegraphics[scale=.3]{06A_5}
	\end{figure}
\end{frame}


\begin{frame}{Is Inequality Bad?}
	\begin{itemize}
		\item Mainstream economists would generally agree that a perfectly even income distribution is suboptimal.
		\item Big disagreement over exactly how much income inequality is optimal.	
	\end{itemize}
\end{frame}

\begin{frame}{Is Inequality Bad?}
	\begin{itemize}
		\item Cronyism:  The practice of favoring one’s close friends, especially in political appointments.
		\begin{itemize}
			\item Examples:  Political lobbying for favorable industrial treatment, firms hiring less qualified friends/family, etc.
		\end{itemize}
		\item Discrimination
		\begin{itemize}
			\item Wage inequality resulting from differential pay according to characteristics irrelevant to the production process.
		\end{itemize}
		\item Inequality at Birth
		\begin{itemize}
			\item Wage inequality resulting from differential access to schooling, quality of schooling, bad infrastructure, etc.
		\end{itemize}
		\item All of these are associated with economic inefficiencies, not to mention ethical issues. 	
	\end{itemize}
\end{frame}

\begin{frame}{Is Inequality Bad?}
	\begin{itemize}
		\item Existence of wage inequality can also provide strong incentives to increase productivity and output.
		\item Investments in human capital spurred by desire to increase future earnings increase individuals' productivity.
		\item Discovery of new ideas responsible for much economic growth; presence of financial rewards for successful innovation incentivizes valuable innovations.
		\item Key here is that inequality can foster increases in aggregate productivity and output.
		
	\end{itemize}
\end{frame}

\begin{frame}{Is Inequality Bad?}
	\begin{itemize}
		\item Previous argument highlights the basic issue:
		Inequality can foster productive behavior, but it can also foster unproductive (rent-seeking) behavior.
		\item How should policy be written to weigh the costs and benefits?
		\item How much of the inequality in the world today results from each of the (broadly speaking) ``good'' and ``bad'' inequality?
		
	\end{itemize}
\end{frame}

\begin{frame}{Readings}
	\begin{itemize}
		\item Borjas 7.1 - 7.3
		\item Separate and Unequal: ``The Price of Inequality'' by Thomas Edsall, \textit{The New York Times}
		\item Bad and Good Inequality by Gary Becker, \textit{The Becker-Posner Blog}
	\end{itemize}
\end{frame}

\section{Why is Inequality Rising?}

\begin{frame}{Why is Inequality Rising?}
	\begin{itemize}
		\item Any way you slice it: wage inequality has greatly increased in the US over the past thirty years.
		\item Why has this happened?
		\item We'll discuss a few theories which attempt to explain rising inequality in the US.	
	\end{itemize}
\end{frame}


\begin{frame}{Institutional Changes}
	\begin{itemize}
		\item Straightforward argument:  Decreasing tax rate on high income earners $\Rightarrow$ higher levels of post-transfer wage inequality.
		\item Are taxation trends consistent with this?
		
	\end{itemize}
\end{frame}

\begin{frame}{Institutional Changes}
	\begin{itemize}
		\item Tax rates on top earners have decreased significantly post-WWII.
	\end{itemize}
	\begin{figure}
		\centering 
		\includegraphics[scale=.4]{06B_1}
	\end{figure}
\end{frame}

\begin{frame}{Institutional Changes}
	\begin{itemize}
		\item Magnitude of tax changes for the rest of the income distribution is small by comparison.	
	\end{itemize}
	
	\begin{figure}
		\centering 
		\includegraphics[scale=.5]{06B_2}
	\end{figure}
	
\end{frame}

\begin{frame}{Institutional Changes}
	\begin{itemize}
		\item Policy shifts may explain some of the overall changes in the post-transfer wage distribution. 
		\item But, 
		\begin{itemize}
			\item overall level of redistribution is not lower today than in the past; difference in Gini pre-tax vs. post-tax is similar today to the 1980's.
			\item We've seen significant growth in pre-tax inequality
			\item Factors outside of policy can also have major effects
		\end{itemize} 	
	\end{itemize}
\end{frame}

\begin{frame}{Institutional Changes}
	\begin{figure}
		\centering
		\includegraphics[scale=.45]{06B_7}
	\end{figure}
\end{frame}


\begin{frame}{Institutional Changes}
	\begin{figure}
		\centering
		\includegraphics[scale=.3]{06A_5}
	\end{figure}
\end{frame}

\begin{frame}{Institutional Changes}
	\begin{figure}
		\centering
		\includegraphics[scale=.45]{06B_6}
	\end{figure}
\end{frame}

\begin{frame}{Institutional Changes}
	\begin{itemize}
		\item Why is inequality more severe in US than Europe?
		\begin{itemize}
			\item It's almost entirely due to policy differences.
			\item EU spends far more of GDP on redistribution than US.
			\item Pre-Tax Gini coefficients look very similar between US and major European economies.
			\item Post-Tax/Transfer Gini coefficients look very different.
			\item In short, Europe has historically been more aggressive with redistribution.
			
		\end{itemize}
	\end{itemize}
\end{frame}


\begin{frame}{Institutional Changes}
	\begin{figure}
		\centering
		\includegraphics[scale=.55]{07H}
	\end{figure}
\end{frame}

\begin{frame}{Supply Shifts in Skilled Labor Market}
	\begin{itemize}
		\item Basic idea:
		\begin{itemize}
			\item Relative wages of skilled and unskilled workers determined by supply curves for skilled and unskilled workers.
			\item 	Increase in supply of skilled workers $\Rightarrow$ decrease in relative wage of skilled workers $\Rightarrow$ decrease in inequality
			\item Decrease in supply of skilled workers $\Rightarrow$ increase in relative wage of skilled workers $\Rightarrow$ increase in inequality
		\end{itemize}
		\item Need to look at supply shocks over the past 40 years and see if this story makes sense.
		
	\end{itemize}
\end{frame}

\begin{frame}{Supply Shifts in Skilled Labor Market}
	\begin{itemize}
		\item 1970's: Baby boomers $\Rightarrow$ outward supply shift of college graduates
		\begin{itemize}
			\item Result: Decrease in the wage premium paid to college graduates
		\end{itemize}
		\item	1980's: Decrease in supply of college graduates
		\begin{itemize}
			\item Result: Increase in the wage premium paid to college graduates
		\end{itemize}
		\item 1980's: Increase in number of unskilled immigrants
		\begin{itemize}
			\item Result: Increased relative number of workers at the bottom of the skill distribution
		\end{itemize}
		\item This would seem to support the supply shift hypothesis.
		
	\end{itemize}
\end{frame}


\begin{frame}{Supply Shifts in Skilled Labor Market}
	\begin{itemize}
		\item But, what if supply shifts in unskilled labor went the same direction?
		\begin{itemize}
			\item Baby Boomers: Increase in supply of both skilled and unskilled labor
			\item Not necessarily clear that these supply shifts should decrease the wage premium paid to college educated workers
			
		\end{itemize}
		\item Overall, the number of college graduates
		relative to the number of high school graduates continued to rise in the 1980s at the same time that the relative wage of college graduates was rising
		\item Supply shifts may explain some, but not all of rising inequality
	\end{itemize}
\end{frame}

\begin{frame}{Supply Shifts in Skilled Labor Market}
	\begin{figure}
		\centering
		\includegraphics[scale=.9]{06B_3}
	\end{figure}
\end{frame}

\begin{frame}{Skill-Biased Technological Change}
	\begin{itemize}
		\item Over past 40+ years, economy has shifted from a manufacturing economy towards a technology-based economy
		\item New technologies good substitutes for low-skill labor, but complements to high-skill labor	
	\end{itemize}
\end{frame}


\begin{frame}{Skill-Biased Technological Change}
	\begin{itemize}
		\item Ex: Mechanization of industrial production
		\item Decrease in demand for low-skill workers who physically build products
		\item Increase in demand for high-skill workers (engineers, programmers) needed to run new machines
		\item Low-skill workers' wages fall
		\item High-skill workers' wages rise
		\item Increase in income inequality from this		
	\end{itemize}
\end{frame}

\begin{frame}{Skill-Biased Technological Change}
	\begin{itemize}
		\item Prediction:  We should find significant wage growth in those industries which employ advanced technology.
		\item Problem with this explanation:  Tech-intensive industries have seen significant inequality growth.  But, so have industries which are not traditionally thought of as tech-intensive.
		\item In essence, SBTC should cause increase in inequality between industries, but data shows significant increase in inequality within industries.
		
	\end{itemize}
\end{frame}

\begin{frame}{Skill-Biased Technological Change}
	\begin{itemize}
		\item Prediction:  Wage inequality should increase fastest during tech booms.
		\item Ex:  IT Revolution, 1990 - 2000.
		\item Puzzle:  Inequality increased in 1980's and 2000's, but actually did not increase all that much from late '80s to early '00s.
		
	\end{itemize}
\end{frame}

\begin{frame}{Skill-Biased Technological Change}
	\begin{figure}
		\centering 
		\includegraphics[scale=.55]{06B_4}
	\end{figure}
\end{frame}

\begin{frame}{Skill-Biased Technological Change}
	\begin{itemize}
		\item Very popular story of rising income inequality.
		\item But, surprisingly difficult to see the effects of this in the data.
		\item Difficult to say how large the effect of SBTC is on increasing inequality (lots of disagreement among researchers on this topic).
		
	\end{itemize}
\end{frame}

\begin{frame}{Labor Force Composition}
	\begin{itemize}
		\item Autor, Katz, and Kearney (2005): Rising Wage Inequality: The Role of Composition and Prices
		\item Consider the following:
		\begin{itemize}
			\item Older workers (age 40 - 60) generally have greater variance in wages.
			\item Younger workers (age 20 - 40) generally have lower variance in wages.
			\item Between 1970 and 1990, the proportion of Americans in the older group rose, while the proportion in the younger group fell.
		\end{itemize}
		\item This type of ``composition effect'' could drive wage inequality, as economy composed of more individuals in high wage variance group.
		
	\end{itemize}
\end{frame}

\begin{frame}{Labor Force Composition}
	\begin{figure}
		\centering
		\includegraphics[scale=.7]{06B_5}
	\end{figure}
\end{frame}

\begin{frame}{Labor Force Composition}
	\begin{itemize}
		\item Consider other groups which have different wage variances:
		\begin{itemize}
			\item College graduates: High wage variance
			\item HS graduates: Lower wage variance
		\end{itemize}
		\item 1970: 70\% had at most HS education, 14\% had college degree
		\item 1996: 40\% had at most HS education, 28\% had college degree
		\item Once again: Economy composed of greater proportion of high wage variance group $\Rightarrow$ increase in wage inequality
		
	\end{itemize}
\end{frame}

\begin{frame}{Labor Force Composition}
	\begin{itemize}
	\item Composition effects might explain bulk of increase in inequality if wage variance within education groups were similar now and 40 years ago.
\item Problem with this explanation:
\item Most of the increase in variation can be attributed to between-group price changes
\item Both (i) the premium and (ii) the variance of wages paid to college graduates has risen greatly over the past 40 years.
\item So how much is due to composition effects?	
	\end{itemize}
\end{frame}



\begin{frame}{Labor Force Composition}
	\begin{itemize}
		\item Autor, Katz, and Kearney:
		\item A decent amount of the increase in the lower tail (50/10) wage gap is explained by composition effects.
		\item very little of the increase in the upper tail (90/50) wage gap is explained by composition effects.
		\item Composition does seem to have an effect, but can't explain the rise in upper tail income inequality.
		
	\end{itemize}
\end{frame}

\begin{frame}{Readings}
	\begin{itemize}
		\item Borjas 7.4
		\item Autor, Katz, and Kearney (2005). Rising Wage Inequality: The Role of Composition and Prices. \textit{NBER Working Paper No. 11628}
	\end{itemize}
\end{frame}

\section{Superstars and Intergenerational Inequality}



\begin{frame}{Occupational Wage Dispersion}
	\begin{itemize}
		\item Previous analysis tended to focus on economy-wide wage inequality.
		\item Or, inequality within demographic groups (composition effects and age).
		\item A finer question:  Why do some occupations exhibit far more inequality than others?
		
	\end{itemize}
\end{frame}

\begin{frame}{Occupational Wage Dispersion}
	\begin{itemize}
		\item Motivating feature of the data:
		Biggest growth in inequality is captured by growth in wages of extreme right tail, i.e., top 0.1\%, top 0.01\%, or top 400 families
		\item Earlier theories mostly explain growth in wage dispersion between $\sim$5th percentile and 95th percentile, not the extremes.
		\item Fundamental question:  How is it possible for an individual to be worth 50x, 100x, 1000x times as much as a typical individual to some firm?
		
	\end{itemize}
\end{frame}

\begin{frame}{Occupational Wage Dispersion}
	\begin{itemize}
		\item Certain industries have relatively low wage dispersion. 
		\begin{itemize}
			\item 	Entry-level PhD economist:  $\sim$\$100,000
			\item 	Nobel-laureate economist:  $\sim$\$300,000 - \$400,000
		\end{itemize}
		\item Top 0.1\% earn roughly 3x wage of ``median'' economist
	\end{itemize}
\end{frame}

\begin{frame}{Occupational Wage Dispersion}
	\begin{itemize}
		\item Other industries have massive wage dispersion.  
		\begin{itemize}
			\item Barcelona's Lionel Messi: \$53,400,000
			\item New York City FC's Tommy McNamara (\$85,000).
		\end{itemize}
		\item We refer to such individuals as ``superstars'' in labor economics; workers who command a wage 50x, 100x, 1000x what their peers receive.
		
	\end{itemize}
\end{frame}

\begin{frame}{Superstars}
	\begin{figure}
		\centering
		\includegraphics[scale=.5]{06C_1}
	\end{figure}
\end{frame}


\begin{frame}{Superstars}
	\begin{itemize}
		\item Many other industries have superstar wage earners.  Classic example is entertainment (including sports) industry:
		\begin{itemize}
			\item Robert Downey Jr $\sim$\$80M 
			\item Jennifer Lawrence $\sim$\$55M 
			\item Taylor Swift $\sim$\$40M
			\item Aaron Rodgers:  $\sim$\$22M
		\end{itemize}
		\item What characteristics do these occupations share?
	\end{itemize}
\end{frame}

\begin{frame}{Superstars}
	\begin{itemize}
		\item Increasing returns to scale
		\item Compare: 
		\begin{itemize}
			\item Cost function for Ford when producing F-250s
			\item Cost function for NFL when streaming games
		\end{itemize}
		\item This characteristic essentially enables top talent to distribute their skills to a wider audience.
		
	\end{itemize}
\end{frame}

\begin{frame}{Superstars}
	\begin{itemize}
		\item Large demand differentials
		\begin{itemize}
			\item Willingness to pay for BMW 3-Series vs. Honda Accord: ``First-Best'' only costs about 50\% more than ``Second-Best''
			\item Willingness to pay for football tickets: 	
			\begin{itemize}
				\item Manchester City vs Borussia Monchengladbach (Champions League): Starting at $\sim$ \$100 
				\item Manchester City vs Middlesborough (English Premier League): Starting at $\sim$ \$55
				\item Average League Two tickets: $\sim$ \$17
			\end{itemize}
		\end{itemize}
		\item 	Industries with ``superstars'' tend to exhibit significant difference in willingness to pay between top products and other products.
	\end{itemize}
\end{frame}

\begin{frame}{Superstars}
	\begin{itemize}
		\item How important is this? Could it explain much of the wage growth of the top 0.1\%?
		\item Maybe some. But bulk of growth seems to be due to increasing wages of business tycoons, CEO's, etc.
	\end{itemize}
\end{frame}

\begin{frame}{Inequality Across Generations}
	\begin{itemize}
		\item Basic question:  How mobile is income across generations?
		\item Relates back to question of whether inequality has resulted because of
		\begin{itemize}
			\item Enterprising behavior (schooling investments, etc.)
			\item Cronyism
		\end{itemize}	
	\end{itemize}
\end{frame}


\begin{frame}{Inequality Across Generations}
	\begin{figure}
		\centering
		\includegraphics[scale=.55]{07J}
	\end{figure}
\end{frame}

\begin{frame}{Inequality Across Generations}
	\begin{figure}
		\centering
		\includegraphics[scale=.55]{07K}
	\end{figure}
\end{frame}

\begin{frame}{Inequality Across Generations}
	\begin{itemize}
		\item Human capital investment decisions often made by parents
		\item High-income parents typically invest more $\Rightarrow$ positive correlation between socioeconomic outcomes of the parents and the outcomes of their children
		\item Regression towards the mean: A tendency for income differences across families to decrease over time as families move toward the mean income in the population
		\begin{itemize}
			\item Not all parental income is invested in children
			\item Diminishing returns to education
			\item Regression toward the mean in ability
		\end{itemize}
	\end{itemize}
\end{frame}



\begin{frame}{Inequality Across Generations}
	\begin{itemize}
		\item Intergenerational Earnings Elasticity: Gives estimated effect of one percent rise in parents' earnings on child's earnings.
		\item Recent studies: intergenerational coefficient of $\sim 0.3-0.4$.
	\end{itemize}
\end{frame}


\begin{frame}{Inequality Across Generations}
	\begin{exmp}
		Consider two families, one with a yearly income of \$60,000 and the other with a yearly income of \$108,000. If the intergenerational correlation coefficient is 0.3, what is the expected difference in earnings of the children in the two families? What if it was 0.6?
	\end{exmp}
\end{frame}



\begin{frame}{Readings}
	\begin{itemize}
		\item Borjas 7.5-7.6
		\item American Inequality in Six Charts by John Cassidy, \textit{The New Yorker}.
		\item Krueger, Alan (2005). The Economics of Real Superstars: The Market for Rock Concerts in the Material World. \textit{Journal of Labor Economics}.
		\item How Superstars' Pay Stifles Everyone Else, \textit{The New York Times}
		\item Gabaix, Xavier \& Augustin Landier (2007). Why has CEO Pay Increased So Much? \textit{The Quarterly Journal of Economics}.
	\end{itemize}
\end{frame}
	
\end{document}