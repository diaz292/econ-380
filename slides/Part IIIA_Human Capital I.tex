\documentclass[pdf]{beamer}
\usetheme{Frankfurt}  
\usecolortheme{whale}
\usepackage{tikz} 
\usepackage{amsmath}
\usepackage{amsthm}
\usepackage{amssymb}              % used for \eqref{} in this document
\usepackage{dsfont}
\usepackage{hyperref}
\usepackage{threeparttable}
\usepackage{multirow}
\graphicspath{{Figures/}}
\usepackage{booktabs}
\usepackage{tikz}
\newtheorem{exmp}{Example}[section]
\usepackage{subcaption}
\usepackage{adjustbox}
\usepackage{graphicx}
\usepackage[mathscr]{euscript}
\usepackage{remreset}% tiny package containing just the \@removefromreset command
\makeatletter
\@removefromreset{subsection}{section}
\makeatother
\setcounter{subsection}{1}


\section{Human Capital}

%% preamble
\title{Part IIIA: Human Capital Models}
\author[David A. D\'iaz]{David A. D\'iaz}
\institute{UNC Chapel Hill}
\date{}


\AtBeginSection[] %Section links on slides


\begin{document}
	
	\begin{frame}
		
		\titlepage
		
	\end{frame}
	

\begin{frame}{Human Capital}
\begin{itemize}
	\item So far, we have largely discussed competitive labor markets where there was a single equilibrium wage
	\begin{itemize}
		\item Assumption: Homogeneous firms and workers 
	\end{itemize}
	\item Real-world labor markets: workers and jobs are different
	\begin{itemize}
		\item Jobs differ in their characteristics
		\item Workers differ in their skills
	\end{itemize}
	\item Thus, wages will vary among workers
\end{itemize}
\end{frame}

\begin{frame}{Human Capital Intro}
\begin{itemize}
	\item Human capital: The set of skills and abilities each individual brings to the workforce
	\item Basic idea: Through education, training, experience, we acquire knowledge and skills that increase our productivity
	\item Key questions:
	\begin{itemize}
		\item Why do some workers obtain a lot of schooling?
		\item Is the money spent on schooling a good investment?
		\item How much are individuals compensated for schooling?
		\item Why are they compensated for schooling?
	\end{itemize}
\end{itemize}
\end{frame}

\begin{frame}{Human Capital Intro}
	\begin{itemize}
		\item Major topic in labor econ:  Returns to skill
		\begin{itemize}
			\item Education
			\item Certification \& training (CPA, RN, etc.) 
			\item Experience 
		\end{itemize}
	\end{itemize}
	\begin{figure}
		\centering
		\includegraphics[scale=.4]{01C_7.png}
	\end{figure}
\end{frame}

\begin{frame}{Education in the U.S.}
\begin{figure}
	\centering
	\includegraphics[scale=.35]{05_1.png}
\end{figure}
\end{frame}

\begin{frame}{Education in the U.S.}
	\begin{figure}
		\centering
		\includegraphics[scale=.5]{05_2.png}
	\end{figure}
\end{frame}

\begin{frame}{Education in the U.S.}
	\begin{figure}
		\centering
		\includegraphics[scale=.5]{05_3.png}
		\caption{Labor Market Characteristics by Education Group, 2013}
	\end{figure}
\end{frame}

\begin{frame}{Education in the U.S. - Trends}
	\begin{figure}
	\centering
	\includegraphics[scale=.4]{05A_4.png}
\end{figure}
\end{frame}


\begin{frame}{Education in the U.S. - Trends}
	\begin{figure}
	\centering
	\includegraphics[scale=.4]{05A_6.png}
\end{figure}
\end{frame}


\begin{frame}{Education in the U.S. - Trends}
	\begin{figure}
		\centering
		\includegraphics[scale=.3]{05A_7.png}
	\end{figure}
\end{frame}


\begin{frame}{Education in the U.S. - Trends}
	\begin{figure}
		\centering
		\includegraphics[scale=.4]{05A_8.png}
	\end{figure}
\end{frame}


\begin{frame}{Education in Other Developed Nations}
	\begin{figure}
		\centering
		\includegraphics[scale=.7]{05A_9.png}
	\end{figure}
\end{frame}

\begin{frame}{Education in Other Developed Nations}
	\begin{figure}
		\centering
		\includegraphics[scale=.7]{05A_10.png}
	\end{figure}
\end{frame}

\begin{frame}{Education Trends}
\begin{itemize}
	\item As economists, great interest in understanding
	\begin{itemize}
		\item Why are individuals with college degrees paid more?
		\item Why have tuition and debt have risen faster than inflation?
		\item Why does attendance continue to rise, despite rising costs?
	\end{itemize}
\end{itemize}
\end{frame}

\begin{frame}{Human Capital Theory - Intro}
	\begin{itemize}
		\item Gary Becker: Choose educational attainment based on long-term costs and benefits
		\item Investments often give payoffs far into the future 
		\item Decisions must take account of ``time-value'' of money
		\item Example: Give up \$1,000 today, get \$1,100 next year. Good deal?
		\item Depends on individual's time-preference
	\end{itemize}
\end{frame}

\begin{frame}{Human Capital Theory - Intro}
\begin{itemize}
	\item Present-value of payment $X$ received $t$ years in the future:
	\[PV_t = \frac{X}{(1+r)^t}\]
	\item Net present value of income ``stream'' with payouts at $t=0,\dots,T$
	\[NPV_t = \sum_{t=0}^{T}\frac{X}{(1+r)^t}\]
	\item Important: $t$ represents the number of periods we are discounting, not necessarily the period the payment is received.
	\item $r$ is the ``rate of return,'' ``discount rate,'' or ``interest rate''
	\item $\uparrow  r \Rightarrow \downarrow PV/NPV$
\end{itemize}
\end{frame}

\begin{frame}{Human Capital Theory - Intro}
\begin{itemize}
	\item Consider a simple two-period model:
	\begin{itemize}
		\item Individuals can borrow/lend at rate $r\ge 0$
		\item College educated workers are paid $W_C$ at the \underline{beginning} of each period they work
		\item Non-college educated workers are paid $W_N$ at the \underline{beginning} of each period they work
		\item Cost of college is \$$c$ and is paid at beginning of period before starting school
	\end{itemize}
\end{itemize}
\end{frame}

\begin{frame}{Human Capital Theory - Intro}
\begin{itemize}
	\item Value of attending college:
	\[V(C) = -c + \frac{W_C}{(1+r)}\]
	\item Value of not attending college:
	\[V(N) = W_N + \frac{W_N}{(1+r)}\]
	\item What would change if payments were made at the end of each period?
	\item Individual only attends college if $V(C) \ge V(N)$
	\item Relationship between $r$ and college choice?
\end{itemize}
\end{frame}

\begin{frame}{Human Capital Theory - Intro}
		\begin{exmp} 
			\tiny (Borjas 6.1) Debbie is about to choose a career path. She has narrowed her options to two alternatives.
			She can become either a marine biologist or a concert pianist. Debbie lives
			two periods. In the first, she gets an education. In the second, she works in the labor
			market. If Debbie becomes a marine biologist, she will spend \$15,000 on education
			in the first period and earn \$472,000 in the second period. If she becomes a
			concert pianist, she will spend \$40,000 on education in the first period and then earn
			\$500,000 in the second period. All payments are made at the beginning of each period.
			\begin{enumerate}
				\item Suppose Debbie can lend and borrow money at a 5 percent rate of interest
				between the two periods. Which career will she pursue? What if she can lend
				and borrow money at a 15 percent rate of interest? 
				\item Suppose musical conservatories raise their tuition so that it now costs Debbie
				\$60,000 to become a concert pianist. What career will Debbie pursue if the interest rate is 5 percent?
			\end{enumerate}
		\end{exmp}
\end{frame}

\begin{frame}{Human Capital Theory - Intro}
\begin{itemize}
	\item What would change if payments were made at the end of each period?
	\item Relationship between $r$ and college choice?
\end{itemize}
\end{frame}


\begin{frame}{Readings}
\begin{itemize}
	\item Borjas 6.1-6.2
\end{itemize}
\end{frame}
	
\section{The Schooling Model}



\begin{frame}{The Schooling Model}
	\begin{itemize}
		\item If higher levels of education are associated with higher earnings, why don't all workers choose to get the highest level of education possible?
		\item Goal: Develop a model of lifetime schooling accumulation
		\begin{itemize}
			\item When is the optimal stopping point?
			\item How does ability affect optimal schooling attainment?
			\item Assumption: Workers acquire education level that maximizes their utility from the present value of lifetime earnings
		\end{itemize}
	\end{itemize}
\end{frame}

\begin{frame}{The Schooling Model}
	\begin{itemize}
		\item Choice: How much time to spend in school
		\item Individuals have rational preferences over wages, $U(w)$, where $MU_w > 0$ (i.e., $\uparrow w \Rightarrow \uparrow U(w)$)
		\item Implication:
		\begin{itemize}
			\item Individuals do not receive any utility from schooling outside of its effect on wages
		\end{itemize}
	\end{itemize}
\end{frame}

\begin{frame}{The Schooling Model}
	\begin{itemize}
		\item In reality, expected future wages are just one of the many determinants of our schooling path
		\item Other costs/benefits of schooling
		\begin{itemize}
			\item ``Nicer'' occupations
			\item Psychological costs of schooling
			\item Social benefits
			\item Etc.
		\end{itemize}
		\item A more general approach would assume workers choose education/skill level to maximize lifetime utility
		\item For now, only focus on monetary rewards of school
	\end{itemize}
\end{frame}

\begin{frame}{The Schooling Model}
	\begin{itemize}
		\item The market wage individuals receive is a function of an individual's ability level ($A$) and level of schooling ($S$)
		\[ w = w(S;A)\]
		\item $S$ is an individual choice
		\item $A$ is taken as given by each individual
	\end{itemize}
\end{frame}

\begin{frame}{The Schooling Model}
	\begin{itemize}
		\item What is ability?
		\item Hard to define
		\item Generally, can be thought of as some uncontrollable stock of raw talent each individual is born with
		\item Key: Exogenously determined, so individuals take this as given
	\end{itemize}
\end{frame}

\begin{frame}{The Schooling Model}
	\begin{itemize}
		\item Wage-schooling locus: Shows salary level employers are willing to pay a particular worker for every level of schooling
		\item Notice that this may vary across workers
		\item The slope of the locus tells us the wage increase associated with an additional year of schooling
		\item Properties:
		\begin{itemize}
			\item Locus is upward sloping
			\item Locus is concave (diminishing returns to education)
		\end{itemize}
	\end{itemize}
\end{frame}

\begin{frame}{The Schooling Model}
	\begin{itemize}
		\item Using wage-schooling locus, can define the marginal benefit from one additional year of schooling similar to a ``return on investment''
		\[MB \equiv MRR = \frac{\%\Delta w}{\Delta S} = \frac{\Delta w}{\Delta S} \times \frac{1}{w_0} \]
		\item Properties:
		\begin{itemize}
			\item $MRR > 0$
			\item $MRR$ is decreasing
		\end{itemize}
	\end{itemize}
	
\end{frame}

\begin{frame}{The Schooling Model}
	\begin{itemize}
		\item What is the marginal cost of schooling?
		\item Harder to define in practice
		\item Suppose tuition is \$20,000/year, and financed as follows:
		\begin{itemize}
			\item \$10,000/year paid by outside source (parents/college fund/etc.)
			\item \$1,000/year paid by work-study job
			\item \$9,000/year paid through student loans
		\end{itemize}
		\item What is the ‘cost’ borne by the student?  When do they pay this?
		\begin{itemize}
			\item Small amount (\$1,000) paid through working during school.
			\item Large amount (\$9,000/year) paid throughout life after school.
		\end{itemize}
	\end{itemize}
\end{frame}

\begin{frame}{The Schooling Model}
	\begin{itemize}
		\item To simplify, suppose each worker has a constant rate of discount $r$:
		\[MC \equiv r\]
		\item Intuition: Higher interest rate $\Rightarrow$ higher opportunity cost of delaying earnings
		\item For our purposes, we will assume the discount rate for individuals is equal to the market interest rate offered by financial institutions 
	\end{itemize}
\end{frame}

\begin{frame}{The Schooling Model}
	\begin{itemize}
		\item Decision rule: Continue schooling until $MRR = r$
		\item What if an individual stopped at a schooling level where $MRR > r$?
		\item What if an individual continued schooling at levels where $MRR < r$?
	\end{itemize}
\end{frame}

\begin{frame}{The Schooling Model}
	\begin{exmp}
		(Borjas 6.6) Suppose Carl's wage-schooling locus is given by
		\begin{table}[H]
			\centering
			\begin{tabular}{cc}
				Years of Schooling & Earnings  \\
				\midrule
				9 & \$18,500\\
				10 & \$20,350\\
				11 & \$22,000\\
				12 & \$23,100\\
				13 & \$23,900\\
				14 & \$24,000\\
			\end{tabular}
			\label{SA1}
		\end{table}
		When will Carl quit school if his discount rate is 4 percent? What if the discount rate is 9 percent?
	\end{exmp}
\end{frame}	




\begin{frame}{Readings}
	\begin{itemize}
		\item Borjas 6.3
	\end{itemize}
\end{frame}

\section{The Returns to Schooling}

\begin{frame}{Schooling Model - Recap}
	\begin{itemize}
		\item Recall the schooling model from last time:
		\begin{itemize}
			\item Individuals choose their schooling level, $S$
			\item Wages are a function of schooling and ability: $w = w(S;A)$, where ability is taken as given
			\item Each particular individual continues to acquire schooling until their $MRR$ (determined by \textit{their own} wage-schooling locus) equals their discount rate $r$
		\end{itemize}
		\item Implication: Individuals will differ in their school attainment due to differences in either their (i) discount rate or (ii) $MRR$ schedules (and potentially both)
	\end{itemize}
\end{frame}


\begin{frame}{Differences in the Rate of Discount}
	\begin{itemize}
		\item Consider two workers, $A$ and $B$ who face the same $MRR$ curve, but differ in their rates of discount
		\item Without loss of generality, assume $r_A > r_B$
		\item Which individual will attain more schooling?
	\end{itemize}
\end{frame}


\begin{frame}{Differences in the Rate of Discount}
	\begin{itemize}
		\item As we noted earlier, a higher rate of discount will decrease schooling years
		\begin{itemize}
			\item Individuals with a higher discount rate do not value future earnings as highly and are more ``present-oriented'' 
		\end{itemize}
		\item Thinking about $r$ as an interest rate, the intuition is:
		\begin{itemize}
			\item Higher $r \Rightarrow$ higher cost of loans
			\item Higher $r \Rightarrow$ greater cost of forgone interest income from earnings
		\end{itemize}
		\item Standard price effect: $\uparrow r \Rightarrow \downarrow S$
	\end{itemize}
\end{frame}


\begin{frame}{Differences in the Rate of Discount}
	\begin{itemize}
		\item Since we are assuming the two individuals face the same $MRR$ schedule, this implies they have the same wage-schooling locus
		\item Thus, differences in $r$ simply put the workers at two different points on their \textit{common} locus
	\end{itemize}
\end{frame}

\begin{frame}{Differences in the Rate of Discount}
\begin{itemize}
	\item Individual $B$ attains more schooling and ends up at a higher point on the locus, where she earns more
	\item Implication: We can estimate the rate of return to schooling from the observed (school choice, wage outcome) for each individual without bias
	\begin{itemize}
		\item Predicted impacts of a given policy would be correct
	\end{itemize}
\end{itemize}
\end{frame}

\begin{frame}{Differences in Ability}
	\begin{itemize}
		\item Now, suppose two individuals, $C$ and $D$ have the same rate of discount, but individual $C$ has a higher ability level: $A^C > A^D$
		\item Assumption: Individuals with a higher ability level have higher returns on education 
	\end{itemize}
\end{frame}

\begin{frame}{Differences in Ability}
\begin{itemize}
	\item In effect, the higher ability individual will have an $MRR$ schedule to the right of a lower ability individual
	\begin{itemize}
		\item Equivalently, we can say their wage-schooling locus is above that of a lower ability individual
	\end{itemize}
	\item Implications for schooling decision?
	\begin{itemize}
		\item Higher return, same discount rate $\Rightarrow$ High ability individual chooses more schooling
		\item Lower  return, same discount rate $\Rightarrow$ Low ability individual choose less schooling 
	\end{itemize}
\end{itemize}
\end{frame}


\begin{frame}{Differences in Ability}
	\begin{itemize}
		\item Recall that ability is seldom observed
		\item Data would show us $(S_C, w_C)$ and ($S_D, w_D)$, but not $A^C$ or $A^D$
		\item Naive estimate of rate of return to schooling:
		\[\widehat{MRR} = \frac{w_C - w_D}{s_C - s_D} \times \frac{1}{w_D}\]
	
	\end{itemize}
\end{frame}


\begin{frame}{Differences in Ability}
\begin{itemize}
	\item However, this assumes that both individuals are on the \textit{same} wage-schooling locus, which we know is not true
	\item Thus, the observed data on earnings and schooling do not allow us to accurately estimate the rate of return to schooling (i.e., the estimate is biased)
\end{itemize}
\end{frame}

\begin{frame}{Ability Bias}
	\begin{itemize}
		\item The differences in the earnings of individuals $C$ and $D$ were due to 
		\begin{enumerate}
			\item Differences in schooling
			\item Differences in innate ability 
		\end{enumerate}
		\item Because ability is unobserved, earnings differentials across workers do not accurately estimate the returns to education (\textbf{ability bias})
	\end{itemize}
\end{frame}

\begin{frame}{Ability Bias}
\begin{itemize}
	\item Direction of bias?
	\begin{enumerate}
		\item Higher ability individuals face lower costs of schooling (less effort) and obtain more schooling $\Rightarrow$ upward bias
		\item Higher ability individuals may be more productive and earn more independent of schooling $\Rightarrow$ downward bias
	\end{enumerate}
\end{itemize}
\end{frame}

\begin{frame}{Estimating Returns to Schooling}
	\begin{itemize}
		\item Raw premium for college: $\sim$80\% for four-year degree in 2015 \begin{itemize}
			\item $\Rightarrow \sim$ 16\% return on education per year (unrealistic)
		\end{itemize}
		\item General empirical model used to estimate returns to education: 
		\[\log wage = \beta_0 + \beta_1 S + \text{other variables}\]
		\item $\widehat{\beta_1}$ is the estimate of the returns to education
		\item Controlling for race, gender, industry, and other observable factors likely influencing wages, the estimate is around $\sim$11-12\% per year
	\end{itemize}
\end{frame}

\begin{frame}{Estimating Returns to Schooling}
	\begin{itemize}
		\item How to account for ability bias?
		\begin{itemize}
			\item Including IQ, AFQT as proxy for ability. Good solution?
			\item Twin or sibling studies to control for unobserved ability. Issues?
			\item Instrumental variables for schooling
			\begin{itemize}
				\item Need to find ``instrument'' (e.g., a policy) that is correlated with schooling, but uncorrelated with any unobserved characteristics that affect wages
				\item Examples: Compulsory schooling laws, school construction initiatives, tuition changes, college proximity 
				\item Potential issues?
			\end{itemize}
		\end{itemize}
		\item Controlling for relevant factors \textit{and} ability bias, estimate is around $\sim$9-11\% per year 
	\end{itemize}
\end{frame}

\begin{frame}{Estimating Returns to Schooling - Other Issues}
	\begin{itemize}
		
		\item Variable choice: What should we include in ``other variables''?
		\begin{itemize}
			\item Typically included: Experience, experience squared, occupation, etc.
			\item Potentially omitted: ability variable
		\end{itemize}
		\item Measurement error
		\begin{itemize}
			\item Earnings, schooling, and other variables may be measured with error 
		\end{itemize}
		\item Selection bias
		\begin{itemize}
			\item Self-selection into schooling is not random
			\item Self-selection into labor force is not random
		\end{itemize}
		\item Reverse causality
		\begin{itemize}
			\item School choice may respond to anticipated wages
		\end{itemize}
	\end{itemize}
\end{frame}

\begin{frame}{Estimating Returns to Schooling - Other Issues}
	\begin{itemize}
		\item Distributional effects
		\begin{itemize}
			\item $\beta_1$ assumed constant in empirical model $\Rightarrow$ estimating the average return to college for all individuals
			\item Do returns to schooling vary across the wage distribution?
		\end{itemize}
		\item Time effects
		\begin{itemize}
			\item Do returns to schooling vary over time and over the life cycle?
		\end{itemize}
		\item Are we observing an individual's true optimal school choice?
		\begin{itemize}
			\item Remember individuals choose $S^*$ such that $MRR = r$ at $S^*$
			\item This assumes individuals know the true $MRR$
			\item Is this a reasonable assumption?
		\end{itemize}
		\item What do returns to schooling measure?
		\begin{itemize}
			\item Productivity effect of schooling?
			\item Signaling value of schooling on the labor market?
			\item Both?
		\end{itemize}
	\end{itemize}
\end{frame}

\begin{frame}{Readings}
	\begin{itemize}
		\item Borjas 6.4-6.5
	\end{itemize}
\end{frame}


	
\end{document}