\documentclass[pdf]{beamer}
\usetheme{Frankfurt}  
\usecolortheme{whale}
\usepackage{tikz} 
\usepackage{amsmath}
\usepackage{amsthm}
\usepackage{amssymb}              % used for \eqref{} in this document
\usepackage{dsfont}
\usepackage{hyperref}
\usepackage{threeparttable}
\usepackage{multirow}
\graphicspath{{Figures/}}
\usepackage{booktabs}
\usepackage{tikz}
\newtheorem{exmp}{Example}[section]
\usepackage{subcaption}
\usepackage{adjustbox}
\usepackage{graphicx}
\usepackage[mathscr]{euscript}
\usepackage{remreset}% tiny package containing just the \@removefromreset command
\makeatletter
\@removefromreset{subsection}{section}
\makeatother
\setcounter{subsection}{1}


\section{Signaling}

%% preamble
\title{Part IIIB: Signaling Model; HC and Development}
\author[David A. D\'iaz]{David A. D\'iaz}
\institute{UNC Chapel Hill}
\date{}


\AtBeginSection[] %Section links on slides


\begin{document}
	
	\begin{frame}
		
		\titlepage
		
	\end{frame}



\begin{frame}{Review: Human Capital Theory}
	\begin{itemize}
		\item Firms observe individual productivity 
		\item Wages are higher for more productive individuals; $w = w(S;A)$
		\item Attend school $\Rightarrow$ increase productivity $\Rightarrow$ increase wages
		\item Individuals attend school only if schooling raises their productivity
	\end{itemize}
\end{frame}

\begin{frame}{The Importance of Information}
	\begin{itemize}
		\item What do firms know about a job candidate?
		\begin{itemize}
			\item Education
			\item Previous experience
			\item Information from references
		\end{itemize}
		\item What do candidates know about themselves?
		\begin{itemize}
			\item All of the above, plus
			\item Work ethic
			\item Intelligence
			\item etc.
		\end{itemize}
	\end{itemize}
\end{frame}

\begin{frame}{The Importance of Information}
	\begin{itemize}
		\item Basic idea of signaling:
		\begin{itemize}
			\item Workers try to present employers with information that makes them look as productive as possible (``signals of productivity'')
			\begin{itemize}
				\item Signaling may be costly: Time spent in school, unpaid internships, extracurriculars, etc.
			\end{itemize}
			\item Firms hiring workers try to take signals presented by potential worker to estimate productivity
		\end{itemize}
	\end{itemize}
\end{frame}

\begin{frame}{Job Market Signaling}
	\begin{itemize}
		\item Is it possible to describe an economic environment where
		\begin{enumerate}
			\item schooling does not directly increase productivity and
			\item firms pay well-educated workers higher wages?
		\end{enumerate}
		\item Yes!
		\item Note: This does not mean schooling doesn't increase productivity. Rather, it just means that increasing productivity is not the only way schooling can increase wages
	\end{itemize}
\end{frame}

\begin{frame}{Job Market Signaling}
	\begin{itemize}
		\item Suppose there are two types of workers:
		\begin{itemize}
			\item High productivity make up a proportion $(1-q)$ of the population
			\item Low productivity make up $q$ of the population
		\end{itemize}
		\item Productivity differences between the workers exist since birth and \textit{do not depend on how much schooling a worker gets}
		\item For illustrative purposes, assume high-ability workers have a present value of lifetime productivity = \$300,000
		\item Low-ability workers have PV of lifetime productivity = \$200,000
	\end{itemize}
\end{frame}

\begin{frame}{Job Market Signaling}
	\begin{itemize}
		\item Case 1: Perfect information
		\begin{itemize}
			\item Employers observe which workers are high-ability and low-ability
			\item Pay high-ability workers \$300,000, low-ability workers \$200,000 over the life cycle
			\item Not often the case. In real-world, there is \textbf{asymmetric information}
		\end{itemize}
		
	\end{itemize}
\end{frame}

\begin{frame}{Job Market Signaling}
	\begin{itemize}
		\item Case 2: ``Pooling equilibrium''
		\begin{itemize}
			\item No information about productivity available to firms and they do not observe a worker's type
			\item Firms will pool all applicants together and treat them equally
			\item Average wage will be a weighted average of the two worker type productivities: \\
			\text{Average salary} = $(300,000)(1-q) + 200,000q = 300,000 - 100,000q$
			\item Good for low-ability workers, bad for high-ability workers and firms
		\end{itemize}
	\end{itemize}
\end{frame}

\begin{frame}{Job Market Signaling}
	\begin{itemize}
		\item Case 3: ``Separating Equilibrium''
		\begin{itemize}
			\item Firms do not observe a worker's type
			\item High-productivity workers may be able to signal their productivity to firms through some form of information (e.g., a threshold education level)
			\item Assumption: Costs to obtain signaling information is higher for low-productivity workers
			\begin{itemize}
				\item Merit-based scholarships awarded to high productivity individuals
				\item Lower-productivity individuals may need to put in greater effort to obtain degree
				\item Lower-productivity individuals may incur greater psychic cost to obtain degree
			\end{itemize}
		\end{itemize}
	\end{itemize}
\end{frame}

\begin{frame}{Job Market Signaling}
	\begin{itemize}
		\item Assume costs to obtain a degree are as follows
		\begin{itemize}
			\item \$80,000 for high-productivity individuals
			\item \$160,000 for low-productivity individuals 
		\end{itemize}
		\item Firms pay \$300,000 to those with a degree, \$200,000 to those without a degree
		\item Is it worth it for individuals to get a degree?
		\begin{itemize}
			\item Yes, but only for the high-productivity type!
		\end{itemize}
		\item Education level is a perfect signal of worker productivity, even though firms do not actually observe this
	\end{itemize}
\end{frame}

\begin{frame}{Job Market Signaling}
	\begin{exmp}
		Suppose that there are two types of workers in an economy, low-productivity and high-productivity. Firms follow the rule of thumb that workers who obtain at least $\bar{y}$ of college are assumed to be highly productive and are paid a lifetime salary of \$400,000. Workers with less than $\bar{y}$ years of education are assumed to be low-productivity and are paid \$250,000. High-productivity workers have a cost of \$25,000 for each year of college, while low-productivity workers have a per year cost of \$30,001. What is the range of $\bar{y}$ that firms can choose as the threshold education level so that only high-productivity workers go to college?
	\end{exmp}
\end{frame}

\begin{frame}{HC Theory vs Signaling}
	\begin{itemize}
		\item Consider two individuals, John and Allie, who are seemingly identical in every way except that
		\begin{itemize}
			\item John completed 3.99 years of college
			\item Allie completed 4 years of college
		\end{itemize}
		\item What does Human Capital Theory say about their earnings?
		\item What does Signaling Theory say about their earnings?
		\item In a signaling framework, individuals can be rewarded for passing threshold level of education
		\begin{itemize}
			\item Referred to as a ``Sheepskin Effect''
		\end{itemize}
	\end{itemize}
\end{frame}



\begin{frame}{HC Theory vs Signaling}
	\begin{itemize}
		\item Very difficult to empirically separate productivity-enhancing component of schooling from ``sheepskin effect'' 
		\begin{itemize}
			\item Under both frameworks, one would observe that more education leads to higher earnings
		\end{itemize}
		\item Policy implications much different under each framework
	\end{itemize}
\end{frame}

\begin{frame}{Readings}
	\begin{itemize}
		\item Borjas 6.9
	\end{itemize}
\end{frame}


\section{Human Capital and Development}

\begin{frame}{Human Capital and Development}
	\begin{itemize}
		\item How can education improve the welfare of individuals in low-income countries?
		\begin{enumerate}
			\item Productivity-enhancing component increases private returns to schooling
			\item Both private and social returns to schooling estimated to be high in developing countries
			\item Education may also aid individuals adopting new technologies $\Rightarrow$ greater productivity
			\item Education can be a means to improve health outcomes
		\end{enumerate}
		\item Mixed evidence on the causal effect of education on overall economic growth
	\end{itemize}
\end{frame}


\begin{frame}{Human Capital and Development - Trends}
	\begin{itemize}
		\item Tremendous increase in overall school enrollment rates since 1960 across developing countries (Glewwe \& Kremer, 2006)
	\end{itemize}
	\begin{figure}
		\centering
		\includegraphics[scale=.6]{05E_1}
	\end{figure}
\end{frame}

\begin{frame}{Human Capital and Development - Trends}
	\begin{figure}
		\centering
		\includegraphics[scale=.45]{05E_2}
	\end{figure}
\end{frame}


\begin{frame}{Human Capital and Development - Trends}
	\begin{figure}
		\centering
		\includegraphics[scale=.55]{05E_3}
	\end{figure}
\end{frame}

\begin{frame}{Human Capital and Development - Trends}
	\begin{itemize}
		\item School attainment and literary rates have also increased since 1960s
	\end{itemize}
	\begin{figure}
		\centering
		\includegraphics[scale=.45]{05E_4}
	\end{figure}
\end{frame}


\begin{frame}{Human Capital and Development - Trends}
	\begin{figure}
		\centering
		\includegraphics[scale=.5]{05E_5}
	\end{figure}
\end{frame}



\begin{frame}{Human Capital and Development - Trends}
	\begin{itemize}
		\item Gender disparities in access to education are significant in certain regions
	\end{itemize}
	\begin{figure}
		\centering
		\includegraphics[scale=.45]{05E_6}
	\end{figure}
\end{frame}

\begin{frame}{Human Capital Demand}
\begin{itemize}
	\item Human capital investments in schooling respond to (among other things) the price of education 
	\item Basic framework: Continue schooling as long as $MRR \ge MC$
	\item Knowing how individuals respond to changes in the price of education (e.g., how sensitive individuals are to changes in prices) is important when designing policy 
	\begin{itemize}
		\item Subsidies and grants
		\item Scholarships
		\item Conditional cash transfers
		\item Etc.
	\end{itemize}
\end{itemize}
\end{frame}

\begin{frame}{Human Capital Demand}
	\begin{itemize}
		\item Kremer, et al. (2009)
		\item Study randomized trial in Kenyan primary schools
		\begin{itemize}
			\item Treatment schools enacted scholarship program for girls who scored well on academic exams
			\item Scholarship paid for school fees and provided a grant
		\end{itemize}
	\end{itemize}
\end{frame}

\begin{frame}{Human Capital Demand}
\begin{itemize}
	\item Girls showed substantial exam score gains 
\item Positive program effects among girls with low pretest scores who were unlikely to win
\item Boys (who were ineligible for the award) show slightly higher test scores
\item These positive externalities are likely to be due to higher teacher attendance or positive peer effects among students (or both)
\item Find no evidence for weakened intrinsic motivation
\end{itemize}
\end{frame}

\begin{frame}{Human Capital Demand}
\begin{itemize}
	\item Schultz (2004)
	\item \textit{Progressa} conditional cash transfer program in Mexico
\end{itemize}
\begin{figure}
	\centering
	\includegraphics[scale=.4]{05E_7}
\end{figure}
\end{frame}

\begin{frame}{Human Capital Demand}
\begin{itemize}
	\item Randomized trial: Program was randomly allocated among an initial group of  localities
	\item Evaluation: Compare mean enrollment rates of those eligible for assistance in the treatment and control villages (Difference-in-differences)
\end{itemize}
\end{frame}


\begin{frame}{Human Capital Demand}
	\begin{figure}
	\centering
		\includegraphics[scale=.6]{05E_8}
\end{figure}
\end{frame}

\begin{frame}{Human Capital Demand}
	\begin{figure}
	\centering
	\includegraphics[scale=.6]{05E_9}
\end{figure}
\end{frame}

\begin{frame}{Human Capital Demand}
\begin{itemize}
	\item Framework assumes individuals know the true $MRR$ to schooling when making optimal choice
	\item May not always be true, especially in developing countries (Jensen, 2010)
	\begin{itemize}
		\item Decision to drop out often made at a younger age
		\item Little information may available on labor market earnings
		\item Perceived returns influenced by surroundings - rural individuals may not true potential realize returns in urban sector
	\end{itemize}
\end{itemize}
\end{frame}


\begin{frame}{Human Capital Demand}
\begin{itemize}
	\item Intervention: Students at randomly selected schools were provided information about the returns to education
\item Relative to those not provided with the information, students reported a higher perceived return to education when re-interviewed 
\item Treatment group completed an average of .20 more years of schooling than control group over next 4 years
\item Heteregenous effects: Large effect among poor students, but no effect on the poorest students 
\end{itemize}
\end{frame}


\begin{frame}{HC Supply \& Political Economy}
\begin{itemize}
	\item Focus so far has been on the quantity of education (e.g., enrollment or years completed)
	\item Quality of education is also important in order to enhance productivity
	\item In general, the quality of education in developing countries is very low

\end{itemize}
\end{frame}

\begin{frame}{HC Supply \& Political Economy}
	\begin{figure}
	\centering
	\includegraphics[scale=.6]{05E_10}
\end{figure}
\end{frame}


\begin{frame}{HC Supply \& Political Economy}
	\begin{figure}
	\centering
	\includegraphics[scale=.60]{05E_11}
\end{figure}
\end{frame}

\begin{frame}{HC Supply \& Political Economy}
	\begin{figure}
		\centering
		\includegraphics[scale=.55]{05E_12}
	\end{figure}
\end{frame}

\begin{frame}{HC Supply \& Political Economy}
	\begin{figure}
		\centering
		\includegraphics[scale=.55]{05E_13}
	\end{figure}
\end{frame}


\begin{frame}{HC Supply \& Political Economy}
\begin{itemize}
	\item Contributing causes: 
	\begin{itemize}
		\item Lack of infrastructure
		\item Lack of resources (e.g., materials, teachers)
		\item Low teacher quality
		\item Low teacher effort 
	\end{itemize} 

\end{itemize}
\end{frame}

\begin{frame}{Healthcare}
\begin{itemize}
	\item Health is part of human capital
	\item Health, education, and learning intimately related
	\item Das, et al., 2008
	\item Healthcare has improved drastically in developing countries
	\item \textit{Quality} of healthcare still very low 
	\item Similar issues: low quality doctors, low effort, lack of access for poor individuals
	\item Spillovers: Miguel \& Kremer (2004)
\end{itemize}
\end{frame}


\begin{frame}{Readings}
	\begin{itemize}
		\item Das, et al. (2008). The Quality of Medical Advice in Low-Income Countries. \textit{The Journal of Economic Perspectives}
		\item Jensen (2010). The (Perceived) Returns to Education and the Demand for Schooling. \textit{The Quarterly Journal of Economics}
		\item Kremer, et al. (2009). Incentives to Learn. \textit{The Review of Economics and Statistics}
		\item Miguel \& Kremer (2004). Worms: Identifying Impacts on Education and Health in the Presence of Treatment Externalities
		\item Schultz (2004). School Subsidies for the Poor: Evaluating the Progresa Poverty Program. \textit{Jounal of Development Economics}
	\end{itemize}
\end{frame}


	
\end{document}