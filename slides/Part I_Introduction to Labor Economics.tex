\documentclass[pdf]{beamer}
\usetheme{Frankfurt}  
\usecolortheme{whale}
\usepackage{tikz} 
\usepackage{amsmath}
\usepackage{amsthm}
\usepackage{amssymb}              % used for \eqref{} in this document
\usepackage{dsfont}
\usepackage{hyperref}
\usepackage{threeparttable}
\usepackage{multirow}
\graphicspath{{Figures/}}
\usepackage{booktabs}
\usepackage{tikz}
\newtheorem{exmp}{Example}[section]
\usepackage{subcaption}
\usepackage{adjustbox}
\usepackage{graphicx}
\usepackage[mathscr]{euscript}
\usepackage{remreset}% tiny package containing just the \@removefromreset command
\makeatletter
\@removefromreset{subsection}{section}
\makeatother
\setcounter{subsection}{1}


\section{Class Information}

%% preamble
\title{Part I: Introduction to Labor Economics}
\author[David A. D\'iaz]{David A. D\'iaz}
\institute{UNC Chapel Hill}
\date{}



\AtBeginSection[] %Section links on slides


\begin{document}
 
	
	\begin{frame}
		
		\titlepage
		
	\end{frame}

	
\begin{frame}{About Me (Less Fun)}
	\begin{itemize}
		\item Economics PhD student at UNC
		\item Appalachian State University alumnus
		\begin{itemize}
			\item Majors: Actuarial Science, Economics
			\item Minors: Stats, Physics
		\end{itemize}
		\item Research interests: Migration, development, labor, demography
	\end{itemize}
\end{frame}


\begin{frame}{About Me (More Fun)}
	\begin{itemize}
		\item Fav. Music: Glass Animals, T\O P, Misterwives, Two Door Cinema Club, Childish Gambino (allegedly ``angsty'')
		\begin{itemize}
			\item Current Anthem: Glass Animals - Youth
		\end{itemize}
		\item Fav. TV: Rick \& Morty, It's Always Sunny, Archer, Real Housewives of Atlanta
		\item Fav. Movies: Fantastic Mr. Fox , Kiss Kiss Bang Bang, The Social Network, No Country for Old Men
	\end{itemize}
\end{frame}

\begin{frame}{About Me (More Fun)}
	\begin{itemize}
		\item Fav. Team: FC Barcelona 
\item Celebrity Crush: Kevin Spacey
\item B\ae s: Elizabeth (main) \& Stella
\item Likes: Beer, t-shirts, normal shoes, Tina Fey
\item Dislikes: Gyms, loud chewing, open-toed shoes, students packing up early $\sim$ shade emoji $\sim$
	\end{itemize}
\end{frame}

\begin{frame}{B\ae s}
\begin{figure}
			\centering
			\includegraphics[scale=.14]{baes.jpg}
			\caption{Stella Artois D\'iaz \& Elizabeth}
\end{figure}
\end{frame}


\begin{frame}{Class Details}
	
	\begin{itemize}
	\item \textbf{Email:} \url{diazda@live.unc.edu}
	\item \textbf{Office:} Phillips \underline{Annex} 103A
	\item \textbf{Office Hours:} Monday \& Wednesday, 2:30-3:30PM
	\item \textbf{Website:} \href{https://sakai.unc.edu/portal/site/8456c8f3-2ffa-4262-84e9-5993c613ddb9
	}{https://sakai.unc.edu}
	\item \textbf{Prerequisites:} ECON 101 \& ECON 310/410
				
	\end{itemize}
	
\end{frame}


\begin{frame}{Class Details}
	\begin{figure}
		\centering
		\includegraphics[scale=.6]{annex.jpg}
		\caption{The Annex}
	\end{figure}
\end{frame}




	\begin{frame}{Textbook}
		\begin{itemize}
			\item George Borjas, \textit{Labor Economics}, 8$^{th}$ edition
			\begin{itemize}
				\item Note: Older editions should work fine, but it is your responsibility to match up content.
				\item Any assigned readings from the text are fair game for exams unless told otherwise.
				\item Any assigned readings outside the text or from homework are also fair game unless told otherwise.
			\end{itemize}
	\end{itemize}
	\end{frame}


\begin{frame}{Grading Scale}
	\begin{itemize}
		\item The grading scale is as follows:
	\end{itemize}
	\begin{center}
		\begin{tabular}{ p{3.5cm} p{3.5cm} }
			A : 93 -- 100 &  C+ : 77 -- 79.99\\
			A-- : 90 -- 92.99 & C : 73 -- 76.99\\
			B+ : 87 -- 89.99 & C-- : 70 -- 72.99\\
			B : 83 -- 86.99 & D+ : 65 -- 69.99\\
			B-- : 80 -- 82.99 & D : 60 -- 64.99\\
			& F : $<$ 60		
		\end{tabular}
	\end{center}
	\begin{itemize}
		\item Grades will be regularly updated on Sakai. Please bring up any discrepancies in a timely manner (within 1 week). 
	\end{itemize}
\end{frame}


\begin{frame}{Grade Components}
	\begin{itemize}
			\item Homework: 25\%
		\item Midterm 1: 20\%
		\item Midterm 2: 20\%
		\item Final Exam: 35\%
	\end{itemize}
\end{frame}



	
	\begin{frame}{Homework}
		\begin{itemize}
			\item Six (potentially seven) homework assignments will be assigned throughout the semester.
			\item Homework is due by the end of class on the assigned date.
			\item You are encouraged to work together, but everyone should turn in an individual assignment.
			\item No late homework will be accepted without prior approval.
		\end{itemize}
	\end{frame}
	

	
		\begin{frame}{Exams}
			
			\begin{itemize}
				\item Two in-class exams will take place during regular class hours. Each is worth 20\% of the final grade.
				\item Dates for the in-class midterms will be determined at least a week ahead of time. 
				\item The cumulative final exam is scheduled for 12PM on Monday December 11. 
				\item All exams will be closed book/note. You will be allowed the use of a \textbf{non-graphing} calculator. 
			\end{itemize}
			
		\end{frame}
	
\begin{frame}{Policies \& Expectations}
	
		\begin{itemize}
			
			\item Regular attendance and participation is strongly encouraged. It is \textbf{your} responsibility to get any notes/announcements you may have missed from a classmate.
			\item Be respectful to both me and your classmates.
			\item Emails should be written in a professional manner.
				\begin{itemize}
					\item Use your UNC email, as email from other clients may end up in spam.
					\item I will do my best to respond to emails within 24 hours. If you have not heard back in 48 hours, please follow up to make sure your email was received.
				\end{itemize}  
			
		\end{itemize}

	
\end{frame}

\begin{frame}{Lectures}
\begin{itemize}
	\item Slides and the board will both be used to present material throughout the course.
		\begin{itemize}
			\item Data and concepts will mostly be presented on slides.
			\item Exposition, diagrams, and math will usually be done on the board.
		\end{itemize}
	\item Slides will generally be posted at least 24 hours before the corresponding class meeting.
\end{itemize}
\end{frame}

\begin{frame}{General Advice}
	
\begin{itemize}
	\item This course does not use calculus, but you will be required to use algebra, including:
		\begin{itemize}
			\item Setting up equations.
			\item Solving equations.
			\item Interpreting mathematical expressions.
			\item Etc.
		\end{itemize}
	\item Get in touch with me early if you are struggling with the material.
	\item Try each homework assignment on your own at least once before working through it with others.
	\item Ask questions if you have them! When? Any time - during lecture, before/after class, during office hours, or through email.
\end{itemize}
		
\end{frame}

\begin{frame}{General Advice}
	\begin{figure}
	\centering
	\includegraphics[scale=.23]{Grad_Questions.png}
\end{figure}
\end{frame}


\section{Introduction to Labor Economics}

\begin{frame}{Introduction to Labor Economics}
	
\begin{itemize}
	\item We will study various aspects of labor markets. Our goal is to gain insight into labor market behavior by applying concepts from microeconomic theory.
	\item Throughout the course, we will apply what we learn to various real-world policy issues such as the minimum wage, immigration, and wage inequality.
	\item Though our focus will be on the U.S. labor market and applicable policies, we will also look at some labor market issues in developing nations for certain topics.
\end{itemize}
	
\end{frame}

\begin{frame}{Actors in the Labor Market}
	\begin{itemize}
	\item In their simplest form, labor markets have three actors:
		\begin{itemize}
			\item Workers
			\begin{itemize}
				\item Objective: Maximize utility by choosing education, labor force participation, effort level, occupation, etc.
				\item Contribute to \textbf{labor supply}.
			\end{itemize}
			\item Firms
				\begin{itemize}
					\item Objective: Maximize profits by employing labor and capital.
					\item Contribute to \textbf{labor demand}.
				\end{itemize}
			\item Government
				\begin{itemize}
					\item Sets regulatory structure of labor markets.
					\item Enacts policies such as income tax, education subsidies, minimum wages, workplace safety regulations, etc.
				\end{itemize}
		\end{itemize}
	\end{itemize}
\end{frame}



\begin{frame}{Labor Economics: The ECON 101 Story}
	
	\begin{itemize}
		\item Each point on the labor supply curve represents a worker's reservation price for selling labor.
			\begin{itemize}
				\item Higher wage $\Rightarrow$ more individuals willing to sell labor $\Rightarrow$ higher quantity of labor supplied.
				\item Thus, the labor supply curve slopes upward.
			\end{itemize}
		\item Each point on the labor demand curve represents a firm's reservation price for purchasing labor.
			\begin{itemize}
				\item Higher wage $\Rightarrow$ fewer firms willing to purchase labor $\Rightarrow$ lower quantity of labor demanded.
				\item Thus, the labor demand curve slopes downward.
			\end{itemize}
	\end{itemize}
	
\end{frame}

\begin{frame}{Labor Economics: The ECON 101 Story}
	\begin{itemize}
		\item Labor markets can be analyzed just like any other market through the supply and demand framework.
		\item Implications:
			\begin{itemize}
				\item Tax equivalence: Payroll tax on firms has the same effect as an income tax on workers.
				\item Markets clear in the absence of regulation: No \textit{involuntary} unemployment in the simplest framework.
				\item Law of One Price holds: Market bears a sole equilibrium wage paid to all workers of a particular skill level.
			\end{itemize}
	\end{itemize}
\end{frame}

\begin{frame}{Labor Economics in 380}
	\begin{itemize}
		\item Real-world labor markets exhibit much richer (and interesting!) behavior such as:
			\begin{itemize}
				\item Wage dispersion due to differences in job characteristics, education levels, demographics, bargaining, and willingness to ``search'' for better opportunities.
				\item Market rigidities which prevent the market from clearing leading to significant levels of involuntary unemployment and a costly search and matching process between workers and firms.
			\end{itemize}
	\end{itemize}
\end{frame}


\begin{frame}{Road Map}
	\begin{itemize}
		\item Part I: Labor Market Basics and the U.S. Labor Market
			\begin{itemize}
				\item Overview of the U.S. labor market.
				\item How do we measure the labor force, employment, and unemployment? 
				\item Does the unemployment rate accurately reflect economic conditions?
				\item What are the different types of unemployment?
			\end{itemize}
		\item Part II: Neoclassical Labor Supply and Demand
			\begin{itemize}
				\item Model of individual and firm labor market behavior.
				\item How do firms and workers interact to drive market outcomes?
				\item How do firms and workers respond to changes in their incentives (e.g., minimum wage policies)?
			\end{itemize}
	\end{itemize}
\end{frame}

\begin{frame}{Road Map}
\begin{itemize}
	\item Part III: Human Capital
	\begin{itemize}
		\item How do workers choose their particular set of acquired skills that they offer employers?
		\item Why do firms value educated workers?
		\item What information is conveyed by our willingness to spend more time in school?
	\end{itemize}
	\item Part IV: Wage Inequality
\begin{itemize}
	\item What factors determine the shape of the wage distribution?
	\item Why is wage inequality rising?
	\item Why do some industries display exceptional wage dispersion?
\end{itemize}
\end{itemize}
\end{frame}

\begin{frame}{Road Map}
	\begin{itemize}
		\item Part V: Labor Market Discrimination
		\begin{itemize}
				\item Why do different demographic groups earn different average wages?
				\item What are the different way in which discrimination can take place?
				\item How can we measure the extent of discrimination in the labor market?
		\end{itemize}
			\item Part VI: Selected Topics
			\begin{itemize}
				\item Human capital: The supply side
				\item Social interactions \& social networks
			\end{itemize}
	\end{itemize}
\end{frame}

\begin{frame}{Readings}
\begin{itemize}
	\item Borjas 1.1-1.3
\end{itemize}
\end{frame}



\section{The U.S. Labor Force}

\begin{frame}{Measuring the Labor Force}
	\begin{itemize}
		\item What are we trying to measure?
		\begin{itemize}
			\item Micro-level: Number of hours individuals are willing to work at a given wage rate.
			\item Macro-level: Total amount of labor supplied in the economy.
			\begin{itemize}
				\item Different ways to measure: total number of workers, proportion of the population who works, etc.
			\end{itemize}
			\item Today we will consider how to measure this in practice.
		\end{itemize}
	\end{itemize}
\end{frame}

\begin{frame}{Measuring Labor Supply: CPS}
	\begin{itemize}
		\item \textbf{Current Population Survey (CPS)} is conducted monthly by the Bureau of the Census for the \textbf{Bureau of Labor Statistics (BLS)}
		\begin{itemize}
			\item $\sim$ 60,000 households are questioned about their work activities during a particular week (``reference'' week) of the month.
			\item Goal: Elicit employment status of household members
		\end{itemize}  
	\end{itemize}
\end{frame}


\begin{frame}{Measuring Labor Supply: Working Population}
	
	\begin{itemize}
		\item Most members of the population aged 16 or older make up the \textbf{working population (P)}. 
		\item Excluded from \textbf{P} are 
		\begin{itemize}
			\item people living in institutions (e.g., jail, nursing homes)
			\item those on active duty in the Armed Forces
		\end{itemize}
	    \item Individuals in \textbf{P} are further classified as either
		\begin{enumerate}
			\item in the labor force \textbf{(L)} 
			\item out of the labor force \textbf{(O)}
		\end{enumerate} 		
	\end{itemize}
	
\end{frame}



\begin{frame}{Measuring Labor Supply: The Labor Force}
	\begin{itemize}
		\item An individual who is in \textbf{LF} is further categorized as either
		 \textbf{Employed (E)} or \textbf{Unemployed (U)}
		\item To be classified as employed, an individual must have
			\begin{itemize} 
				\item worked at least one hour as a paid employee (including self-employment) OR
				\item worked at least 15 hours in an unpaid role for a family business OR
				\item a job but is on temporary leave due to illness, maternity leave, etc.
			\end{itemize} 
		\end{itemize}
\end{frame}

\begin{frame}{Measuring Labor Supply: The Labor Force}
\begin{itemize}
	\item To be classified as unemployed, an individual must 
		\begin{itemize}
			\item have had no employment during the reference week AND
			\item be currently available for work AND
			\item have \underline{actively} looked for work in the four-week period prior to the reference week
		\end{itemize}
\end{itemize}
\end{frame}

\begin{frame}{Measuring Labor Supply: The Labor Force}
\begin{itemize}
	\item The out of the labor force group, \textbf{O}, includes:
	\begin{itemize}
		\item Retirees
		\item Students
		\item Non-market laborers (e.g., stay at home parents)
		\item ``Off-the-books'' laborers
		\item Black market laborers
		\item \textbf{Marginally attached workers:} Have searched for a job in the past 12 months, but not in the past 4 weeks regardless of the reason
		\item \textbf{Discouraged workers:} Have searched for a job in the past 12 months, but not the past 4 weeks because they believe they won't find a job in their line of work.
	\end{itemize}
\end{itemize}
\end{frame}


\begin{frame}{Measuring Labor Supply: The Labor Force}
\scriptsize
\begin{exmp}
	Classify the labor force status of the following individuals (\textit{How the Government Measures Unemployment}, BLS 2014).
	\begin{enumerate}
		\item Lisa spends most of her time taking care of her home and children, but she helps in
		her husband's computer software business all day Friday and Saturday. 
		\item Ms. Jenkins tells the interviewer that her teenage daughter, Katherine Marie, was
		thinking about looking for work in the prior 4 weeks but knows of no specific efforts
		she has made.
		\item Last week, Megan, who was working for a comic book store, went to a home
		electronics store on her lunch hour to be interviewed for a higher paying job. 
		\item Avery lost her full-time job at a book store on Wednesday of the survey reference
		week. She submitted several applications with other local retailers on Thursday and
		Friday but had not obtained a new job by the end of the week
	\end{enumerate}
\end{exmp}
\end{frame}


\begin{frame}{Measuring Labor Supply: The Labor Force}
	\begin{itemize}
		\item The Labor Force trends upwards with population growth.
	\end{itemize}
	
	\begin{figure}
		\centering
		\includegraphics[scale=.3]{01B_1.png}
		\caption{US Labor Force (thousands), 1948 - 2017}
	\end{figure}
	
	
\end{frame}


\begin{frame}{Measuring Labor Supply: The Labor Force}
	\begin{itemize}
		\item Employed workers generally trend upwards with population growth, but dips during recessions.
	\end{itemize}
	
	\begin{figure}
		\centering
		\includegraphics[scale=.3]{01B_2.png}
		\caption{US Employed Workers (thousands), 1948 - 2017}
		\item 
	\end{figure}
	
	
\end{frame}

\begin{frame}{Measuring Labor Supply: The Labor Force}
	\begin{itemize}
		\item Unemployed workers spike during recessions.
	\end{itemize}
	
	\begin{figure}
		\centering
		\includegraphics[scale=.3]{01B_3.png}
		\caption{US Unemployed Workers (thousands), 1948 - 2017}
		\item 
	\end{figure}
	
\end{frame}




\begin{frame}{Levels or Rates?}
	
	\begin{itemize}
		
		\item Tough to distinguish most of these trends from population growth.
		\item Labor force data generally makes more sense as a rate versus a level.
		\begin{itemize}
			\item e.g., Texas had more unemployed workers than Michigan during the 2009 recession, but had a lower unemployment \textit{rate}.
		\end{itemize}  
		
	\end{itemize}
	
	
\end{frame}


\begin{frame}{Measuring Labor Supply: The Labor Force}

\begin{itemize}
	\item \textbf{The Labor for Participation rate} measures the proportion of the working population who is in the labor force.
	\[\textbf{LFPR = LF/P = (E+U)/(E+U+O)}\]
\end{itemize}
\begin{exmp} 
	\small
	A country has a population of 160 million. 30 million are under the age of 16, and 10\% of the adult population is either in the military or institutionalized. If 70 million people have jobs and 5 million are looking for work, what is the labor force participation rate in this country? 
\end{exmp}
\end{frame}

\begin{frame}{Measuring Labor Supply: The Labor Force}
\begin{exmp} 
	What effect do each of the following scenarios have on the labor force participation rate?
	\begin{enumerate}
		\item Sue lost her job and begins looking for a new one. 
		\item Jon, a steelworker who has been out of work since his mill closed last year, becomes discouraged and gives up looking for work. 
		\item Sam, the sole earner in his family of 5, just lost his \$80,000 job as a research scientist. Immediately, he takes a part-time job at McDonald's until he can find another job in his field.
		\item Robert comes out of retirement and begins working at a proctoring center.
	\end{enumerate}
\end{exmp}
\end{frame}

\begin{frame}{Measuring Labor Supply: The Labor Force}

\begin{figure}
	\centering
	\includegraphics[scale=.3]{01B_11.png}
	\caption{US Labor Force Participation Rate, 1948 - 2017}
\end{figure}

\begin{itemize}
	\item Why has the LFPR fallen since 2000?
\end{itemize}

\end{frame}

\begin{frame}{Measuring Labor Supply: Falling LFPR}
\begin{itemize}
	\item Aging population.
	\begin{itemize}
		\item Perhaps most important factor.
		\item ``Baby-boomer'' generation aging out of workforce.
	\end{itemize}
\end{itemize}
\begin{figure}
	\centering
	\includegraphics[scale=.4]{01B_9.png}
	\caption{LFPR and Demographics, 1948-2006}
	\item 
\end{figure}

\end{frame}

\begin{frame}{Measuring Labor Supply: Falling LFPR}
\begin{itemize}
\item Education
\begin{itemize}
	\item Decision to stay in or go back to school significant driver in decline of LFPR in 16-24 year old group.
\end{itemize}
\end{itemize}

\begin{figure}
\centering
\includegraphics[scale=.8]{01B_8.png}
\caption{LFPR for 16-19 year olds, 1948-2006}
\item 
\end{figure}

\end{frame}


\begin{frame}{Measuring Labor Supply: Falling LFPR}
\begin{itemize}
	\item Declining male participation since 1940s
	\item Declining female participation since $\sim$2000
\end{itemize}
\begin{figure}
	\centering
	\includegraphics[scale=.3]{01B_17.png}
	\caption{LFPR Males and Females, 1948-2006}
	\item 
\end{figure}
\end{frame}

\begin{frame}{Measuring Labor Supply: Falling LFPR}
\begin{itemize}
\item Cyclical effects: Marginally attached/discouraged workers are much more prevalent during recession/recovery periods.
\end{itemize}

\begin{figure}
\centering
\includegraphics[scale=.8]{01B_4.png}
\caption{US Discouraged Workers (thousands), 2005 - 2016}
\item 
\end{figure}

\end{frame}


\begin{frame}{Measuring Labor Supply: Trends}
\begin{itemize}
	\item Tough to explain fall in the LFPR among prime-age males.
\end{itemize}
\begin{figure}
	\centering
	\includegraphics[scale=.5]{01B_13.png}
	\caption{Prime-Age Male LFPR, 1948-2014}
\end{figure}
\end{frame}

\begin{frame}{Measuring Labor Supply: Trends}
\begin{figure}
	\centering
	\includegraphics[scale=.5]{01B_10.png}
	\caption{Prime-Age Male LFPR by Education, 1948-2014}
\end{figure}
\end{frame}

\begin{frame}{Measuring Labor Supply: Trends}
\begin{itemize}
	\item Tremendous rise in labor force participation of women since 1950s
\end{itemize}
	\begin{figure}
		\centering
		\includegraphics[scale=.3]{01B_14.png}
		\caption{Female LFPR 1948-2015}
	\end{figure}
	
\end{frame}


\begin{frame}{Measuring Labor Supply: Trends}
\begin{itemize}
	\item Labor force participation increased both within and across birth cohorts
	\begin{itemize}
		\item Women within cohorts participated more as they aged
		\item More recent cohorts has larger participation rates
	\end{itemize}
	\item Key determinants?
	\begin{itemize}
		\item Rising real wages
		\item Lower fertility rates 
		\item Technological advances in household production
		\item Cultural and legal changes
	\end{itemize}
\end{itemize}
\end{frame}

\begin{frame}{Measuring Labor Supply: Trends}
\begin{figure}
	\centering
	\includegraphics[scale=.4]{femwages.png}
\end{figure}
\end{frame}


\begin{frame}{Measuring Labor Supply: Trends}
	\begin{figure}
		\centering
		\includegraphics[scale=.6]{01B_7.png}
		\caption{LFPR by Race/Ethnicity, 1980-2008}
		\item 
	\end{figure}
	
\end{frame}

\begin{frame}{Readings}
\begin{itemize}
	\item Borjas 2.1-2.2; 2.9
	\item Toossi, Mitra (2012). Projections of the labor force to 2050: 
	A Visual Essay. \textit{Monthly Labor Review}.
	\item US Bureau of Labor Statistics (2014). How the Government
Measures Unemployment.
\end{itemize}
\end{frame}


\section{Unemployment}


\begin{frame}{Unemployment: Measuring the Unemployed}
	\begin{itemize}
		\item \textbf{Unemployed:} A person that had no employment in the reference week AND has ``actively looked for work'' in the past four weeks.
		\item The \textbf{unemployment rate (UR)} is the proportion of individuals who are in the labor force, but are unemployed.
	\[ \textbf{UR = U/LF = U/(E + U)} \]
	\end{itemize}
	
\end{frame}

\begin{frame}{Unemployment: Trends}
\begin{figure}
	\centering
	\includegraphics[scale=.3]{01B_3.png}
	\caption{US Unemployed Workers (thousands), 1948 - 2017}
	\item 
\end{figure}
\end{frame}


\begin{frame}{Unemployment: Measuring the Unemployed}
	\begin{exmp} 
	What effect do each of the following scenarios have on the unemployment rate?
	\begin{enumerate}
		\item Sue lost her job and begins looking for a new one. 
		\item Jon, a steelworker who has been out of work since his mill closed last year, becomes discouraged and gives up looking for work. 
		\item Sam, the sole earner in his family of 5, just lost his \$80,000 job as a research scientist. Immediately, he takes a part-time job at McDonald's until he can find another job in his field.
		\item Robert comes out of retirement and begins working at a proctoring center.
	\end{enumerate}
\end{exmp}
\end{frame}



\begin{frame}{Measuring Labor Supply: Unemployment}
\begin{figure}
	\centering
	\includegraphics[scale=.8]{01B_16.png}
	\caption{US Unemployment Rate, 2005 - 2016}
	\item 
\end{figure}
\end{frame}

\begin{frame}{Unemployment: Trends}
	\begin{figure}
		\centering
		\includegraphics[scale=.6]{01C_3.png}
		\caption{Unemployment Rate by Sex, 2006-2013}
	\end{figure}
\end{frame}

\begin{frame}{Unemployment: Trends}
	\begin{figure}
		\centering
		\includegraphics[scale=.5]{01C_1.png}
		\caption{Unemployment Rate by Race, 1973-2014}
	\end{figure}
\end{frame}

\begin{frame}{Unemployment: Trends}
	\begin{figure}
		\centering
		\includegraphics[scale=.4]{01C_2.png}
		\caption{Unemployment Rate by Education, 1992-2012}
	\end{figure}
\end{frame}


\begin{frame}{The Unemployment Rate: Issues}
	\begin{itemize}
		\item Is the unemployment rate a good measure of joblessness? What's missing? 
	\end{itemize}
	\begin{figure}
	\centering
	\includegraphics[scale=.55]{NY_Unemployment}
\end{figure}
\end{frame}

\begin{frame}{The Unemployment Rate: Issues}
	\begin{itemize}
		\item Discouraged/marginally attached workers are considered out of the labor force
	\end{itemize}
	\begin{figure}
		\centering
		\includegraphics[scale=.4]{01C_10.png}
		\caption{Marginally Attached Workers (thousands), 2005-2016}
	\end{figure}
	
\end{frame}

\begin{frame}{The Unemployment Rate: Issues}
	\begin{itemize}
		\item Underemployment is not taken into account
		\begin{itemize}
			\item Highly skilled workers in low paying jobs
			\item Highly skilled workers in low skill jobs  
			\item Part-time workers who would prefer to be full time.
		\end{itemize}
	\end{itemize}
	\begin{figure}
		\centering
		\includegraphics[scale=.4]{01C_12.png}
	\end{figure}
\end{frame}

\begin{frame}{The Unemployment Rate: Issues}
	\begin{itemize}
		\item Length of unemployment not taken into account
		\begin{itemize}
			\item Short spells indicate labor market fluidity
			\item Long spells indicate more serious issues
		\end{itemize}
		\begin{figure}
			\centering
			\includegraphics[scale=.5]{01C_4.png}
		\end{figure}	
	\end{itemize}
\end{frame}	

\begin{frame}{The Unemployment Rate: Issues}
	\begin{itemize}
		\item These factors are more severe during recessions
		\item Official \textbf{U3} unemployment rate may understate the depths of recessions and state of economic hardship 
		\item Alternative measures of labor underutilization:
		\begin{itemize}
			\item U4 unemployment rate: Essentially classifies discouraged workers (previously in \textbf{O}) as unemployed 
			\item U5 unemployment rate: Classifies marginally attached workers (previously in \textbf{O}) as unemployed
			\item U6 unemployment rate: Classifies both marginally attached (previously in \textbf{O}) and part-time workers for economic reasons (previously in \textbf{E}) as unemployed 
		\end{itemize}
	\end{itemize}
\end{frame}

\begin{frame}{The Unemployment Rate: Issues}
	\begin{figure}
	\centering
	\includegraphics[scale=.5]{01C_5.png}
	\caption{Alternative Measures of Unemployment, 1994-2015}
\end{figure}
\end{frame}

\begin{frame}{Unemployment Types}
	\begin{itemize}
		\item Frictional unemployment
		\begin{itemize}
			\item Arises due to the time it takes for workers and firms to match
			\item Generally leads to short unemployment spells 
		\end{itemize}
		\item Seasonal unemployment
	\end{itemize}
\end{frame}

\begin{frame}{Unemployment Types}
\begin{itemize}
	\item Cyclical unemployment
	\begin{itemize}
		\item Structural imbalance between the number of workers looking for
		jobs and the number of jobs available
		\item There is an excess supply of workers and the market does not clear because the wage is sticky and cannot adjust downward
	\end{itemize}
	\item Structural unemployment
	\begin{itemize}
		\item Can arise because of a mismatch between the skills that workers are supplying and the skills that firms are demanding
		\item May result because of long-lasting shocks to permanent features of an economy
	\end{itemize}
\end{itemize}
\end{frame}

\begin{frame}{Unemployment Types}
	\begin{itemize}
		\item Natural unemployment rate
		\begin{itemize}
			\item The average level of unemployment that is expected to prevail in an economy 
			\item Combination of frictional and structural unemployment 
		\end{itemize}
		\begin{figure}
			\centering
			\includegraphics[scale=.4]{01C_14.png}
			\caption{Natural Unemployment Rate, 1985-2012}
		\end{figure}  
	\end{itemize}
\end{frame}

\begin{frame}{Readings}
\begin{itemize}
	\item Borjas 12.1-12.2
	\item Brundage, Vernon (2014). Trends in unemployment and other labor
	market difficulties. \textit{Beyond the Numbers}, US Bureau of Labor Statistics.
\end{itemize}
\end{frame}

\end{document}