\documentclass[pdf]{beamer}
\usetheme{Frankfurt}  
\usecolortheme{whale}
\usepackage{tikz} 
\usepackage{amsmath}
\usepackage{amsthm}
\usepackage{amssymb}              % used for \eqref{} in this document
\usepackage{dsfont}
\usepackage{hyperref}
\usepackage{threeparttable}
\usepackage{multirow}
\graphicspath{{Figures/}}
\usepackage{booktabs}
\usepackage{tikz}
\newtheorem{exmp}{Example}[section]
\usepackage{subcaption}
\usepackage{adjustbox}
\usepackage{graphicx}
\usepackage[mathscr]{euscript}
\usepackage{remreset}% tiny package containing just the \@removefromreset command
\makeatletter
\@removefromreset{subsection}{section}
\makeatother
\setcounter{subsection}{1}



\section{Labor Mobility}

%% preamble
\title{Part VI: Labor Mobility}
\author[David A. D\'iaz]{David A. D\'iaz}
\institute{UNC Chapel Hill}
\date{}

\AtBeginSection[] %Section links on slides

\begin{document}
	
	
	\begin{frame}
		
		\titlepage
		
	\end{frame}
	
	\begin{frame}{Table of Contents}
		
		\tableofcontents
	\end{frame}

\begin{frame}{Labor Mobility}
\begin{itemize}
	\item Most economic analysis of migration decisions view migration as a human capital investment
	\item Similar framework to school choice decision outlined earlier
	\begin{itemize}
		\item Worker compares $NPV$ of future income earned by staying in current location to $NPV$ of future income earned in new location
		\item Moving has some cost $C$ 
		\begin{itemize}
			\item Includes both explicit costs (e.g., moving costs, visa costs, etc)
			and implicit costs (e.g., psychic costs of stress, being away from family, etc)
			\item One-time cost incurred at the time of migration 
		\end{itemize}
	\end{itemize}
\end{itemize}
\end{frame}

\begin{frame}{Labor Mobility}
	\begin{itemize}
		\item Wage in home country: $w^H$
		\item Net present value of earnings:
		\[NPV^H = \sum_{t=0}^{T} \frac{w_t^H}{(1+r)^t} = w_0^H + \frac{w_1^H}{(1+r)} + \frac{w_2^H}{(1+r)^2} + \cdots + \frac{w_T^H}{(1+r)^T}\]
		\item Wage in foreign country: $w^F$
		\item Net present value of earnings:
		 \[NPV^F = \sum_{t=0}^{T} \frac{w_t^F}{(1+r)^t} = w_0^F + \frac{w_1^F}{(1+r)} + \frac{w_2^F}{(1+r)^2} + \cdots + \frac{w_T^F}{(1+r)^T}\]
	\end{itemize}
\end{frame}

\begin{frame}{Labor Mobility}
	\begin{itemize}
		\item Net gain to migration: 
		
		
		\[NPV^F - NPV^H - C\]
		\item Worker will choose to migrate if the net gain is positive:
		
		\[\underbrace{NPV^F - NPV^H}_{\text{Marginal Benefit of migrating}} > \underbrace{C}_{\text{Marginal cost of migrating}}\]
		\begin{enumerate}
			\item Better economic opportunities in the destination country increases the net gains to migration $\Rightarrow$ more likely to migrate (\textbf{Pull factors})
			\item Worse economic opportunities in the host country increases the net gains to migration $\Rightarrow$ more likely to migrate (\textbf{Push factors})
			\item Increased migration costs lowers net gains to migration $\Rightarrow$ less likely to migrate
		\end{enumerate}
	\end{itemize}
\end{frame}

\begin{frame}{Application: Angelucci (2015)}
\begin{itemize}
	\item Migration and Financial Constraints: Evidence from Mexico
	\item Motivation: Cash transfer program in Mexico could potentially induce migration by relaxing financial constraints (either through a direct effect or by allowing for borrowing)
	\item Oportunidades is a conditional cash transfer program
	targeting poor Mexican households.
	\item 506 poor rural villages selected based on eligibility for
	Oportunidades.
	\item Randomization of transfers (at village level) for the first 18
	months introduces exogenous variation needed to analyze
	program effects.
\end{itemize}
\end{frame}

\begin{frame}{Application: Angelucci (2015)}
	\begin{itemize}
			\item Individuals with varying skills $s$ and two locations: home ($h$)
			or away ($a$).
			\item Present value of lifetime earnings: $w(s)^h$ and $w(s)^a$: Both
			increasing in $s$. 
			\item Migration costs $K(s)$ are decreasing in $s$.
			\item Individuals migrate if the net benefits are positive:
			\[\underbrace{\Delta w}_{w(s)^a - w(s)^h}  > K(s)\]
			\item $\Delta w$ increasing in $s$
	\end{itemize}
\end{frame}

\begin{frame}{Application: Angelucci (2015)}
\begin{figure}
	\centering 
	\includegraphics[scale=.55]{08A_2}
\end{figure}
\end{frame}

\begin{frame}{Application: Angelucci (2015)}
	\begin{itemize}
		\item Model implication: Lowering migration costs will induce more low-skill individuals to migrate 
		\item Results: 
		\begin{itemize}
		\item Migration rates in treatment villages are significantly larger
		than that in control villages (50\% increase).
		\item Absolute migration rates remain low. Increases from .7\% to
		1.1\% and the average treatment effect is .36\%.
		\item Transfer is mostly consumed and there is little evidence that
		trips are financed through dissaving.
		\item Data on loans is used to show that entitlement to program
		transfers enhances household ability to obtain loans $\Rightarrow$ greater ability to borrow is likely the mechanism through which the program increases propensity to migrate
		\end{itemize}
	\end{itemize}
\end{frame}

\begin{frame}{Migration}
	\begin{exmp}
		Suppose that an individual just turned 18 years old and is choosing whether to migrate. If she migrates, she will earn a salary of \$45,000 each year until she retires at age 60. Assume she gets paid in one lump-sum at the end of each year. She will pay a one-time migration cost of \$38,000 at the time of migration. If she decides to stay, she will earn \$34,000 each year. Assume her discount rate is 5\%.
		\begin{enumerate}[(a)]
			\item What is the net present value of her earnings if she chooses to migrate?
			\item What is the net present value of her earnings if she chooses to stay?
			\item What is her optimal decision?
		\end{enumerate}
	\end{exmp}
\end{frame}

\begin{frame}{Labor Mobility in the US}
\begin{itemize}
	\item Demographic characteristics also seem to play a large role in migration decisions
	\begin{figure}
		\centering
		\includegraphics[scale=.7]{08A_1}
	\end{figure}
\end{itemize}
\end{frame}

\begin{frame}{Labor Mobility in the US}
	\begin{itemize}
		\item Return migration is very common. Why?
		\begin{enumerate}
			\item Uncertainty in the migration decision $\Rightarrow$ lower than expected earnings, bad economic conditions, etc
			\item Might actually maximize present value of lifetime earnings $\Rightarrow$ accumulate HC in one area, returns in other areas may increase as well
		\end{enumerate}
		\item Question: If regional wage gaps are persistent, why isn't there more migration?
		\begin{itemize}
		\item High migration costs very likely the inhibiting factor
		\end{itemize}
	\end{itemize}
\end{frame}



\begin{frame}{Family Migration}
	\begin{itemize}
		\item So far we've modeled the migration decision as an individual choice
		\item Individual moves as long as $NPV^F - NPV^H - C > 0$
		\item However, migration decisions are often not made at the individual level, but at the household level
		\item Similar decision rule: Only migrate if the whole family will be better off
	\end{itemize}
\end{frame}

\begin{frame}{Family Migration}
	\begin{itemize}
		\item Set up:
		\begin{itemize}
			\item $\Delta PV^H$: Change present value of husband's earnings stream if he were to move (includes migration costs)
			\item $\Delta PV^W$: Change present value of wife's earnings stream if she were to move (includes migration costs)
		\end{itemize}
		\item Individually, each would migrate if $\Delta PV^i > 0$
		\item As a family unit, migrate if and only if
		\[\Delta PV^H + \Delta PV^W > 0\]
	\end{itemize}
\end{frame}


\begin{frame}{Family Migration}
	\begin{itemize}
		\item Optimal decision for the family is not necessarily the same as the optimal choice for an individual (see graph)
		\item Tied stayer: Family member who stays because net family gains are negative, even though individual gains would be positive
		\item Tied mover: Family member who moves because net family gains are positive, even though individual gains would are negative
	\end{itemize}
\end{frame}

\begin{frame}{Readings}
	\begin{itemize}
		\item Borjas 8.1-8.3
	\end{itemize}
\end{frame}


\section{Assimilation}
	
	\begin{frame}{Economic Assimilation}
		\begin{itemize}
			\item Question: How do migrants perform in the U.S. labor market?
			\item Typical equation estimated (in cross-section):
			
			\[w_{i} = \beta\mathbf{X}_{i} + \delta LOS_{i} +  +  \varepsilon_{i} \]
			
			\begin{itemize}
				\item $w_{i}$: Wages earned by individual $i$ 
				\item $\color{red}LOS$: Length of stay (Age at survey minus age at migration)
				\item $X$: Observed characteristics (age, experience, schooling, etc.)
				\item $\delta$: Captures how earnings grow with the assimilation process
			\end{itemize}
		\end{itemize}
	\end{frame}
	
	\begin{frame}{Economic Assimilation}
		\begin{itemize}
			\item Chiswick (1978): \textit{The Effect of Americanization on the Earnings of Foreign-born Men}
			\item Early study that used U.S. Census (cross-section) to trace the age-earnings profiles of immigrants and compared it to that of natives
			\begin{itemize}
				\item Cross-section: One-time snapshot of current status
				\item Allows for comparison of current earnings of new migrants to current earnings of previous migrants to current earnings of native workers 
			\end{itemize}
		\end{itemize}
	\end{frame}
	
	\begin{frame}{Economic Assimilation}
		\begin{itemize}
			\item Why might the wages of natives and foreign workers be different?
			\begin{itemize}
				\item Recent arrivals have less knowledge about customs, language, job opportunities, and less country/firm-specific training
				\item As time passes, migrants attain more knowledge, skills, and other human capital that allows their wages to grow
				\item Over time, economic assimilation occurs $\Rightarrow$ migrant earnings begin to converge to the earnings of natives
				\begin{itemize}
					\item Only happens if age-earnings profile for migrants is steeper than that of natives
				\end{itemize}
			\end{itemize}
			\item Both the initial earnings deficiency and steepness of earnings profile depend on similarity of the home and foreign country
		\end{itemize}
	\end{frame}
	
	\begin{frame}{Economic Assimilation}
		\begin{figure}
			\centering 
			\includegraphics[scale=.90]{08B_1}
		\end{figure}
	\end{frame}
	
	\begin{frame}{Economic Assimilation}
		\begin{itemize}
			\item Observe both an earnings deficiency for recent migrants and a steeper age-earnings profile
			\item Why do we see that migrant earnings overtake that of natives?
			\begin{itemize}
				\item Possibly a ``positive selection'' story: Individuals that choose to migrate are more skilled, motivated, etc. compared to those that choose to not migrate
				\item More on self-selection later
			\end{itemize}
		\end{itemize}
	\end{frame}
	
	\begin{frame}{Economic Assimilation}
		\begin{itemize}
			\item Many other cross-section studies also found a significant and positive effect of length of stay on wages ($\delta > 0$)
			\item Issue with cross-sectional studies: We are comparing the earnings of individuals who migrated years ago to those who migrated more recently
			\begin{itemize}
				\item Assumption that economic experience of newer migrants will be identical to that of previous cohorts is likely not reasonable
				\item Migrant cohorts may differ in their observable and unobservable characteristics 
			\end{itemize}
			\item As a result, estimates of $\delta$ are likely biased
			\begin{itemize}
				\item Upward bias: Average ``quality'' of cohorts decreased over time
				\item Downward bias: Average ``quality'' increased over time
			\end{itemize}
		\end{itemize}
	\end{frame}
	
	\begin{frame}{Economic Assimilation}
		\begin{itemize}
			\item To illustrative, consider the following example:
			\begin{itemize}
				\item 1960 cohort: Highly productive, age-earnings profile above that of natives
				\item 1980 cohort: Equally productive, age-earnings profile equivalent to that of natives
				\item 2000 cohort: Lowly productive, age-earnings profile below that of natives
				\item For simplicity, assume there is no wage convergence (age-earnings profiles are parallel) and all migrants arrived at age 20
			\end{itemize}
			\item What happens if we use 2000 Census data to compare earnings of migrants and natives?
		\end{itemize}
	\end{frame}
	
	\begin{frame}{Economic Assimilation}
		\begin{itemize}
			\item Borjas (1985): \textit{Assimilation, Changes in Cohort Quality, and the Earnings of Immigrants}
			\item Uses two waves of census data to study earnings growth of \underline{specific} immigrant cohorts
			\item Empirical model:
			\[w_{it} = \beta \mathbf{X}_{it} + \delta LOS_{it} + \phi_t + C_i + \varepsilon_{it} \]
			\begin{itemize}
				\item $\phi$: Time-trend capturing economic fluctuations
				\item $C$: Captures cohort-specific unobserved heterogeneity 
			\end{itemize}
		\end{itemize}
	\end{frame}
	
	\begin{frame}{Economic Assimilation}
		\begin{itemize}
			\item Finding: Within-cohort wage growth is significantly smaller that what was predicted by cross-sectional studies
			\item Likely driver: Declining quality of migrant cohorts admitted to the U.S.
			\item Can also be caused by selective return migration
		\end{itemize}
	\end{frame}
	
	\begin{frame}{Economic Assimilation}
		\begin{figure}
			\centering 
			\includegraphics[scale=1]{08B_2}
		\end{figure}
	\end{frame}
	
	\begin{frame}{Economic Assimilation}
		\begin{figure}
			\centering 
			\includegraphics[scale=1]{08B_3}
		\end{figure}
	\end{frame}

\begin{frame}{Readings}
	\begin{itemize}
	\item Borjas 8.4-8.5
	\end{itemize}
\end{frame}

\section{Self-Selection}


\begin{frame}{Self-Selection}
	\begin{itemize}
		\item Much variation in migrant labor market performance depending on country of origin
		\item Likely driven by degree of skill transferability between home and host country
	\end{itemize}
	\begin{figure}
		\centering 
		\includegraphics[scale=.6]{08C_2}
	\end{figure}
\end{frame}

\begin{frame}{Self-Selection}
	\begin{itemize}
		\item Migrants are not a randomly selected subset of the home country's population
		\item Question: Which subset of workers in a given source country finds it worthwhile to migrate to the United States?
		\begin{itemize}
			\item What factors drive this ``self-selection'' into migration?
		\end{itemize}
	\end{itemize}
\end{frame}

\begin{frame}{Self-Selection}
	\begin{figure}
		\centering
		\includegraphics[scale=.45]{08C_1}
		\caption{MXFLS Migration Stats}
	\end{figure}
\end{frame}

\begin{frame}{Self-Selection}
	\begin{itemize}
		\item Widely used model of self-selection: The Roy model (Roy 1951)
		\item Model widely applicable to wide set of situations where self-selection is present (e.g., migration, college choice, etc)
		\item Applied to migration context in seminal paper by Borjas (1987) 	
		\item Goal: Determine what factors drive positive or negative selection in terms of immigrant flows
		\item Positive selection: Immigrants have above-average skills relative to home country population
		\item  Negative selection: Immigrants have above-average skills relative to home country population
	\end{itemize}
\end{frame}


\begin{frame}{Self-Selection}
	\begin{itemize}
		\item Assumptions:
		\begin{itemize}
			\item Earnings in home and host country only depend on skill level $s$
			\item Skills are perfectly transferable across countries 
		\end{itemize}
		\item As always, each worker makes migration decision by comparing earnings in the home and host country (net of migration costs)
		\item For now, assume migration costs are zero
	\end{itemize}
\end{frame}

\begin{frame}{Self-Selection}
	\begin{itemize}
		\item Case 1: Rate of return to skills is greater in the host country
		\item Translation: Payoff to an an additional unit of ``skill'' (i.e., human capital) is higher in the host country vs the home country
		\item Result: Wage-skills line is steeper for the host country
		\item Implications for selection?
	\end{itemize}
\end{frame}

\begin{frame}{Self-Selection}
	\begin{itemize}
		\item Case 2: Rate of return to skills is greater in the home country
		\item Translation: Payoff to an an additional unit of ``skill'' (i.e., human capital) is higher in the home country vs the host country
		\item Result: Wage-skills line is steeper for the home country
		\item Implications for selection? 
	\end{itemize}
\end{frame}

\begin{frame}{Self-Selection}
	\begin{itemize}
		\item Key implication: The \underline{relative} payoff for skills across countries is what drives the composition of the immigrant workforce 
		\item Evidence tends to back this: Negative correlation between home country's level of income inequality and earnings of migrants in the US
	\end{itemize}
\end{frame}

\begin{frame}{Self-Selection}
	\begin{itemize}
		\item What happens if we change ``base level'' of income in either the home or host country?
		\item Result: Selection process remains the same!
		\item However, magnitude of migrant flows will change
	\end{itemize}
\end{frame}

\begin{frame}{Self-Selection}
	\begin{itemize}
		\item What about including migration costs?
		\item For simplicity, assume migration costs are constant regardless of skill level (reasonable?)
		\item Migration costs essentially act to shift down the wage-skills line in the host country
		\item Acts just like a decrease in the income level in the host country (no change in selection process, but will change magnitude of migrant flow)
	\end{itemize}
\end{frame}

\begin{frame}{Self-Selection}
	\begin{itemize}
		\item Thus, parallel shifts of wage-skill lines (of either country) do not change direction of selection
		\item Only changes in returns to skills (in either country) will (potentially) change the direction of the selection process
	\end{itemize}
\end{frame}


\begin{frame}{Readings}
	\begin{itemize}
		\item Borjas 8.6
	\end{itemize}
\end{frame}


\section{Economic Benefits from Immigration}


\begin{frame}{Case 1: Perfect Substitutes}
\begin{itemize}
	\item We've already seen how immigrants can adversely impact the surplus of native workers with similar skills
	\item Additionally, we saw that perfect substitute immigration increased the surplus of native firms by driving wages down
	\item Overall, the net impact on total surplus was positive
\end{itemize}
\end{frame}


\begin{frame}{Case 1: Perfect Substitutes}
	\begin{itemize}
		\item Now, we will focus on analyzing the magnitude of gains from immigration in the host country
		\item What factors impact this gain?
		\item Do these effects last in the long run?
	\end{itemize}
\end{frame}


\begin{frame}{Case 1: Perfect Substitutes}
\begin{itemize}
	\item Consider the case of perfect substitutes if labor is supplied inelastically
	\item Effect of $M$ immigrants on employment and wages? 
	\begin{itemize}
		\item $E^* \uparrow$
		\item $w^* \downarrow$ 
	\end{itemize}
	\item Effect on surplus?
	\begin{itemize}
		\item $\downarrow$ Native WS
		\item $\uparrow$ Native FS
		\item $\uparrow$ Total surplus
	\end{itemize}
	\item Increase in national income accruing to natives if called the \textbf{immigration surplus}
	\end{itemize}
\end{frame}


\begin{frame}{Case 1: Perfect Substitutes}
	\begin{itemize}
		\item Immigration surplus arises because the wage rate equals the productivity of the \textit{last} immigrant hired
		\item Essentially, immigrants contribute at least as much as they are paid
		\item What factors impact the size of the surplus?
		\begin{itemize}
			\item Number of immigrants
			\item Elasticity of labor demand curve
		\end{itemize}
	\end{itemize}
\end{frame}


\begin{frame}{Case 1: Perfect Substitutes}
	\begin{itemize}
		\item How much does immigration add to national income?
		\[\frac{\text{Immigration Surplus}}{\text{National Income}} = \frac{1}{2} \times \%\Delta w \times \%\Delta E \times (\text{labor's share of income}) \]
		\item Note that the estimate of immigration surplus is for the short-run
		\item In the long-run, return to capital and wage rate are not affected by immigration 
		\item Thus, in long-run the immigration surplus is zero
	\end{itemize}
\end{frame}


\begin{frame}{Case 2: Complements}
	\begin{itemize}
		\item What about the case of complements?
		\item Migration of high-skill individuals may generate human capital externalities or spillovers
	\end{itemize}
\end{frame}


\begin{frame}{Case 2: Complements}
	\begin{itemize}
		\item In the case of complements, the $VMP_E$ of native workers increases $\rightarrow$ labor demand shifts right
		\item Assume spillover effect is greater than labor supply effect $\rightarrow$ labor demand increases more than labor supply
		\item Effect on employment and wages?
		\begin{itemize}
			\item $\uparrow E^*$
			\item $ \uparrow w^*$
		\end{itemize}
		\item Effect on surplus?
		\begin{itemize}
			\item $\uparrow $ Native WS
			\item $\uparrow $ Native FS
		\end{itemize}
	\end{itemize}
\end{frame}


\begin{frame}{Readings}
	\begin{itemize}
		\item Borjas 8.7-8.8
	\end{itemize}
\end{frame}
	
\end{document}