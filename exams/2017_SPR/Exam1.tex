\documentclass[addpoints,11pt]{exam}

\usepackage{alltt}
\usepackage[margin=1in]{geometry}   % set up margins
\usepackage[T1]{fontenc}
\usepackage[usenames,dvipsnames]{xcolor}
\usepackage{enumerate}              % fancy enumerate
\usepackage{amsmath}                % used for \eqref{} in this document
\usepackage{amsthm}
\theoremstyle{definition}
\newtheorem{exmp}{Example}[section]
\usepackage{verbatim}               % useful for \begin{comment} and \end{comment}
\usepackage{eurosym}                % used for euro symbol
\usepackage{caption} 
\usepackage{graphicx}
\graphicspath{{Figures/}}
\usepackage{subcaption}
\usepackage{color}
\usepackage{float}
\usepackage{amssymb}
\usepackage{MnSymbol,wasysym}
\usepackage[colorlinks=true]{hyperref}
\hypersetup{colorlinks=true, citecolor=ForestGreen, linkcolor=BlueViolet, urlcolor=Magenta}

\usepackage{array}
\newcolumntype{H}{@{}>{\lrbox0}l<{\endlrbox}}


%Solutions or nah
%\printanswers
%\newcommand{\dd}[1]{{\textbf{\textcolor{red}{#1}}}}
%\newcommand{\ddp}[1]{\par {\textcolor{ForestGreen}{#1}}}

\newcommand{\dd}[1]{}  
\newcommand{\ddp}[1]{}

\setlength\parindent{0pt}
\unframedsolutions
\SolutionEmphasis{\color{red}}
\CorrectChoiceEmphasis{\color{red}}
\renewcommand{\choicelabel}{(\alph{choice})}
\newcommand{\blank}[0]{\underline{\hspace{3cm}}}
\pointformat{\bfseries[\thepoints]}
\pointpoints{pt}{pts}
\pointsinrightmargin

\begin{document}


\title{\textbf{Exam 1 \dd{\\Solutions}} \\ \vspace{2 mm} {\large ECON 380} \\ \large{Spring 2017} \\ \large{UNC Chapel Hill}}
\date{}
\maketitle

\makebox[\textwidth]{Name:\enspace\hrulefill}
\\

\makebox[\textwidth]{ONYEN:\enspace\hrulefill}
\\

\makebox[\textwidth]{Honor Code Signature:\enspace\hrulefill}
\\

\begin{center}
	\fbox{\fbox{\parbox{6in}{\centering
				\underline{Directions:}
				\begin{itemize}
					\item For multiple choice questions, clearly circle the answer choice which best answers the question.
					\item For short answer questions, show all of your work and justify your answers where needed. 
					\item Round answers to the nearest hundredth.
					\item Assume preferences are transitive, complete, and monotone.
					\item Assume that utility functions exhibit diminishing marginal returns to consumption.
					\item Assume that leisure is income normal. 
					\item Points available: 50
					\item Write legibly, write legibly, write legibly!
					\item Good luck! \smiley{}
				\end{itemize}
			}}}
\end{center}

\newpage
	
\subsection*{Multiple Choice \textbf{[2 pts each]}}


Use the following information to answer questions \ref{q1}-\ref{q2}: 
\\

Suppose there are 25,000 individuals living in Candyland. 5,000 of these individuals are under age 16 \dd{[Not in P]}. Of the remaining 20,000 individuals, 

\begin{itemize}
	\item 8,000 work full-time in the private sector \dd{E}
	\item 2,000 work full-time in the public sector (non-military) \dd{E}
	\item 2,000 work part-time. 15\% of these part-time workers would prefer to work full-time \dd{E. 300 are part-time for ``economic reasons''} 
	\item 5,000 have been laid off in the last six months. Of these laid off individuals, 4,000 have actively sought work since being laid off, while 1,000 searched for work immediately after being laid off, but not in the last four weeks. \dd{4,000 U, 1,000 O - marginally attached}
	\item 1,000 do not have formal employment and instead choose to stay home to care for children \dd{O}
	\item 2,000 are incarcerated. \dd{Not in P} 
\end{itemize}

\dd{P = 18,000. E = 12,000, U = 4,000 $\Rightarrow$ LF = 16,000. }

\begin{questions}

\question \label{q1} The labor force participation rate according to BLS standards is \blank.

\begin{choices}
	\CorrectChoice 89\% 
	\choice 80\%
	\choice 64\%
	\choice 85\%
	\choice 75\%
\end{choices}

\begin{solution}
	LFPR = LF/P = 16,000/18,000 = 89\%.
\end{solution}

\question The \textbf{U3} unemployment rate according to BLS standards is \blank.

\begin{choices}
	\CorrectChoice 25\%
	\choice 31\%
	\choice 20\%
	\choice 29\%
	\choice 22\%
\end{choices}

\begin{solution}
	U3 = U/LF = 4,000/16,000 = 25\%
\end{solution}

\question The \textbf{U5} unemployment rate according to BLS standards is \blank.

\begin{choices}
	\choice 25\%
	\choice 31\%
	\choice 20\%
	\CorrectChoice 29\%
	\choice 22\%
\end{choices}

\begin{solution}
	U5 = (U + marginally attached)/(LF + marginally attached) = 5,000/17,000 = 29\%
\end{solution}

\newpage

\question \label{q2} The \textbf{U6} unemployment rate according to BLS standards is \blank.

\begin{choices}
	\choice 25\%
	\CorrectChoice 31\%
	\choice 20\%
	\choice 29\%
	\choice 22\%
\end{choices}

\begin{solution}
	U6 = (U + marginally attached + part-time for economic reasons)/(LF + marginally attached) = 5,300/17,000 = 31\%
\end{solution}

\question Elizabeth has rational preferences over bundles of consumption and leisure represented by a utility function $U(C,L)$. If her preferences satisfy the assumption of monotonicity, how many of the following statements \textbf{must} be true? 

\begin{itemize}
	\item $U(\$120,10) > U(\$100,10)$ 
	\item $U(\$100,20) > U(\$90,30)$
	\item $U(\$90,40) > U(\$120,20)$
	\item $U(\$130,20) > U(\$120,10)$
\end{itemize}

\begin{choices}
	\choice 2
	\CorrectChoice 1
	\choice 0
	\choice 3
	\choice 4
\end{choices}

\begin{solution}
	To satisfy monotonicity, Elizabeth's utility function must show that:
	\begin{enumerate}[(i)]
		\item Increasing $C$ or $L$ individually (holding the other variable constant) will either keep utility the same or increase it
		\item Increasing both $C$ and $L$ will increase utility
	\end{enumerate}

 Thinking of going from the bundle on the right to the bundle on the left:

	\begin{itemize}
		\item Statement 1: Does not need to be true - utility could remain the same since only $C$ was increased 
		\item Statement 2: Does not need to be true - $C$ increased, but $L$ did not remain constant
		\item Statement 3: Does not need to be true - $L$ increased, but $C$ did not remain constant
		\item Statement 4: Must be true - both $C$ and $L$ increased, so utility must increase.
	\end{itemize}
\end{solution}

\question Tom earns \$15 per hour, regardless of the number of hours he works, and faces a tax rate of 15\%. Additionally, Tom pays \$3 per hour in child care expenses for each hour he works and receives \$200 in child support payments each week. There are 110 hours in a week for Tom to allocate between work and leisure. Which of the following represents the equation for Tom's weekly budget line?

\begin{choices}
	\choice $C = 1402.5 - 12.75L$
	\choice $C = 1602.5 - 12.75L$
	\choice $C = 1850 - 15L$
	\CorrectChoice $C = 1272.5 - 9.75L$
	\choice $C = 1072.5 - 9.75L$
\end{choices}

\begin{solution}
	Tom's effective wage: $W^N = 15(1-.15) - 3 = \$9.75$. 
	$C = (wT + V) - wL = (9.75\cdot 110 + 200) - 9.75L = 1272.5 -9.75L$
\end{solution}

\question A worker has preferences given by $U(C,L) = 2C^{1/2}L^{1/2}$. If the worker is indifferent between bundle $A$, given by (\$900, 100 hours), and bundle $B$, given by (\$625, $X$), what is $X$?

\begin{choices}
	\choice 169 hours
	\choice 150 hours
	\choice 121 hours
	\choice 160 hours
	\CorrectChoice 144 hours
\end{choices}

\begin{solution}
	$U(900,100) = 2(900)^{1/2}100^{1/2} = 600$.
	$U(625,X) = 2(625)^1/2X^{1/2} = 600 \Rightarrow 50X^{1/2} = 600 \Rightarrow X^{1/2} = 12 \Rightarrow X = 144$
\end{solution}

\newpage

\question Refer to Figure \ref{fig3} below.

\begin{figure}[H]
	\centering
	\includegraphics[scale=.6]{exam1_mc}
	\caption{Harold's Budget Set}
	\label{fig3}
\end{figure}

Which of the following would cause Harold's budget constraint to change from $B^0$ to $B^1$?

\begin{choices}
	\choice An increase in both her wage rate and non-labor income. 
	\choice A decrease in both her wage rate and non-labor income. 
	\choice An increase in her wage rate and a decrease in her non-labor income.
	\CorrectChoice A decrease in her wage rate and an increase in her non-labor income. 
	\choice None of the above.
\end{choices}

\begin{solution}
	Harold's endowment point would shift up due to an increase in non-labor income. Her budget line's slope would decrease due to a decrease in her wage rate.
\end{solution}

\question How many of the following statements regarding predictions under the Neoclassical Model of Labor Supply are FALSE? 

\begin{itemize}	
	\item A decrease in non-labor income is predicted to unambiguously increase the number of hours worked for a worker currently working more than zero hours.
	\item An increase in the wage rate is predicted to unambiguously decrease the number of hours worked for a worker currently working more than zero hours.
	\item An increase in non-labor income is predicted to unambiguously increase labor force participation among workers currently out of the labor force.
	\item An increase in the wage rate is predicted to unambiguously increase labor force participation among workers currently out of the labor force.
\end{itemize}

\begin{choices}
	\CorrectChoice 2
	\choice 1
	\choice 0
	\choice 3
	\choice 4
\end{choices}

\begin{solution}
	\begin{itemize}
		\item Statement 1: True (assuming leisure is income normal)
		\item Statement 2: False. An increase in the wage rate may increase hours worked if the substitution effect is larger than the income effect.
		\item Statement 3: False. An increase in non-labor income is predicted to decrease labor force participation because it raises a worker's reservation wage (assuming diminishing marginal returns to consumption).
		\item Statement 4: True. Increasing the wage rate will raise the wage rate above the reservation wage of some workers.
	\end{itemize}
\end{solution}

\question The neoclassical model of labor supply predicts that the Earned Income Tax Credit should unambiguously

\begin{choices}
	\choice decrease labor force participation.
	\choice increase work hours among workers currently working.
	\CorrectChoice increase labor force participation
	\choice decrease work hours among workers currently working.
\end{choices}

\begin{solution}
	By increasing a worker's net wage, the EITC is predicted to increase labor force participation. However, the effect of the EITC on work hours for those already working is ambiguous.
\end{solution}


\end{questions}

\subsection*{Short Answer}


\begin{questions}
	
\question For each of the following, determine which type of unemployment is present.

\begin{parts}
	\part[1] Jonathan decided to leave his job as chocolatier three months ago in order to pursue a career as a pastry chef. He is actively looking for work, but it is taking time for him to search for job openings, fill out applications, and hear back from interested firms. 
	\begin{solution}[1in]
		Frictional unemployment
	\end{solution}
	\part[1] Jill worked as a licorice maker for 25 years, but was laid off a year ago because firms in the industry transitioned to automated processes. She has looked for work since then, but has not found employment because her skills as a licorice maker are not readily transferable to other sectors in the economy. 
	\begin{solution}[1in]
		Structural unemployment
	\end{solution}
	\part[1] Tina is currently looking for work as a Barista. She only started looking for work a few weeks ago, but it seems that most coffee shops are still recovering from an economic downturn and are hesitant to hire.
		\begin{solution}[1in]
		Cyclical unemployment
	\end{solution}
	\part[1]  The city council of Ski Mountain Resort observes that unemployment in the region increases during the summer months.
	\begin{solution}[1in]
	Seasonal unemployment
\end{solution}
\end{parts}	


\newpage

\question Charlie has 5,000 hours per year to allocate between work and leisure. If he works, he can earn a gross hourly wage, $w^G$, and he faces the following marginal tax rates on his gross earnings:

\begin{table}[H]
	\caption{Marginal Tax Rates}
	\centering
	\begin{tabular}{ c| c} 
		
		Marginal Tax Rate &  Gross Earnings\\
		\hline
		10\% & \$15,000 \\
		20\% & \$15,001 - \$40,000  \\
		25\% & \$40,001 - \$90,000 \\
		30\% & \$90,001+ \\
	\end{tabular}
	\label{MC30}
\end{table}

Finally, Charlie earns \$5,000 a year in non-labor income. This income is not taxed. Figure \ref{fig1} shows Charlie's budget line for the year. Note that points B and C represent ``kink'' points in his budget line where his marginal tax rate changes.

\begin{figure}[H]
	\centering
	\includegraphics[scale=.40]{charlie.png}
	\caption{Charlie's Budget Line}
	\label{fig1}
\end{figure}

\begin{parts}
\part[4] What is Charlie's \textbf{net} hourly wage between points $B$ and $C$ on his budget line?
\begin{solution}[1in]
	$w^N$ is the (absolute) slope of the budget line. Between $B$ and $C$, $w^N = |(38,500 - 18,500)/(1000-3,500)| = \$8$.
\end{solution}
\part[4] What is Charlie's \textbf{gross} hourly wage, $w^G$?	
\begin{solution}[1in]
	$w^N = w^G(1-\tau) \Rightarrow w^G = w^N/(1-\tau) = \$8/(1-.20) = \$10$.
\end{solution}
\end{parts}


\newpage

\question Dennis' daily budget line is denoted $B^1$ in Figure \ref{fig2} below. Suppose his marginal rate of substitution is given by $MRS_{L,C} = \frac{2C}{3L}$.

\begin{figure}[H]
	\centering
	\includegraphics[scale=.45]{dennis.png}
	\caption{Dennis' Budget Line}
	\label{fig2}
\end{figure}

%\begin{figure}[H]
%	\centering
%	\includegraphics[scale=.8]{dennis_ans.png}
%	\caption{Dennis' Budget Line}
%	\label{fig2}
%\end{figure}

\begin{parts}
\part[2] What is Dennis' reservation wage?
\begin{solution}[.6in]
$w^{res} = MRS$ at the endowment point, where $C = V = 100$ and $L = T = 24$. $w^{res} = 2(100)/3(24) = 200/72 = \$2.78$.
\end{solution}
\part[4] Suppose bundle $X$ gives Dennis' optimal choice of consumption and leisure. Explain how Dennis could rearrange his bundle to increase his utility if he is currently at point $Y$. What is the relationship between the $MRS_{L,C}$ and Dennis' wage ($w$) at point $Y$? 
\begin{solution}[1.5in]
Bundle $X$ is the optimal bundle, so $MRS = w$ at point $X$. At point $Y$, Dennis could increase his utility by consuming more dollars and taking less leisure time. That is, the next hour of leisure would yield less utility per dollar spent than the next dollar of consumption spent at point $Y$: $(MU_L/w) < MU_C \Rightarrow (MU_L/MU_C) < w \Rightarrow MRS_{L,C} < w$.

\ddp{Points: (2) $MRS<w$, (2) how to rearrange bundle (increase $C$, decrease $L$)}
\end{solution}

\part[2] Sketch Dennis' indifference curves going through points $X$ and $Y$. You don't need to use his actual preferences, just general convex indifference curves.
\ddp{(1) each. Make sure the one going through point $X$ is tangent to the budget line and that the indifference curves don't cross. The slope of the indifference curve going through $Y$ should be smaller than the slope of $B^1$ at point $Y$.}
\part[4] Suppose Dennis' budget line changes to $B^2$. If Dennis' new optimal bundle of consumption and leisure is given by point $Z$, what does that tell you about the relationship between the income and substitution effects? 

\begin{solution}[.75in]
Dennis' wage increased (since his budget line became steeper). At this higher wage, Dennis will increase his leisure time, i.e., he decreases how many hours he works. Thus, it must be that the income effect dominates the substitution effect. 
\end{solution}
\end{parts}


\newpage


\question Art Sloan produces Flip Cups according to the following production function:

\[q=f(K,E)=2K^{1/2}E^{1/2}\]

All markets are perfectly competitive, and Sloan currently has 5 units of capital.  The marginal product of labor is given by $MP_E=\frac{K^{1/2}}{E^{1/2}}$.  His Flip Cups sell for $p=\$20$, the market wage rate is $w=\$10$, and the capital rental rate is $r=\$5$.   

\begin{parts}
	\part[3] Determine Sloan's optimal short-run level of labor employment.
	\begin{solution}[2in]
		Optimal hiring rule: $VMP_E = w \Rightarrow p\times MP_E = w$. $K=K_0 = 5$.
		\\
		$20\times \frac{5^{1/2}}{E^{1/2}} = 10 \Rightarrow 10E^{1/2} = 20(5^{1/2}) \Rightarrow E^{1/2} = 2(5^{1/2}) \Rightarrow E^* = 4\times 5 = 20$.
	\end{solution}
	
	\ddp{Points: (1) $VMP_E = w$, (1) work, (1) $E^*=20$}
	
	\part[3] Now, assume that Sloan is able to vary his capital stock (i.e., we've moved to the long-run).  He notices that at his current input bundle, $MP_E=0.5$, $MP_K=2$, and prices are still $w=\$10$, $r=\$5$.  Explain how he can alter his inputs to increase his profits.
\begin{solution}[3in]
$MP_E/w = 0.05$ and $MP_K/r = 0.4$. Capital produces more additional output per dollar spent, so Sloan should reallocate resources towards \underline{capital} in order to increase his profits $(\uparrow K, \downarrow L)$.
\end{solution}

	\ddp{Points: (1) Reallocate towards capital (2) Explanation}
	
\end{parts}
	
\end{questions}


\end{document}