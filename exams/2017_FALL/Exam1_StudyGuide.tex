\documentclass[addpoints,11pt]{exam}

\usepackage{alltt}
\usepackage[margin=1in]{geometry}   % set up margins
\usepackage[T1]{fontenc}
\usepackage[usenames,dvipsnames]{xcolor}
\usepackage{enumerate}              % fancy enumerate
\usepackage{amsmath}                % used for \eqref{} in this document
\usepackage{amsthm}
\theoremstyle{definition}
\newtheorem{exmp}{Example}[section]
\usepackage{verbatim}               % useful for \begin{comment} and \end{comment}
\usepackage{eurosym}                % used for euro symbol
\usepackage{caption} 
\usepackage{graphicx}
\usepackage{threeparttable}
\graphicspath{{Figures/}}
\usepackage{subcaption}
\usepackage{booktabs}
\usepackage{color}
\usepackage{float}
\usepackage{amssymb}
\usepackage{sgamevar}
\usepackage{sgame}
\usepackage[colorlinks=true]{hyperref}
\hypersetup{colorlinks=true, citecolor=ForestGreen, linkcolor=BlueViolet, urlcolor=Magenta}

\usepackage{array}
\newcolumntype{H}{@{}>{\lrbox0}l<{\endlrbox}}


%Solutions or nah (blank next two lines out for no solutions, unblank #3)
\printanswers
\newcommand{\dd}[1]{{\textbf{\textcolor{red}{#1}}}}
\newcommand{\ddp}[1]{\par {\textcolor{ForestGreen}{#1}}}

%\newcommand{\dd}[1]{}  
%\newcommand{\ddp}[1]{}

\setlength\parindent{0pt}
\unframedsolutions
\SolutionEmphasis{\color{red}}
\CorrectChoiceEmphasis{\color{red}}
\renewcommand{\choicelabel}{(\alph{choice})}
\newcommand{\blank}[0]{\underline{\hspace{3cm}}}
\pointformat{\bfseries[\thepoints]}
\pointpoints{pt}{pts}
\pointsinrightmargin

\begin{document}
	
	
	\title{\textbf{Exam 1 Study Guide} \\ \vspace{2 mm} {\large ECON 380} \\ \large{UNC Chapel Hill}}
	\date{}
	\maketitle
	
Useful stuff to know for exam 1:

\subsection*{Labor Force Accounting \& Unemployment}

\begin{enumerate}
	\item How to classify individuals - E, U, O, or not in P
	\item What are discouraged workers? Marginally attached workers?
	\item How to compute the LFPR
	\item Differences between U3/U4/U5/U6 unemployment rates and how to compute them
	\item What are the major issues with the official unemployment rate?
	\item What are the different types of unemployment? 
	\item What policies might be most effective at combating different unemployment types? 
\end{enumerate}

\subsection*{Neoclassical Model of Labor Supply}

\begin{enumerate}
	\item Properties of indifference curves
	\item What do $MU_L$ and $MU_C$ measure?
	\item What is $MRS_{L,C}$ in words? Graphically?
	\item How to write the equation for an individual's budget constraint and how to graph a budget constraint (CHARLIE)
	\item What is the price of leisure? 
	\item How do changes in (i) non-labor income and (ii) wages affect the budget constraint?
	\item How does a worker choose the optimal number of work hours? Should know how to solve algebraically and how to show graphically.
	\item What's the intuition behind the tangency condition? How should a worker reallocate his bundle if $MRS_{L,C} < w$? If $MRS_{L,C} > w$?
	\item How do changes in non-labor income affect the hours-worked decision if leisure is income normal?
	\item How do changes in the wage rate affect the hours-worked decision? Is the overall impact unambiguous? 
	\item What happens to hours-worked if the income effect dominates the substitution effect? What if the substitution effect dominates the income effect? What do the graphs look like for each case?
	\item What is the endowment point? 
	\item What is the reservation wage (in words)? How to calculate it?
	\item How do changes in non-labor income affect the reservation wage? Changes in the wage rate?
	\item How to compute elasticity of labor supply. 
	\item What does the magnitude of labor supply elasticity tell you? What does the sign tell you?
	\item What are two of the main issues with empirically estimating the elasticity of labor supply?
	\item What are the main theoretical predictions under (i) AFDC/TANF and (ii) the Earned Income Tax Credit
\end{enumerate}

\subsection*{Labor Demand}

\begin{enumerate}
	\item What do we assume is the goal of a firm?
	\item What is the condition to find the optimal number of workers to hire in the short-run? 
	\item Solving for optimal number of workers to hire in SR 
	\item Condition to optimally allocate labor and capital in the LR.
	\item How to reallocate labor and capital given $r$, $w$, and $MRTS$. Know how to explain why the reallocation is more efficient
	\item Relationship between short-run and long-run demand elasticity? Why are they different?
\end{enumerate}

\subsection*{Competitive Markets}

\begin{enumerate}
	\item How to find equilibrium wage and employment level given labor supply and demand equations
	\item How to find employment and unemployment level under a binding minimum wage
	\item How to find producer, worker, and total surplus from a graph
	\item How to find producer, worker, total surplus, and DWL under a binding minimum wage
	\item Be able to explain why the allocation of resources (i.e., the equilibrium employment level) is efficient in a labor market with no externalities
	\item How the free entry/exit of workers (or firms) across regions will eventually lead to a single equilibrium wage
\end{enumerate}

\subsection*{Immigration Impacts}

\begin{enumerate}
	\item Short-run vs long-run wage \& employment effects of migration in the case of perfect substitutes and how to reflect this on a graph
	\item Short-run vs long-run wage \& employment effects of migration in the case of complements and how to reflect this on a graph
	\item How to calculate native worker and native firm surplus before and after immigration in the case of perfect substitutes (short-run)
\item How to calculate immigration surplus
	\item Welfare implications if migrants are perfect substitutes: who is better off, who is worse off? What is the effect on total welfare?
\end{enumerate}

\subsection*{Noncompetitive Labor Markets}

\begin{enumerate}
	\item Welfare implications in a market with a perfectly discriminating monopolist
	\item Under a non-discriminatory monopolist, what is the relationship between the marginal cost of hiring labor and the wage rate?
	\item What is the implication of this relationship on the number of workers employed when compared to a perfectly competitive market or perfectly discriminating monopsonist 
	\item What are the welfare impacts under a non-discriminating monoposonist?
	\item How to find PS, WS, TS, and DWL from a graph under this structure
	\item Can a minimum wage improve the allocation of resources in this case?
	\item What is the ``optimal'' minimum wage that maximizes total surplus?
\end{enumerate}

\subsection*{LMDCs}

\begin{enumerate}
	\item Know 2-3 of the main differences between labor markets in developed and developing countries as described by Fields (2011)
	\item Know 2-3 of the differences between ``traditional'' and ``modern'' labor markets from Campbell \& Ahmed (2012)
	\item What are the two possible explanations for why labor markets are segmented?
	\item Arias \& Khamis (2008) found evidence for both segmentation stories. In which sector (self-employed or informal work) did they find evidence for the ``exclusion view'' and in which did they find evidence for the comparative advantage story?
	\item Why might the income effect due to a wage drop be stronger in developing countries? (Jayachandran 2006)? 
	\item What were the results on women's labor market participation as a result of the program analyzed in Jensen (2012)?
\end{enumerate}

\end{document}