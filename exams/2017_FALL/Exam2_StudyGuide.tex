\documentclass[addpoints,11pt]{exam}

\usepackage{alltt}
\usepackage[margin=1in]{geometry}   % set up margins
\usepackage[T1]{fontenc}
\usepackage[usenames,dvipsnames]{xcolor}
\usepackage{enumerate}              % fancy enumerate
\usepackage{amsmath}                % used for \eqref{} in this document
\usepackage{amsthm}
\theoremstyle{definition}
\newtheorem{exmp}{Example}[section]
\usepackage{verbatim}               % useful for \begin{comment} and \end{comment}
\usepackage{eurosym}                % used for euro symbol
\usepackage{caption} 
\usepackage{graphicx}
\usepackage{threeparttable}
\graphicspath{{Figures/}}
\usepackage{subcaption}
\usepackage{booktabs}
\usepackage{color}
\usepackage{float}
\usepackage{amssymb}
\usepackage{sgamevar}
\usepackage{sgame}
\usepackage[colorlinks=true]{hyperref}
\hypersetup{colorlinks=true, citecolor=ForestGreen, linkcolor=BlueViolet, urlcolor=Magenta}

\usepackage{array}
\newcolumntype{H}{@{}>{\lrbox0}l<{\endlrbox}}


%Solutions or nah (blank next two lines out for no solutions, unblank #3)
\printanswers
\newcommand{\dd}[1]{{\textbf{\textcolor{red}{#1}}}}
\newcommand{\ddp}[1]{\par {\textcolor{ForestGreen}{#1}}}

%\newcommand{\dd}[1]{}  
%\newcommand{\ddp}[1]{}

\setlength\parindent{0pt}
\unframedsolutions
\SolutionEmphasis{\color{red}}
\CorrectChoiceEmphasis{\color{red}}
\renewcommand{\choicelabel}{(\alph{choice})}
\newcommand{\blank}[0]{\underline{\hspace{3cm}}}
\pointformat{\bfseries[\thepoints]}
\pointpoints{pt}{pts}
\pointsinrightmargin

\begin{document}
	
	
	\title{\textbf{Exam 2 Study Guide} \\ \vspace{2 mm} {\large ECON 380} \\ \large{UNC Chapel Hill}}
	\date{}
	\maketitle
	
Stuff to know for exam 2:


\subsection*{Human Capital Theory}

\begin{enumerate}
	\item What has been the trend in the proportion of people without a HS degree over the last 70 years? In the proportion of people with a college degree?
	\item How has the relative pay of those with a college degree compared to those without a HS degree changed since 1975? What about the relative pay of those with some college compared to those with no HS degree?
	\item How to find present value and net present value
	\item How to find an individual's optimal choice from two choices with different costs/future earnings
	\item What is the relationship between $r$ and an individual's college choice?
\end{enumerate}

\subsection*{The Schooling Model}

\begin{enumerate}
	\item How to find $MRR$ given an individual's wage-schooling locus (both from a table or a graph)
	\item What are the properties of the $MRR$ curve?
	\item How to find individual stopping choice given $MRR$ schedule and an individual's discount rate (both from a table or a graph)
	\item How does an individual's choice change with $r$?
\end{enumerate}

\subsection*{Estimating the Returns to Schooling}

\begin{enumerate}
	\item Why might two individuals differ in their schooling choices?
	\item If individuals have different rates of discount, how does that affect their optimal schooling choice? Can we accurately estimate $MRR$ if individuals differ only in their discount rate?
	\item If individuals have different ability levels, how does that affect their optimal schooling choice? Can we accurately estimate $MRR$ if individuals differ in ability?
	\item Why might unobserved ability bias $MRR$ estimates upwards? Downwards?
	\item What methods have been attempted to deal with ability bias? What are some of the drawbacks of each?
	\item Know 2-3 of the other issues that come up when estimating the returns to schooling 
\end{enumerate}

\subsection*{Job Market Signaling}

\begin{enumerate}
	\item What is the basic idea of the signaling model?
	\item Under a pooling equilibrium, who is better off? Worse off?
	\item Know how to find threshold level of earnings or school level given information about worker productivity and costs of education (e.g., homework question and example from class)
	\item Can we easily separate the productivity enhancing component of schooling from the the ``sheepskin effect''? Why or why not?
\end{enumerate}

\subsection*{Human Capital and Development}

\begin{enumerate}
	\item Homework 4 questions about Duflo (2001) and Jensen (2010)
\end{enumerate}


\subsection*{Labor Market Discrimination}

\begin{enumerate}
	\item What is labor market discrimination?
	\item What are pre-labor market differences?
	\item What has been the trend in the black-white wage since the 1970s (shrinking gap, rising, neither)?
	\item What has been the trend in the male-female wage since the 1970s (shrinking gap, rising, neither)? 
\end{enumerate}

\subsection*{Taste-Based Discrimination}

\begin{enumerate}
	\item What is taste-based discrimination?
	\item Know the graphs from class related to employer-based discrimination 
	\item Who are the ``winners'' and ``losers'' from the presence of discriminatory firms?
	\item What is the relationship between the profits of discriminatory firms vs ND firms? 
	\item What is the long-run outcome predicted by employer-discrimination?
	\item Does employer-based discrimination explain the existence of wage gaps? 
	\item Does employee-based discrimination explain the existence of wage gaps?
	\item Does customer-based discrimination explain the existence of wage gaps?
\end{enumerate}

\subsection*{Statistical Discrimination}

\begin{enumerate}
	\item What is statistical discrimination?
	\item Know graphs from class related to statistical discrimination
	\item Implications of statistical discrimination if average test scores are the same, but the the signal is more ``noisy'' for one group? Who benefits? Who is hurt?
	\item Can a ``noisy'' signal (even if test scores, on average, are equal) lead to wage gaps? Will average wages be different?
	\item Implications of statistical discrimination if the average test score of one group is lower than the other (assuming the signal is just as ``noisy'' for each group)? 
	\item How do different average test scores lead to wage gaps? Does this hold in the long-run? Why?
\end{enumerate}

\subsection*{Measuring Discrimination}

\begin{enumerate}
	\item How to interpret linear wage schooling locus
	\item How different coefficients are evidence of discrimination 
	\item Oaxaca-Blinder decomposition: pre-market component and discrimination component
	\item Know graph from class and interpretation: discrimination component is $w_m^* - \bar{w}_m$ (what average minority worker would receive if ``treated as white worker'' minus average minority worker wage); pre-market component is $\bar{w}_w - w_m^*$
\end{enumerate}

\subsection*{Discrimination Evidence}

\begin{enumerate}
	\item Know 2-3 of the potential issues that arise when estimating the typical earnings model testing for discrimination
	\item Neal \& Johnson (1996): What do the authors find drives a large portion of the black-white wage gap? Do they argue the wage gap is largely driven by discrimination or pre-labor market factors?
	\item Altonji \& Pierret (2001): Do the authors find evidence that firms statistically discriminate on the basis of race? 
	\item Bertrand \& Mullainathan (2004): Know basic set-up of the study. How did the callback rates for ``white sounding names'' compare to that of ``black sounding names?'' 
	\item In recent years, how has the role of explicit labor market discrimination in driving different socio-economic outcomes changed? Will policies targeting labor market discrimination or policies targeting school achievement gaps be more effective in closing wage gaps and other differences in socio-economic outcomes?
	\item What is likely the greatest mechanism driving the gender wage gap? How does the shape of the relative earnings curve between males and females over the life-cycle reflect this?
	\item What is occupational crowding? Does it necessarily play a large role in driving the gender wage gap?
\end{enumerate}



\end{document}