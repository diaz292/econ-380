\documentclass[addpoints,11pt]{exam}

\usepackage{alltt}
\usepackage[margin=1in]{geometry}   % set up margins
\usepackage[T1]{fontenc}
\usepackage[usenames,dvipsnames]{xcolor}
\usepackage{enumerate}              % fancy enumerate
\usepackage{amsmath}                % used for \eqref{} in this document
\usepackage{amsthm}
\theoremstyle{definition}
\newtheorem{exmp}{Example}[section]
\usepackage{verbatim}               % useful for \begin{comment} and \end{comment}
\usepackage{eurosym}                % used for euro symbol
\usepackage{caption} 
\usepackage{graphicx}
\graphicspath{{Figures/}}
\usepackage{subcaption}
\usepackage{color}
\usepackage{float}
\usepackage{amssymb}
\usepackage{MnSymbol,wasysym}
\usepackage[colorlinks=true]{hyperref}
\hypersetup{colorlinks=true, citecolor=ForestGreen, linkcolor=BlueViolet, urlcolor=Magenta}

\usepackage{array}
\newcolumntype{H}{@{}>{\lrbox0}l<{\endlrbox}}


%Solutions or nah
%\printanswers
%\newcommand{\dd}[1]{{\textbf{\textcolor{red}{#1}}}}
%\newcommand{\ddp}[1]{\par {\textcolor{ForestGreen}{#1}}}

\newcommand{\dd}[1]{}  
\newcommand{\ddp}[1]{}

\setlength\parindent{0pt}
\unframedsolutions
\SolutionEmphasis{\color{red}}
\CorrectChoiceEmphasis{\color{red}}
\renewcommand{\choicelabel}{(\alph{choice})}
\newcommand{\blank}[0]{\underline{\hspace{3cm}}}
\pointformat{\bfseries[\thepoints]}
\pointpoints{pt}{pts}
\pointsinrightmargin

\begin{document}


\title{\textbf{Exam 2 \dd{\\Solutions}} \\ \vspace{2 mm} {\large ECON 380} \\ \large{Fall 2016} \\ \large{UNC Chapel Hill}}
\date{}
\maketitle

\makebox[\textwidth]{Name:\enspace\hrulefill}
\\

\makebox[\textwidth]{ONYEN:\enspace\hrulefill}
\\

\makebox[\textwidth]{Honor Code Signature:\enspace\hrulefill}
\\

\begin{center}
	\fbox{\fbox{\parbox{6in}{\centering
				\begin{itemize}
					\item For partial credit, show all of your work on the following pages, and justify your answers where needed.
					\item Round answers to the nearest hundredth.
					\item Points available: 50
					\item Write legibly, write legibly, write legibly!
					\item Good luck! \smiley{}
				\end{itemize}
			}}}
\end{center}

\newpage
	
\subsection*{Competitive Markets and Immigration}

\begin{questions}
	
\question Suppose the labor market in the nation of San Marcos is composed entirely of blue-collar workers.  Initially, the labor market is in equilibrium in 2016 as illustrated in the following diagram:

\begin{figure}[H]
	\centering
	\includegraphics[scale=.75]{mig}
\end{figure}

%\begin{figure}[H]
%	\centering
%	\includegraphics[scale=.90]{miga}
%\end{figure}

\begin{parts}
	\part[2] Suppose that a massive group of immigrants is allowed to enter San Marcos.  These immigrants are composed entirely of blue-collar workers.  Graphically, shift curve(s) to illustrate the short-run effect of this immigration influx on wages and employment levels.
	\part[3] What will be the short-run impact on each of the following: (i) native worker surplus, (ii) firm surplus, and (iii) total surplus?
	\begin{solution}[1.5in]
		\begin{enumerate}[(i)]
			\item Native worker surplus will decrease (lower wage \& lower number employed)
			\item Firm surplus will increase (lower wage \& greater number employed)
			\item Total surplus will increase (the decrease in native worker surplus is offset by higher firm surplus and surplus to migrant workers)
		\end{enumerate}
	\end{solution}
	\part[4] Explain why this effect on wages as illustrated in part (a) need not hold in the long-run, and illustrate this graphically.
	\begin{solution}[1.5in]
		In the long-run, labor demand will increase. Immigrant workers will demand goods/services from firms, increasing demand for output in San Marcos. In turn, this will increase demand for labor (higher price of output $\Rightarrow$ increased demand for labor).
		\ddp{Points: (1) curve shift on graph, (3) explanation}
	\end{solution}
\end{parts}

\end{questions}

\subsection*{Noncompetitive Labor Markets}

\begin{questions}
	
	\question The labor market in Saxapahaw is governed by the same labor supply and labor demand curves, but is instead a monopsonistic labor market, where the only firm hiring labor is the Saxapahaw Cotton Mill.  
	
\begin{figure}[H]
	\centering
	\caption{Monopsony}
	\begin{subfigure}{.33\textwidth}
		\includegraphics[scale=.5]{mono}
		\caption{Unregulated Market}
		\label{partb}
	\end{subfigure}
		\begin{subfigure}{.33\textwidth}
		\includegraphics[scale=.5]{mono}
		\caption{Regulated Market}
		\label{partc}
	\end{subfigure}
\end{figure}	

	
	
%\begin{figure}[H]
%	\centering
%	\caption{Monopsony}
%	\begin{subfigure}{.33\textwidth}
%		\includegraphics[scale=.55]{monoa}
%		\caption{Unregulated Market}
%		\label{partb}
%	\end{subfigure}
%	\begin{subfigure}{.33\textwidth}
%		\includegraphics[scale=.55]{monob}
%		\caption{Regulated Market}
%		\label{partc}
%	\end{subfigure}
%\end{figure}	


\begin{parts}
	\part[4] Assuming Saxapahaw Cotton Mill is a non-discriminating monopsonist in an unregulated labor market, determine their optimal employment level and their wage rate (write this in the space below).  On Figure \ref{partb} above, label the firm surplus, worker surplus, and deadweight loss resulting in the unregulated monopsony scenario.
		\begin{solution}[.5in]
			A non-discriminating monopolist hires all workers at the same wage rate. Their optimal employment level is where $MC_E = VMP_E$. Graphically, this is where the $MC_E$ curve meets the labor demand curve. $Q_E^* \approx 50$. They pay workers $\sim$\$8 in order to hire this optimal amount.
			\ddp{Points: (1) wage/employment level, (1) each for WS, PS, and DWL on the graph. For all of these, if they are close you can give them the points - it was a little hard to precisely read the graphs.}
	\end{solution}
	\part[4] Now, suppose the government of Saxapahaw imposes a minimum wage of \$14.   Determine Saxapahaw Cotton Mill's optimal employment level and their wage rate under this policy (write this in the space below).  On Figure \ref{partc} above, label the firm surplus, worker surplus, and deadweight loss resulting in this scenario.
		\begin{solution}[.5in]
		Wage = \$14. $Q^* \approx 60$. 
	\end{solution}
	\part[3] What is the optimal minimum wage the government of Saxapahaw should enact in order to eliminate the deadweight loss associated with the monopsonist? How many workers will be employed at this wage?
	\begin{solution}[.5in]
		The optimal minimum wage is the one that would induce the monopsonist to hire the same number of workers as a competitive market (i.e., where $E_S = E_D$). This corresponds to a minimum wage of \$10. At this wage rate, the firm will hire $\sim 80$ workers.
		\ddp{Points: (2) correct minimum wage, (1) corresponding employment level}
	\end{solution} 
\end{parts}


\end{questions}

\subsection*{Human Capital Theory}

\begin{questions}
	
	\question Suppose Shirley lives for three periods, $t=1,2,3$. In period 1, she can either enter directly enter to labor force or she can go to college. If she enters the labor force in period 1, she will earn \$45,000, \$90,000, and \$80,000 in periods 1, 2, and 3, respectively. Instead, if she goes to college she will have to pay \$50,000 in tuition in period 1, but then will earn \$140,000 in both of the following periods. 
	
	\begin{parts}
		\part[4] Write, but do not calculate, Shirley's net present value of each choice if her discount rate is 5\%.
		\begin{solution}[1.5in]
			\[NPV_{C} = -50,000 + \frac{140,000}{(1.05)} + \frac{140,000}{(1.05)^2}\]
			\[NPV_{NC} = 45,000 + \frac{90,000}{(1.05)} + \frac{80,000}{(1.05)^2}\]
			\ddp{Points: (2) each. I was little less clear than I should have been here. If they only calculated the NPV of each choice and only wrote that, they can get credit, \textit{but only if they calculated them correctly}.}
		\end{solution}
		\part[2] Shirley would be indifferent between the two choices if her discount rate was $\sim$10\%. Given this, what is her optimal choice at her actual discount rate of 5\%? What would be her choice if her actual discount rate were 15\%?
		\begin{solution}[1in]
			At a discount rate of 5\%, Shirley would go to college ($r < r_{id}$), while at a discount rate of 15\%, Shirley would directly enter the labor force ($r > r_{id}$).
			\ddp{Points: (1) for choice with $r=5\%$, (1) for choice with $r=15\%$}
		\end{solution}
	\end{parts}
	
\end{questions}

\subsection*{The Schooling Model}

\begin{questions}
	
	\question Leslie's wage-schooling locus is presented in the chart below.
	
		\begin{table}[H]
		\centering
		\caption{Leslie's Wage-Schooling Locus}
		\label{hi2}
		\begin{tabular}{c|c|c}
			Years of Schooling & Earnings & MRR \\
			\hline 
			11 & \$36,000 & -----\\
			12 & \$40,000 & \dd{11.1\%} \\
			13 & \$43,500 & \dd{8.8\%} \\
			14 & \$46,000 & \dd{5.7\%} \\
			15 & \$48,000 & \dd{4.3\%} \\
		\end{tabular}
	\end{table}

\begin{parts}
	\part[2] Fill in the chart with Leslies's marginal rate of return for years 12 through 15.
	\part[2] Suppose that Leslie's discount rate is r = 6\%.  Determine her optimal level of schooling attainment. 
	\begin{solution}[.25in]
		Leslie should go to school for an additional year as long as $MRR \ge r$. Thus, she should obtain 13 years of schooling.
	\end{solution}
\end{parts}

\question[5] Suppose we do not observe Leslie's full wage-schooling locus, but only her actual years of schooling and earnings as found in Question 1, part (b). We also observe Ron's wage-schooling outcome, which is different from Leslie's. In the context of human capital theory, why might we observe Leslie and Ron make different schooling choices?
\begin{solution}[1in]
	In human capital theory, Ron and Leslie make different schooling choices due to any of the following: (i) Ron and Leslie have different discount rates, (ii) Ron and Leslie have different ability levels (i.e., are on different wage-schooling loci), or both (i) and (ii).
	
\ddp{(2.5) each for (i) and (ii). They didn't have to explicitly mention that it could be both (i) and (ii) simultaneously.}
\end{solution}
\end{questions}

\subsection*{Signaling}

\begin{questions}
	
	\question[5] Suppose the market is populated by two types of workers.  Barts have an inherently low productivity level and make up 70\% of the population.  A Bart produces \$1,000,000 for a firm over their lifetime.  Lisas, on the other hand, have an inherently high productivity level and make up 30\% of the population.  A Lisa produces \$1,200,000 for a firm over their lifetime. Earning a college degree costs \$250,000 for a Bart, while the same degree costs \$50,000 for a Lisa.
	
	Firms do not observe a worker's type, but do observe their schooling level. If firms decide to pay \$1,300,000 in lifetime wages to a worker with a college degree, but only \$950,000 to a worker without a college degree, will this wage structure create an equilibrium such that a college degree will perfectly signal a worker's productivity level? Explain why or why not.
	
	\begin{solution}[1.5in]
			\begin{table}[H]
			\centering
			\caption{Barts' \& Lisas' Earnings}
			\begin{tabular}{c|c|c}
				 & Net Lifetime Earnings with degree & Net Lifetime Earnings without degree \\
				\hline 
				Barts & \$1,300,000 - \$250,000 = \dd{\$1,050,000} & \$950,000 \\
				Lisas & \$1,300,000 - \$50,000 = \dd{\$1,250,000} & \$950,000 \\
			\end{tabular}
		\end{table}
	No, a college degree will not be a signal of productivity under this wage structure. Both Barts and Lisas will obtain a degree because their lifetime earnings would be higher with a degree than without one. Firms are paying those with a degree too much, which induces the Barts to obtain a degree. The most firms can pay in order to create a separating equilibrium is \$1,199,999.99.
	\end{solution}
\ddp{Points: (2) ``No'', (2) Causes both groups to obtain a degree, (1) Earnings for those with a degree are too high}
\end{questions}

\subsection*{Labor and Development}

\ddp{For these questions, as long as they get the main gist of it them should receive most of the points.}

\begin{questions}
	
	\question[5] In 3-5 sentences, briefly describe the two possible explanations as to why labor markets in developing countries might be ``segmented.'' 
	\begin{solution}[2in]
		\begin{enumerate}
		\item ``Exclusion view'': Those that are self-employed as well as workers in the informal sector would prefer the higher wages and benefits of the formal labor market sector, but are excluded from participating due to labor market rigidities or other barriers to entry.
		\item Markets are not truly ``segmented,'' but rather workers sort into different labor markets voluntarily due to comparative advantage considerations (workers self-select into different labor markets).
	\end{enumerate}	
\end{solution}	
	\question[5] In 2-3 sentences, briefly describe how \underline{access} to education and health care has changed over the last $\sim$50 years in low-income countries.
	\begin{solution}[2in]
		In the last 50 or so years, access to both education (especially primary education) and health care has increased \underline{substantially} across developing countries. This trend is true across geographic regions, though some have seen greater increases to access than others (e.g., sub-Saharan Africa still lags behind) and access isn't necessarily available to all groups (e.g., gender disparity South Asian secondary education and lack of health care access for the very poor across regions).
\end{solution}


\end{questions}

\end{document}