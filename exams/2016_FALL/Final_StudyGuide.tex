\documentclass[addpoints,11pt]{exam}

\usepackage{alltt}
\usepackage[margin=1in]{geometry}   % set up margins
\usepackage[T1]{fontenc}
\usepackage[usenames,dvipsnames]{xcolor}
\usepackage{enumerate}              % fancy enumerate
\usepackage{amsmath}                % used for \eqref{} in this document
\usepackage{amsthm}
\theoremstyle{definition}
\newtheorem{exmp}{Example}[section]
\usepackage{verbatim}               % useful for \begin{comment} and \end{comment}
\usepackage{eurosym}                % used for euro symbol
\usepackage{caption} 
\usepackage{graphicx}
\usepackage{threeparttable}
\graphicspath{{Figures/}}
\usepackage{subcaption}
\usepackage{booktabs}
\usepackage{color}
\usepackage{float}
\usepackage{amssymb}
\usepackage{sgamevar}
\usepackage{sgame}
\usepackage[colorlinks=true]{hyperref}
\hypersetup{colorlinks=true, citecolor=ForestGreen, linkcolor=BlueViolet, urlcolor=Magenta}

\usepackage{array}
\newcolumntype{H}{@{}>{\lrbox0}l<{\endlrbox}}


%Solutions or nah (blank next two lines out for no solutions, unblank #3)
\printanswers
\newcommand{\dd}[1]{{\textbf{\textcolor{red}{#1}}}}
\newcommand{\ddp}[1]{\par {\textcolor{ForestGreen}{#1}}}

%\newcommand{\dd}[1]{}  
%\newcommand{\ddp}[1]{}

\setlength\parindent{0pt}
\unframedsolutions
\SolutionEmphasis{\color{red}}
\CorrectChoiceEmphasis{\color{red}}
\renewcommand{\choicelabel}{(\alph{choice})}
\newcommand{\blank}[0]{\underline{\hspace{3cm}}}
\pointformat{\bfseries[\thepoints]}
\pointpoints{pt}{pts}
\pointsinrightmargin

\begin{document}
	
	
	\title{\textbf{Final Exam Study Guide} \\ \vspace{2 mm} {\large ECON 380} \\ \large{UNC Chapel Hill}}
	\date{}
	\maketitle
	
Stuff to know for the final exam:

\section*{Part I - Labor Force Accounting, Unemployment, Labor Supply \& Demand}

\begin{enumerate}
	\item See exam 1 and related study guide
\end{enumerate}

\section*{Part II}

\subsection*{Competitive Markets - Human Capital and Development}

\begin{enumerate}
	\item See exam 2 and related study guide
\end{enumerate}

\section*{Part III}

\subsection*{Inequality - Discrimination}

\begin{enumerate}
	\item See exam 3 and related study guide
\end{enumerate}

\section*{Part IV} 

\subsection*{Labor Mobility}

\begin{enumerate}
	\item Migration decision: How individual decides whether or not to migrate. Figuring out whether or not they should given home and away wages, migration costs, and discount rate
	\item What are ``push'' factors? ``Pull'' factors? How do they affect the decision to migrate
	\item How do changes in $C$ affect the migration decision
	\item Gist of Angelucci (2015) and HW 7, question 3
	\item How families decide to migrate. How to solve for this AND show it on graph we drew in class (where we had $\Delta PV^H$ on the x-axis and $\Delta PV^W$ on y-axis)
	\item What is a tied-stayer? Tied-mover? How to determine this from the graph we drew in class
	\item Know the gist of Tunali (2000), especially in regards to how returns to migrants vary and why (HW 8, question 3).
\end{enumerate}

\subsection*{Assimilation}

\begin{enumerate}
	\item Why might the wages of natives and foreign workers be different?
	\item What factors affect the (potential) initial earnings deficiency of migrants and the steepness of their age-earnings profile?
	\item What is the issue with using cross-sectional studies to estimate immigrant earnings assimilation?
	\item How to determine if estimate of economic assimilation is biased up or down. Also know why they would be biased in either direction
	\item HW 8, question 1
\end{enumerate}

\subsection*{Self-Selection}

\begin{enumerate}
	\item Implications of Roy Model in terms of the direction of immigrant self-selection if (i) the rate of return to skills is greater in the host country and (ii) the rate of return to skills is greater in the home country 
	\item Know the graphs for each of the two cases above
	\item What is the effect of (i) a change home country ``base level'' earnings and (ii) a change in host country ``base level'' earnings in regards to (a) the direction of self-selection and (b) the magnitude of migration
	\item What is the effect of a change in migration costs (assuming they are constant across the skills distribution) in regards to (i) the direction of self-selection and (ii) the magnitude of migration
	\item What will change the direction of self-selection? How would this be reflected on a graph?
\end{enumerate}

\subsection*{Immigration Benefits}

\begin{enumerate}
	\item Short-run graph in the case of perfect substitutes 
	\item How to calculate native worker and native firm surplus before and after immigration
	\item How to calculate immigration surplus
\end{enumerate}


\end{document}