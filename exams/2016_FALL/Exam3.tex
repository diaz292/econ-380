\documentclass[addpoints,11pt]{exam}

\usepackage{alltt}
\usepackage[margin=1in]{geometry}   % set up margins
\usepackage[T1]{fontenc}
\usepackage[usenames,dvipsnames]{xcolor}
\usepackage{enumerate}              % fancy enumerate
\usepackage{amsmath}                % used for \eqref{} in this document
\usepackage{amsthm}
\theoremstyle{definition}
\newtheorem{exmp}{Example}[section]
\usepackage{verbatim}               % useful for \begin{comment} and \end{comment}
\usepackage{eurosym}                % used for euro symbol
\usepackage{caption} 
\usepackage{graphicx}
\graphicspath{{Figures/}}
\usepackage{subcaption}
\usepackage{color}
\usepackage{float}
\usepackage{amssymb}
\usepackage{MnSymbol,wasysym}
\usepackage[colorlinks=true]{hyperref}
\hypersetup{colorlinks=true, citecolor=ForestGreen, linkcolor=BlueViolet, urlcolor=Magenta}

\usepackage{array}
\newcolumntype{H}{@{}>{\lrbox0}l<{\endlrbox}}


%Solutions or nah
%\printanswers
%\newcommand{\dd}[1]{{\textbf{\textcolor{red}{#1}}}}
%\newcommand{\ddp}[1]{\par {\textcolor{ForestGreen}{#1}}}

\newcommand{\dd}[1]{}  
\newcommand{\ddp}[1]{}

\setlength\parindent{0pt}
\unframedsolutions
\SolutionEmphasis{\color{red}}
\CorrectChoiceEmphasis{\color{red}}
\renewcommand{\choicelabel}{(\alph{choice})}
\newcommand{\blank}[0]{\underline{\hspace{3cm}}}
\pointformat{\bfseries[\thepoints]}
\pointpoints{pt}{pts}
\pointsinrightmargin

\begin{document}


\title{\textbf{Exam 3 \dd{\\Solutions}} \\ \vspace{2 mm} {\large ECON 380} \\ \large{Fall 2016} \\ \large{UNC Chapel Hill}}
\date{}
\maketitle

\makebox[\textwidth]{Name:\enspace\hrulefill}
\\

\makebox[\textwidth]{ONYEN:\enspace\hrulefill}
\\

\makebox[\textwidth]{Honor Code Signature:\enspace\hrulefill}
\\

\begin{center}
	\fbox{\fbox{\parbox{6in}{\centering
				\begin{itemize}
					\item For partial credit, show all of your work on the following pages, and justify your answers where needed.
					\item Assume firms can hire a non-integer number of workers (e.g., 4.25 workers) and can produce/sell a non-integer amount of output (e.g., 2.59 units)
					\item Round final answers to the nearest hundredth.
					\item Points available: 50
					\item Write legibly, write legibly, write legibly!
					\item Good luck! \smiley{}
				\end{itemize}
			}}}
\end{center}

\newpage

\subsection*{Wage Inequality}

\begin{questions}
	\question[5] Describe the trend in US wage inequality since the 1970s. Has this trend been driven by changes in the bottom, middle, or upper tail of the earnings distribution?
\begin{solution}[1.5in]
US wage inequality has increased \underline{substantially} in the last 30-40 years. Most of this trend has been driven by a widening gap in the upper tail of the earnings distribution.	
	
\end{solution}

\question Consider a simple economy that contains 1,000 individuals. 800 citizens report an annual after-tax income of \$40,000 (the ``low-income'' group), while the other 200 report an annual after-tax income of \$300,000 (the ``high-income'' group).

\begin{parts}
\part[5] Draw the Lorenz curve for this economy in the figure below.
\begin{solution}[.5in]
	
``Low-income'' group makes up 80\% of households. ``High-income'' group makes up 20\% of households.

Total income earned by ``low-income'' group: \$40,000 $\times$ 800 = \$32,000,000
\\
Total income earned by ``high-income'' group: \$300,000 $\times$ 200 = \$60,000,000
\\
Total income earned by both groups: \$92,000,000
\\
Share of income earned by ``low-income'' group: \$32,000,000/\$92,000,000 = 34.78\%
\\
Share of income earned by ``high-income'' group: \$60,000,000/\$92,000,000 = 65.22\%
\\
Lorenz curve is in red below.
\end{solution}
\begin{figure}[H]
	\centering
	\includegraphics[scale=.5]{exam3_1}
\end{figure}


%\begin{figure}[H]
%	\centering
%	\includegraphics[scale=.3]{exam3_1a}
%\end{figure}

\uplevel{Suppose that, under intense political pressure from lobbyists, the government lowers the tax rate for those individuals in the ``high-income'' group. The tax rate for low-income earners remains the same.}

\part[4] Will the post-tax Gini coefficient in this economy increase, decrease, or remain the same? What does this indicate about inequality in this country (i.e., will it fall, rise, or stay the same)?
\begin{solution}
Lowering taxes for the high-income group will raise their after-tax income. In turn, their share of total income will rise, while that of the low-income group will fall. This will shift the Lorenz curve outwards (away from the perfect equality Lorenz curve), which will increase the post-tax Gini coefficient. A higher Gini coefficient indicates greater inequality.
\end{solution}
\end{parts}

\end{questions}

\newpage

\subsection*{Taste-Based Discrimination}

\begin{questions}
	\question Bunce's Beans production function is given by 
	\[q = 2\sqrt{E_b + E_w}\]
	where $E_b$ and $E_w$ refer to black and white workers employed by the firm, respectively.  Suppose the wage rates for whites and blacks are $w_w = \$15$ and $w_b = \$10$. The price of each unit of output is \$25.
	Finally, the marginal product of labor is given by 
	\[MP_E = \frac{1}{(E_b + E_w)^{1/2}}\]
	
	
\begin{parts}
	\part[3] Suppose the firm has some distaste for hiring black workers and has a discrimination coefficient of 0.4. What proportion of the firm's labor will come from white workers? Make sure to state \underline{why}.
	\begin{solution}[1in]
		Utility-adjusted black wage: $w_b' = \$10\times1.4 = \$14$.
		The utility-adjusted black wage is still lower than the white wage, so the firm will only hire black workers. 0\% of its labor will be white workers. 
	\end{solution}
	\part[5] How many workers will this firm optimally hire? What will its profits be?
	\begin{solution}[2in]
		$E_b^*$ where $VMP_E = w_b'$:
		\[\$25\times\frac{1}{E_b^{1/2}} = \$14 \Rightarrow E_b^* = \Bigg(\frac{25}{14}\Bigg)^2 = 3.19\]
		Quantity produced: $q^* = 2\sqrt{3.19} = 3.57$.
		\\
		
		Profit:
		\[\Pi = p\cdot q^* + w_b \cdot E_b^* = \$25 \cdot 3.57 - \$10 \cdot 3.19 = \$57.35\]
	\end{solution}
	\part[4] Compare the profit of Bunce's Beans to that of a non-discriminatory firm (i.e., is it smaller, larger, or the same). Explain why the profits are different or the same.
\begin{solution}[1.5in]
Bunce's Beans will make a smaller profit than a non-discriminatory firm. Even though Bunce's Beans and a non-discriminatory firm both hire only black workers, Bunce will hire a sub-optimal amount of labor due to its perceived higher cost of hiring black workers which will reduce its actual profit below that of a non-discriminating firm.
\end{solution} 
\end{parts}
\end{questions}

\newpage

\subsection*{Statistical Discrimination}

\begin{questions}
	
\question[5] In 3-4 sentences, describe how statistical discrimination can lead to \underline{long-run} wage gaps.

	\begin{solution}[2.5in]
	Under statistical discrimination, long-run wage gaps emerge because two groups start with different average ``test scores'' (e.g., schooling levels). An individual in the more ``advantaged'' group will receive a higher wage for any given ``test score'' than an equivalent individual in the ``disadvantaged'' group because employers use a worker's actual test score in conjunction with the group average to determine wages. In turn, this leads to the more advantaged group having a greater return to, for example, education, so they obtain more schooling than the disadvantaged group in the next generation, leading to higher wages which continues the cyclical pattern and leads to a persistent wage gap in the long-run. 
\end{solution}

\question Suppose that firms statistically discriminate based on sex. The average ``test score'' for males and females is the same, but test scores are ``noiser'' for females, i.e., the test score is a bad predictor of female productivity.

\begin{parts}
\part[2] Will female workers with lower-than-average test scores receive higher, lower, or the same wages as a ``low-skill'' male worker with the same test score? Briefly explain why.
	\begin{solution}[2in]
		Because test scores are more ``noisy'' for females, firms will place more weight on the average score for females rather than on the actual observed worker score. On the other hand, male scores are better predictors of productivity, so the firm will place more relatively more weight on actual observed scores than on the average male score. As a result, ``low-skill'' females will earn a higher wage than similar ``low-skill'' males as their expected productivity is ``brought up'' by the average female score.
\end{solution}
\part[2] Will female workers with higher-than-average test scores receive higher, lower, or the same wages as a ``high-skill'' male worker with the same test score? Briefly explain why.
	\begin{solution}[2in]
	Opposite case to above. Above-average females will be ``brought down'' by the average female test score and they will earn a lower wage than above-average male workers with equivalent scores.
\end{solution}
\end{parts}

\end{questions}

\subsection*{Measuring Discrimination}

\begin{questions}

\question Suppose the wages of males and females are determined as follows

\[w_m = 9 + 0.5\cdot S_m\]
\[w_f = 8 + 0.4\cdot S_f\]

where the $m$ and $f$ subscripts refer to males and females, respectively, and  $S$ refers to the number of years schooling a worker obtains. Assume schooling is the only relevant skill to a worker's productivity. Finally, suppose the average years of schooling for males is 10 years and for females it is 8 years.

\begin{parts}
	\part[2] What is the average male wage? The average female wage? 
	\begin{solution}[1in]
		\[\overline{w}_m = 9 + 0.5\cdot 10 = \$14\]		
		\[\overline{w}_f = 8 + 0.4\cdot 8 = \$11.20\]
	\end{solution}
	\part[2] What is the raw wage differential between the two groups?
	\begin{solution}[1in]
		\[\Delta \overline{w} = \$14 - \$11.20 = \$2.80\]
	\end{solution}
	\part[2] How much would an average female worker earn if ``treated as a male?''
	\begin{solution}[1in]
		\[w_f^* = \alpha_m + \beta_m \cdot \overline{S}_f = 9 + 0.5\cdot 8 = \$13\]
	\end{solution}
	\part[2] How much of the wage differential can be attributed to differences in schooling?
	\begin{solution}[1in]
		Schooling difference component:
		\[\overline{w}_m - w_f^* = \$14 - \$13 = \$1\]
	\end{solution}
	\part[2] How much of the wage differential can be attributed to discrimination?
	\begin{solution}[1in]
		Discrimination component:
		\[w_f^* - \overline{w}_f = \$13 - \$11.20 = \$1.80\]
	\end{solution}
\end{parts}
\end{questions}

\newpage

\subsection*{Discrimination Evidence}

\begin{questions}
	
\question[5] In recent years, how has the role of explicit labor market discrimination in driving different racial socio-economic outcomes likely changed? Will policies targeting labor market discrimination or policies targeting school achievement gaps be more effective in closing the racial wage gap and other differences in socio-economic outcomes? 

\begin{solution}[2in]
	In recent years, the role of labor market discrimination in driving racial wage gaps has diminished. A large portion of differences in socio-economic outcomes between racial groups is driven by pre-market factors (e.g., Neal \& Johnson (1996) showed that the black-white wage gap significantly shrinks once they accounted for differences in skills). As such, policies that target racial achievement gaps, which are often large and persist across grades, will be more effective at closing the racial gap across a variety of socio-economic outcomes as opposed to policies that target explicit labor market discrimination (though they certainly help as well). 
\end{solution}
\end{questions}

\end{document}