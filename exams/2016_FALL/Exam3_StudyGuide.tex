\documentclass[addpoints,11pt]{exam}

\usepackage{alltt}
\usepackage[margin=1in]{geometry}   % set up margins
\usepackage[T1]{fontenc}
\usepackage[usenames,dvipsnames]{xcolor}
\usepackage{enumerate}              % fancy enumerate
\usepackage{amsmath}                % used for \eqref{} in this document
\usepackage{amsthm}
\theoremstyle{definition}
\newtheorem{exmp}{Example}[section]
\usepackage{verbatim}               % useful for \begin{comment} and \end{comment}
\usepackage{eurosym}                % used for euro symbol
\usepackage{caption} 
\usepackage{graphicx}
\usepackage{threeparttable}
\graphicspath{{Figures/}}
\usepackage{subcaption}
\usepackage{booktabs}
\usepackage{color}
\usepackage{float}
\usepackage{amssymb}
\usepackage{sgamevar}
\usepackage{sgame}
\usepackage[colorlinks=true]{hyperref}
\hypersetup{colorlinks=true, citecolor=ForestGreen, linkcolor=BlueViolet, urlcolor=Magenta}

\usepackage{array}
\newcolumntype{H}{@{}>{\lrbox0}l<{\endlrbox}}


%Solutions or nah (blank next two lines out for no solutions, unblank #3)
\printanswers
\newcommand{\dd}[1]{{\textbf{\textcolor{red}{#1}}}}
\newcommand{\ddp}[1]{\par {\textcolor{ForestGreen}{#1}}}

%\newcommand{\dd}[1]{}  
%\newcommand{\ddp}[1]{}

\setlength\parindent{0pt}
\unframedsolutions
\SolutionEmphasis{\color{red}}
\CorrectChoiceEmphasis{\color{red}}
\renewcommand{\choicelabel}{(\alph{choice})}
\newcommand{\blank}[0]{\underline{\hspace{3cm}}}
\pointformat{\bfseries[\thepoints]}
\pointpoints{pt}{pts}
\pointsinrightmargin

\begin{document}
	
	
	\title{\textbf{Exam 3 Study Guide} \\ \vspace{2 mm} {\large ECON 380} \\ \large{UNC Chapel Hill}}
	\date{}
	\maketitle
	
Stuff to know for exam 3:

\subsection*{Inequality}

\begin{enumerate}
	\item What has been the trend in US wage inequality since $\sim$1970 (increasing, decreasing, neither)?
	\item Has the trend in (1) been mostly driven by the changes in the bottom, middle, or top of the wage distribution?
	\item What the 90-10, 90-50, 50-10, etc. wage gaps measure
	\item How to draw a Lorenz curve given data on household shares and income (e.g., class example or homework problem)
	\item How to compute Gini coefficient
	\item Why might wage inequality be bad? In what ways might it be good?
	\item Basic gist of possible explanations for recent trend in wage inequality: (i) tax changes, (ii) supply shifts in skilled labor, (iii) skill-based technology change, and (iv) composition effects
	\item What are the weaknesses of each of the above explanations (e.g., composition effects don't explain the rise in upper tail inequality)?
\end{enumerate}

\subsection*{Superstars \& Intergenerational Inequality}

\begin{enumerate}
	\item What is a ``superstar'' in the context of labor economics?
	\item What characteristics do occupations with superstars generally share?
	\item Can rising superstar earnings explain much of the wage growth of the top .1\%?
	\item What is the basic mechanism driving why inequality might persist across generations?
	\item What is regression towards the mean? Why might we observe this?
	\item How to compute and interpret the intergenerational correlation coefficient
	 
\end{enumerate}


\subsection*{Labor Market Discrimination}

\begin{enumerate}
	\item What is labor market discrimination?
	\item What are pre-labor market differences?
	\item What has been the trend in the black-white wage since the 1970s (shrinking gap, rising, neither)?
	\item What has been the trend in the male-female wage since the 1970s (shrinking gap, rising, neither)? 
\end{enumerate}

\subsection*{Taste-Based Discrimination}

\begin{enumerate}
	\item What is taste-based discrimination?
	\item Know how to do HW6, \#1 
	\item Know the graphs from class related to employer-based discrimination 
	\item Who are the ``winners'' and ``losers'' from the presence of discriminatory firms?
	\item What is the relationship between the profits of discriminatory firms vs ND firms? 
	\item What is the long-run outcome predicted by employer-discrimination?
	\item Does employer-based discrimination explain the existence of wage gaps? 
	\item Does employee-based discrimination explain the existence of wage gaps?
	\item Does customer-based discrimination explain the existence of wage gaps?
\end{enumerate}

\subsection*{Statistical Discrimination}

\begin{enumerate}
	\item What is statistical discrimination?
	\item Know graphs from class related to statistical discrimination
	\item Implications of statistical discrimination if average test scores are the same, but the the signal is more ``noisy'' for one group? Who benefits? Who is hurt?
	\item Can a ``noisy'' signal (even if test scores, on average, are equal) lead to wage gaps? Will average wages be different?
	\item Implications of statistical discrimination if the average test score of one group is lower than the other (assuming the signal is just as ``noisy'' for each group)? 
	\item How do different average test scores lead to wage gaps? Does this hold in the long-run? Why?
\end{enumerate}

\subsection*{Measuring Discrimination}

\begin{enumerate}
	\item How to interpret linear wage schooling locus
	\item How different coefficients are evidence of discrimination 
	\item Oaxaca-Blinder decomposition: pre-market component and discrimination component
	\item Know graph from class and interpretation: discrimination component is $w_m^* - \bar{w}_m$ (what average minority worker would receive if ``treated as white worker'' minus average minority worker wage); pre-market component is $\bar{w}_w - w_m^*$
\end{enumerate}

\subsection*{Discrimination Evidence}

\begin{enumerate}
	\item Know 2-3 of the potential issues that arise when estimating the typical earnings model testing for discrimination
	\item Neal \& Johnson (1996): What do the authors find drives a large portion of the black-white wage gap? Do they argue the wage gap is largely driven by discrimination or pre-labor market factors?
	\item Altonji \& Pierret (2001): Do the authors find evidence that firms statistically discriminate on the basis of race? 
	\item Bertrand \& Mullainathan (2004): Know basic set-up of the study. How did the callback rates for ``white sounding names'' compare to that of ``black sounding names?'' 
	\item In recent years, how has the role of explicit labor market discrimination in driving different socio-economic outcomes changed? Will policies targeting labor market discrimination or policies targeting school achievement gaps be more effective in closing wage gaps and other differences in socio-economic outcomes?
	\item What is likely the greatest mechanism driving the gender wage gap? How does the shape of the relative earnings curve between males and females over the life-cycle reflect this?
	\item What is occupational crowding? Does it necessarily play a large role in driving the gender wage gap?
\end{enumerate}


\end{document}