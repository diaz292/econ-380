\documentclass[11pt]{article}
\usepackage[table,xcdraw]{xcolor}
\usepackage[colorlinks=true]{hyperref}
\hypersetup{colorlinks,urlcolor=blue}
\usepackage{booktabs}
\usepackage{threeparttable}
\usepackage{caption}



\newcommand{\hl}{\begin{flushleft}
	\rule{\textwidth}{1}
\end{flushleft}
}


\newcommand{\n}{\vspace{12pt}}
\usepackage{amssymb} 
\usepackage{verbatim}
\usepackage{amsmath}
\usepackage{graphicx}
\usepackage{geometry}



\parindent 0in


\begin{document}


\begin{center} UNIVERSITY OF NORTH CAROLINA\\
DEPARTMENT OF ECONOMICS \\
\end{center}

\begin{center}\textbf{{\hspace{-.05in}}
\underline{ECON 380: LABOR ECONOMICS}}
\\ Fall 2016
\\ MWF 1:25PM - 2:15PM 
\\ New West 219
\end{center}


\textbf{Instructor:} David A. D\'iaz \hspace{4.5cm} \textbf{Email:} \url{diazda@live.unc.edu}
\textbf{Office:} Phillips Annex 103A  \\
\textbf{Office Hours:} Thursday \& Friday, 11:00AM - 12:00PM\\
\textbf{Course Website:} \url{sakai.unc.edu}\\
The Sakai web site will contain the official course gradebook, announcements, and other supplementary materials.\\

\textbf{Prerequisites:} ECON 101 and ECON 310/410
\\

\textbf{Course Description:} This course applies concepts from microeconomic theory to the study of labor markets. We will apply the ideas you learned in ECON 101 and ECON 310/410 to various topics in labor economics as a means of understanding labor market behavior.
\\

\textbf{Textbook:} George Borjas, Labor Economics, $7^{th}$ Edition.

\subsection*{Grading and Course Components}
\begin{itemize}
	\item Homework: 24\%
	\item Exam 1: 	12\%
	\item Exam 2: 12\%
	\item Exam 3: 12\%
	\item	Final Exam: 40\% 
\end{itemize}

The grading scale is as follows:
\begin{center}
	\begin{tabular}{ p{3.5cm} p{3.5cm} }
		A: 93 -- 100 &  C+: 77 -- 79.99\\
		A--: 90 -- 92.99 & C: 70 -- 76.99\\
		B+: 87 -- 89.99 & C--: 67 -- 69.99\\
		B: 83 -- 86.99 & D+: 64 -- 66.99\\
		B--: 80 -- 82.99 & D: 60 -- 63.99\\
		& F: $<$ 60
		
		
	\end{tabular}
\end{center}

\textbf{Homework:} There will be eight problem sets assigned during the course of the semester. You are free to work with your peers on the homework, but you each must turn in an individual assignment reflecting your own work. Homework is due by the end of class on the assigned date. Due dates are firm and no late homework will be accepted without prior approval. \\

\textbf{Exams:} Three in-class exams will be given during the course in addition to the final exam. Each exam will cover material presented in class, readings scheduled outside of class, and homework assignments. The exam format will be provided a few days prior to a given exam. You should bring a \textbf{non-graphing} calculator with you to each exam (if in doubt about yours, ask!). \\

\textbf{Exam Dates:}\vspace{-.25cm}
\begin{tabbing} 
	\hspace{4cm}\= \hspace{4cm} \kill 
	{\hspace{1cm} Exam 1} \>  September 23 \\
	{\hspace{1cm} Exam 2} \>  October 19 \\
	{\hspace{1cm} Exam 3} \>  November 14 \\
	{\hspace{1cm} Final Exam} \>  December 12 \\
\end{tabbing}\vspace{-.3cm}


\subsection*{Course Policies}

\textbf{Attendance \& Participation:} Regular attendance and participation is strongly recommended, but not required. You are responsible for any notes or class announcements that you may have missed. You should attempt to get these notes/announcements from one of your peers before seeing me. It is expected that you will respect your peers and the instructor with appropriate behavior while in class and that you will arrive to class on time. You should refrain from browsing the web, texting, etc. during class time. \\


\textbf{Missed Exams:} There are no make-up midterm examinations. If you must miss a midterm exam, you may be permitted to transfer the missed credit to the final examination. To qualify for a transfer of credit, you must contact me before the start of the missed midterm examination and provide me with an acceptable explanation. If the reason for your absence could not be foreseen, please make the request as soon as possible thereafter. All requests should be in writing and you may be asked to provide support with suitable documentation.  \\

If you must miss the final exam for any excused reason, it is your responsibility to alert your instructor and Dean as soon as possible to schedule a make-up exam. Please see the \href{http://www.catalog.unc.edu/policies-procedures/attendance-grading-examination/#Final Examinations}{University policy} regarding final examinations.\\

\textbf{Academic Integrity:} All students are expected to adhere to the \href{http://instrument.unc.edu}{UNC Honor Code}. 

\newpage

\begin{table}
	\centering
	\caption*{\textbf{Course Outline$^*$}}
	\resizebox{\textwidth}{!}{%
		\begin{tabular}{@{}llll@{}}
			\toprule
			Date & Topic & Readings & Notes \\ \midrule
			8/24 & Introduction to Labor Economics & Syllabus; 1.1-1.3 &  \\
			8/26 & The U.S. Labor Force & 2.1-2.2 &  \\
			8/29 & Unemployment & 12.1-12.2 &  \\
			8/31 & Neoclassical Model of Labor Supply: Preferences & 2.3 &  \\
			9/2 & Neoclassical Model of Labor Supply: Constraints & 2.4 & Homework 1 Due \\
			9/7 & Labor Supply: Number of Hours & 2.5 &  \\
			9/9 & Labor Supply: The Participation Decision & 2.6 &  \\
			9/12 & Application: Anti-Poverty Policy & 2.10-2.11 &  \\
			9/14 & Labor Demand in the Short Run & 3.1-3.2 &  \\
			9/16 & Labor Demand in the Long Run & 3.3-3.4 &  \\
			9/19 & Application: The Minimum Wage & 3.10 & Homework 2 Due \\
			9/21 & Labor Market Equilibrium: Competitive Markets & 4.1-4.2 &  \\
			9/23 & \textbf{Exam 1} &  &  \\
			9/26 & Application: Immigration Impacts & 4.5 &  \\
			9/28 & Noncompetitive Labor Markets & 4.8 &  \\
			9/30 & Application: Labor Markets and Development & Sakai Readings &  \\
			10/3 & Human Capital: Introduction & 6.1 & Homework 3 Due \\
			10/5 & Human Capital: The Schooling Model & 6.2-6.3 &  \\
			10/7 & Human Capital: Returns to Schooling & 6.4-6.5 &  \\
			10/10 & Human Capital: Signaling Model & 6.9 &  \\
			10/12 & Application: Human Capital and Development & Sakai Readings &  \\
			10/14 & Application: Human Capital and Development & Sakai Readings & Homework 4 Due \\
			10/17 & Income Inequality: Introduction & 7.1-7.3 &  \\
			10/19 & \textbf{Exam 2} &  &  \\
			10/24 & Application: Why is Inequality Rising? & 7.4 &  \\
			10/26 & Wage Dispersion and Intergenerational Inequality & 7.5-7.6 &  \\
			10/28 & Labor Market Discrimination: Introduction & 9.1 & Homework 5 Due \\
			10/31 & Employer Discrimination & 9.2-9.3 &  \\
			11/2 & Employeee \& Customer Discrimination & 9.4-9.5 &  \\
			11/4 & Statistical Discrimination & 9.6 &  \\
			11/7 & Measuring Discrimination & 9.8 &  \\
			11/9 & Evidence on Discrimination & 9.7, Sakai Readings & Homework 6 Due \\
			11/11 & Application: Black-White \& Male-Female Wage Gaps & 9.9, 9.11 &   \\
			11/14 & \textbf{Exam 3} & \textbf{} & \textbf{} \\
			11/16 & Labor Mobility: Introduction & 8.1-8.2 &  \\
			11/18 & Family Migration & 8.3 &  \\
			11/21 & Immigration in the U.S. & 8.4-8.5 &  \\
			11/28 & Application: Migration and Family Left-Behind & Sakai Readings & Homework 7 Due \\
			11/30 & The Decision to Migrate & 8.6 &  \\
			12/2 & The Economic Benefits from Immigration & 8.7 &  \\
			12/5 & Application: Human Capital Externalities & 8.8 &  \\
			12/7 & Application: Intergenerational Mobility of Immigrants & 8.9 & Homework 8 Due \\
			12/12 & \textbf{Final Exam, 12PM-3PM} & \textbf{} & \textbf{} \\ \bottomrule
			\multicolumn{4}{l}{\small $^*$Note: This outline is tentative and subject to change.} \\
		\end{tabular}%
	}
\end{table}

\end{document}
