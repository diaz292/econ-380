\documentclass[11pt]{article}
\usepackage[table,xcdraw]{xcolor}
\usepackage[colorlinks=true]{hyperref}
\hypersetup{colorlinks,urlcolor=blue}
\usepackage{booktabs}
\usepackage{threeparttable}
\usepackage{caption}



\newcommand{\hl}{\begin{flushleft}
	\rule{\textwidth}{1}
\end{flushleft}
}


\newcommand{\n}{\vspace{12pt}}
\usepackage{amssymb} 
\usepackage{verbatim}
\usepackage{amsmath}
\usepackage{graphicx}
\usepackage{geometry}



\parindent 0in


\begin{document}


\begin{center} UNIVERSITY OF NORTH CAROLINA\\
DEPARTMENT OF ECONOMICS \\
\end{center}

\begin{center}\textbf{{\hspace{-.05in}}
\underline{ECON 380: LABOR ECONOMICS}}
\\ Fall 2017
\\ MWF 1:25PM - 2:15PM
\\ Gardner Hall 308
\end{center}


\textbf{Instructor:} David A. D\'iaz \hspace{4.5cm} \textbf{Email:} \url{diazda@live.unc.edu}
\textbf{Office:} Phillips \underline{\textit{Annex}} 103A  \\
\textbf{Office Hours:} Monday \& Wednesday, 2:30PM - 3:30PM\\
\textbf{Course Website:} \href{https://sakai.unc.edu/portal/site/8456c8f3-2ffa-4262-84e9-5993c613ddb9
}{https://sakai.unc.edu}\\
The Sakai web site will contain the official course gradebook, announcements, and other supplementary materials.\\

\textbf{Prerequisites:} ECON 101 and ECON 310/410
\\

\textbf{Course Description:} This course applies concepts from microeconomic theory to the study of labor markets. We will apply the ideas you learned in ECON 101 and ECON 310/410 to various topics in labor economics as a means of understanding labor market behavior.
\\

\textbf{Recommended Textk:} George Borjas, Labor Economics, $8^{th}$ Edition.

\subsection*{Grading and Course Components}
\begin{itemize}
	\item Homework: 25\%
	\item Midterm 1: 20\%
	\item Midterm 2: 20\%
	\item Final Exam: 35\% 
\end{itemize}

The grading scale is as follows:
\begin{center}
	\begin{tabular}{ p{3.5cm} p{3.5cm} }
		A : 93 -- 100 &  C+ : 77 -- 79.99\\
		A-- : 90 -- 92.99 & C : 73 -- 76.99\\
		B+ : 87 -- 89.99 & C-- : 70 -- 72.99\\
		B : 83 -- 86.99 & D+ : 65 -- 69.99\\
		B-- : 80 -- 82.99 & D : 60 -- 64.99\\
		& F : $<$ 60
		
		
	\end{tabular}
\end{center}

\textbf{Homework:} There will be six or seven problem sets assigned during the course of the semester. You are free to work with your peers on the homework, but you each must turn in an individual assignment reflecting your own work. Homework is due by the end of class on the assigned date. Due dates are firm and no late homework will be accepted without prior approval. \\

\textbf{Exams:} Two in-class exams will be given during the course in addition to the final exam. Each exam will cover material presented in class, readings scheduled outside of class, and homework assignments. The exam format will be provided a few days prior to a given exam. You should bring a \textbf{non-graphing} calculator with you to each exam (if in doubt about yours, ask!). \\

Dates for the in-class midterms will be determined at least a week ahead of time. The final exam is scheduled for 12PM on Monday December 11. 

\subsection*{Course Policies}

\textbf{Attendance \& Participation:} Regular attendance and participation is strongly recommended. You are responsible for any notes or class announcements that you may have missed. You should attempt to get these notes/announcements from one of your peers before seeing me. It is expected that you will respect your peers and the instructor with appropriate behavior while in class and that you will arrive to class on time. You should refrain from browsing the web, texting, etc. during class time. \\


\textbf{Missed Exams:} There are no make-up midterm examinations. If you must miss a midterm exam, you may be permitted to transfer the missed credit to the final examination. To qualify for a transfer of credit, you must contact me before the start of the missed midterm examination and provide me with an acceptable explanation. If the reason for your absence could not be foreseen, please make the request as soon as possible thereafter. All requests should be in writing and you may be asked to provide support with suitable documentation.  \\

If you must miss the final exam for any excused reason, it is your responsibility to alert your instructor and Dean as soon as possible to schedule a make-up exam. Please see the \href{http://www.catalog.unc.edu/policies-procedures/attendance-grading-examination/#Final Examinations}{University policy} regarding final examinations.\\

\textbf{Academic Integrity:} All students are expected to adhere to the \href{http://instrument.unc.edu}{UNC Honor Code}. 

\newpage

\subsection*{Tentative Course Outline}

\begin{itemize}
	\item Introduction to Labor Economics
	\begin{itemize}
		\item The U.S Labor Force
		\item Unemployment
	\end{itemize}
	\item Labor Supply
	\begin{itemize}
		\item Worker Preferences
		\item Worker Constraints
		\item Optimal Work Decisions
		\item Application: Anti-Poverty Policy
	\end{itemize}
	\item Labor Demand
	\begin{itemize}
		\item Labor Demand in the Short-Run
		\item Labor Demand in the Long-Run
		\item Application: The Minimum Wage
	\end{itemize}
	\item Labor Market Equilibrium
	\begin{itemize}
		\item Competitive Markets
		\item Noncompetitive Markets
		\item Application: Immigration Impacts
		\item Application: Labor Markets and Development
	\end{itemize}
\end{itemize}
\textbf{Midterm 1}: Introduction to Labor Economics - Labor Market Equilibrium
\begin{itemize}
	\item Human Capital
	\begin{itemize}
		\item The Schooling Model
		\item The Signaling Model
		\item Application: Empirical Studies
		\item Application: Human Capital and Development
	\end{itemize}
	\item Income Inequality
	\begin{itemize}
		\item Measuring Inequality
		\item Why is Inequality Rising?
		\item Wage Dispersion: Superstars
	\end{itemize}
	\item Discrimination
	\begin{itemize}
		\item Taste-Based Discrimination
		\item Statistical Discrimination
		\item Measuring Discrimination
		\item Application: Empirical Studies
	\end{itemize}
\end{itemize}
\textbf{Midterm 2}: Human Capital - Discrimination
\begin{itemize}
\item Selected Topics (Time permitting)
\begin{itemize}
	\item Human Capital: The Supply Side
	\item Social Interactions \& Network Effects
\end{itemize}
\end{itemize}
\textbf{Final Exam}: Monday December 11, 12PM-3PM


\end{document}
