\documentclass[addpoints,11pt]{exam}

\usepackage{alltt}
\usepackage[margin=1in]{geometry}   % set up margins
\usepackage[T1]{fontenc}
\usepackage[usenames,dvipsnames]{xcolor}
\usepackage{enumerate}              % fancy enumerate
\usepackage{amsmath}                % used for \eqref{} in this document
\usepackage{amsthm}
\theoremstyle{definition}
\newtheorem{exmp}{Example}[section]
\usepackage{verbatim}               % useful for \begin{comment} and \end{comment}
\usepackage{eurosym}                % used for euro symbol
\usepackage{caption} 
\usepackage{graphicx}
\usepackage{threeparttable}
\graphicspath{{Figures/}}
\usepackage{subcaption}
\usepackage{booktabs}
\usepackage{color}
\usepackage{float}
\usepackage{amssymb}
\usepackage{sgamevar}
\usepackage{sgame}
\usepackage[colorlinks=true]{hyperref}
\hypersetup{colorlinks=true, citecolor=ForestGreen, linkcolor=BlueViolet, urlcolor=Magenta}

\usepackage{array}
\newcolumntype{H}{@{}>{\lrbox0}l<{\endlrbox}}


%Solutions or nah (blank next two lines out for no solutions, unblank #3)
%\printanswers
%\newcommand{\dd}[1]{{\textbf{\textcolor{red}{#1}}}}
%\newcommand{\ddp}[1]{\par {\textcolor{ForestGreen}{#1}}}

\newcommand{\dd}[1]{}  
\newcommand{\ddp}[1]{}

\setlength\parindent{0pt}
\unframedsolutions
\SolutionEmphasis{\color{red}}
\CorrectChoiceEmphasis{\color{red}}
\renewcommand{\choicelabel}{(\alph{choice})}
\newcommand{\blank}[0]{\underline{\hspace{3cm}}}
\pointformat{\bfseries[\thepoints]}
\pointpoints{pt}{pts}
\pointsinrightmargin

\begin{document}


\title{\textbf{Homework 7 \dd{\\Solutions}} \\ \vspace{2 mm} {\large ECON 380} \\ \large{UNC Chapel Hill}}
\date{}
\maketitle

\makebox[\textwidth]{Name:\enspace\hrulefill}
\\

\makebox[\textwidth]{ONYEN:\enspace\hrulefill}
\\

\begin{center}
	\fbox{\fbox{\parbox{5.5in}{\centering
				This homework is due on \textbf{November 30} by \textbf{2:15PM}. You must turn in your work on a printed copy of this document in order for it to be graded. Your assignment must be stapled and in the correct order. Non-stapled assignments will automatically receive a 10 point deduction. There are a total of 50 available points.}}}
\end{center}
	

\begin{questions}
	

\question A worker with an annual discount rate of 8 percent
currently resides in Nigeria and is deciding whether to remain there
or move to France. There are three work periods left in the life cycle. If
the worker remains in Nigeria, he will earn \$25,000 per year in each
of the three periods. If the worker moves to France, he will earn \$32,000 in
each of the three periods. 

\begin{parts}
	\part[4] What is the net present value of the worker's earning if he chooses to stay in Nigeria? Make sure to write out the entire equation.
\begin{solution}[1in]
	\[NPV^N = \$25,000 + \frac{\$25,000}{(1.08)} + \frac{\$25,000}{(1.08)^2} = \$69,582\]
\end{solution}
\ddp{(2) written equation, (2) answer}
	\part[4] What is the net present value of the worker's earning if he chooses to move to France? Make sure to write out the entire equation.
\begin{solution}[1in]
	\[NPV^F = \$32,000 + \frac{\$32,000}{(1.08)} + \frac{\$32,000}{(1.08)^2} = \$89,064\]
\end{solution}
	\part[4] What is the highest cost of migration that this worker is willing to incur and still make the move?
\begin{solution}[1in]
	Should move as long as 
	\[NPV^F - NPV^N > C\]
	\[\Rightarrow  \$19,482 > C\] 
\ddp{Full credit as long as it follows from (a) and (b)}
\end{solution}
\end{parts}

\question Sally just turned 18 years old and is choosing whether to migrate. If she migrates, she will earn a salary of \$50,000 paid at \underline{end of each year} of work until she retires at age 65  (i.e., at age 19, 20, $\dots$, 64, 65). If she chooses to stay home she will receive \$44,000 each year (also paid at the end of each work year). Her discount rate is r = 0.10. Note: You can do this by hand if you really want to, but Excel is your friend here. I will upload a document you can use as a template for doing this in Excel. You don't need to hand in the excel document, you can just write your answer.


\begin{parts}
	\part[5] What is Sally's net present value of earnings if she chooses to stay home?
\begin{solution}[.5in]
	$NPV^H$ = \$435,011.20 
\end{solution}
\ddp{(5) correct answer, (4) if they added \$50,000 (i.e., if they assumed there was a payment at age 18); minus 1 for each \$20,000 they are off otherwise}
	\part[5] What is Sally's net present value of earnings if she chooses to migrate?
\begin{solution}[.5in]
	 $NPV^F$ =  \$494,330.90  
\end{solution}
\ddp{(5) correct answer, (4) if they added \$44,000 (i.e., if they assumed there was a payment at age 18); minus 1 for each \$20,000 they are off otherwise}
	\part[3] What is the highest cost of migration that Sally is willing to incur and still make the move?
\begin{solution}[.5in]
	\[NPV^F - NPV^H > C\]
	\[\Rightarrow \$59,319.70 > C\]
\end{solution}
\ddp{Full credit as long as it follows from (a) and (b)}
\end{parts}

\question Suppose that the present value of lifetime earnings of workers who decide to stay at home varies with their skill level and is given by

\[w(s)^h = 20,000 + 500s\] 

where $s$ denotes their level of skill and $s\ge 0$. Additionally, the present value of lifetime earnings of a worker who decides to move is given by

\[w(s)^a = 30,000 + 600s\]

\begin{parts}
	\part[4] If the cost of migration is \$11,500, what is the lowest skill level at which a worker would choose to migrate?
\begin{solution}[1in]
Net gain from moving: 
\[\Delta w = w(s)^a - w(s)^h = (30,000 + 600s) - (20,000 + 500s) = 10,000 + 100s\] 

Worker chooses to migrate as long as $\Delta w > C$:

\[10,000 + 100 s > 11,500\]
\[\Rightarrow s > 15\]
\end{solution}
\ddp{(2) work, (2) answer}
	\part[4] Now, suppose that the net present value of migration costs also varies with skill and is given by
	\[C(s) = 19,000 - 500s\]
	What is the lowest skill level at which a worker would choose to migrate?
\begin{solution}[1in]
	Worker chooses to migrate as long as $\Delta w > C(s)$:
	
	\[10,000 + 100 s > 19,000 - 500s\]
	\[\Rightarrow s > 15\]
\end{solution}
\ddp{(2) work, (2) answer}
	\part[4] A cash transfer program is introduced in the country which reduces the cost of migration by relaxing financial constraints. As a result, the net present value of migration costs is now given by
	\[C(s) = 16,000 - 500s\]
	What is the lowest skill level at which a worker would choose to migrate?
\begin{solution}[1in]
	Worker chooses to migrate as long as $\Delta w > C(s)$:
	
	\[10,000 + 100 s > 16,000 - 500s\]
	\[\Rightarrow s > 10\]
\end{solution}	
\ddp{(2) work, (2) answer}
\end{parts}


\question (Borjas) Patrick and Rachel live in Seattle. Patrick's net present value of lifetime earnings in Seattle is \$125,000, while Rachel's is \$500,000. The one-time cost of moving to Atlanta is \$25,000 \underline{per person}. In Atlanta, Patrick's net present value of lifetime earnings would be \$155,000, while Rachel's would be \$510,000. 

\begin{parts}
	\part[4] If Patrick and Rachel choose where to live based on their joint well-being, will they move to Atlanta?
\begin{solution}[1in]
	\[\Delta PV^P = \$155,000 - \$125,000 - \$25,000 = \$5,000\]
	\[\Delta PV^R = \$510,000 - \$500,000 - \$25,000 = -\$15,000\]
	\[\Delta PV^P + \Delta PV^R = -\$10,000 < 0 \]
	Since the joint NPV of moving is negative, they should not move to Atlanta. 
\end{solution} 
\ddp{(2) work, (2) answer}
	\part[2] Is Patrick a tied mover or a tied stayer or neither? 
\begin{solution}[.5in]
	Patrick is a tied stayer (would move individually, but stays for joint well-being of the family)
\end{solution}
	\part[2] Is Rachel a tied mover or a tied stayer or neither? 
\begin{solution}[.5in]
	Rachel is neither 
\end{solution}
\end{parts}



\question[5] Read the introduction to Borjas (1985). Briefly summarize the findings in Borjas' paper and how they compare to previous studies of migrant wage assimilation. What is a reason Borjas gives as to why previous studies estimated strong assimilation rates?

\ddp{As usual, give full credit if they hit most of the main points. Make sure they answer the second part of the question. Take off 2 points if they don't.}

\begin{solution}
Borjas uses data from the 1970 and 1980 U.S. Census to study the earnings of within-immigrant cohorts and finds that, for most immigrant groups, the cohort analysis predicts slow rates of earnings growth. This is in contrast to previous cross-sectional studies of migrant earnings growth that found relatively high rates of wage growth. Additionally, the ``overtaking'' point when migrant earnings surpass that of natives takes place much later in the lifecycle than that predicted by previous studies (if overtaking happens at all). Borjas states that a likely driver of these results is the declining ``quality'' of migrant cohorts since 1950.
\end{solution}

\end{questions}




\end{document}