\documentclass[addpoints,11pt]{exam}

\usepackage{alltt}
\usepackage[margin=1in]{geometry}   % set up margins
\usepackage[T1]{fontenc}
\usepackage[usenames,dvipsnames]{xcolor}
\usepackage{enumerate}              % fancy enumerate
\usepackage{amsmath}                % used for \eqref{} in this document
\usepackage{amsthm}
\theoremstyle{definition}
\newtheorem{exmp}{Example}[section]
\usepackage{verbatim}               % useful for \begin{comment} and \end{comment}
\usepackage{eurosym}                % used for euro symbol
\usepackage{caption} 
\usepackage{graphicx}
\usepackage{threeparttable}
\graphicspath{{Figures/}}
\usepackage{subcaption}
\usepackage{booktabs}
\usepackage{color}
\usepackage{float}
\usepackage{amssymb}
\usepackage{sgamevar}
\usepackage{sgame}
\usepackage[colorlinks=true]{hyperref}
\hypersetup{colorlinks=true, citecolor=ForestGreen, linkcolor=BlueViolet, urlcolor=Magenta}

\usepackage{array}
\newcolumntype{H}{@{}>{\lrbox0}l<{\endlrbox}}


%Solutions or nah (blank next two lines out for no solutions, unblank #3)
%\printanswers
%\newcommand{\dd}[1]{{\textbf{\textcolor{red}{#1}}}}
%\newcommand{\ddp}[1]{\par {\textcolor{ForestGreen}{#1}}}

\newcommand{\dd}[1]{}  
\newcommand{\ddp}[1]{}

\setlength\parindent{0pt}
\unframedsolutions
\SolutionEmphasis{\color{red}}
\CorrectChoiceEmphasis{\color{red}}
\renewcommand{\choicelabel}{(\alph{choice})}
\newcommand{\blank}[0]{\underline{\hspace{3cm}}}
\pointformat{\bfseries[\thepoints]}
\pointpoints{pt}{pts}
\pointsinrightmargin

\begin{document}


\title{\textbf{Homework 8 \dd{\\Solutions}} \\ \vspace{2 mm} {\large ECON 380} \\ \large{UNC Chapel Hill}}
\date{}
\maketitle

\makebox[\textwidth]{Name:\enspace\hrulefill}
\\

\makebox[\textwidth]{ONYEN:\enspace\hrulefill}
\\

\begin{center}
	\fbox{\fbox{\parbox{5.5in}{\centering
				This homework is due on \textbf{December 7} by \textbf{2:15PM}. You must turn in your work on a printed copy of this document in order for it to be graded. Your assignment must be stapled and in the correct order. Non-stapled assignments will automatically receive a 10 point deduction. There are a total of 50 available points.}}}
\end{center}
	

\begin{questions}
	

\question Suppose we are analyzing the economic performance of migrants over time by looking at census data from 2010. There are three migrant cohorts in the population described as follows:

\begin{enumerate}[i.]
	\item 1990 cohort: Low-skilled group with average skill level $\bar{S}_{90} = 2,000$
	\item 2000 cohort: Medium-skilled group with average skill level $\bar{S}_{00} = 6,000$
	\item 2010 cohort: Highly-skilled group with average skill level $\bar{S}_{10} = 12,000$
\end{enumerate}

For simplicity, assume that all migrants in each cohort arrived at age 20. Additionally, suppose that the average native skill level is $\bar{S}_N = 6,000$.
\\

Average wages increase with age (i.e., experience) for each group $g$ as follows:

\[\bar{w}_g = \$1\times \bar{S}_g + \$1,000\times Age\]

\begin{parts}
	\part[3] What is the average wage each migrant cohort received when they first migrate to the U.S?
\begin{solution}[1.65in]
Each migrant cohort arrives at age 20, so the average wage received is
\[\bar{w}_{90}(20) = \$1\times 2,000 + \$1,000\times 20 = \$22,000\]
\[\bar{w}_{00}(20) = \$1\times 6,000 + \$1,000\times 20 = \$26,000\]
\[\bar{w}_{10}(20) = \$1\times 12,000 + \$1,000\times 20 = \$32,000\]
\end{solution}
\ddp{(1) each}
	\part[3] What is the average wage of each migrant cohort when we observe them in the 2010 census?
\begin{solution}[1.5in]
Each migrant cohort arrives at age 20, so we observe the 1990 cohort at age 40, the 2000 cohort at age 30, and the 2010 cohort at age 20:
\[\bar{w}_{90}(40) = \$1\times 2,000 + \$1,000\times 40 = \$42,000\]
\[\bar{w}_{00}(30) = \$1\times 6,000 + \$1,000\times 30 = \$36,000\]
\[\bar{w}_{10}(20) = \$1\times 12,000 + \$1,000\times 20 = \$32,000\]
\ddp{(1) each}
\end{solution}
\uplevel{In Figure \ref{fig1} below, draw and clearly label each of the following:}
	\part[4] The age-earnings profile of each migrant cohort as well as the age-earnings profile of native workers. \ddp{(1) each}
	\part[2] The predicted age-earnings profile for migrants if we naively assume that migrant cohorts are equivalent and use only the age-earnings data we observe. \ddp{(2) as long as their profile matches their data points from (b)}

\begin{figure}[H]
	\centering
	\includegraphics[scale=.55]{hw8_1}
	\caption{Age-Earnings Profile}
	\label{fig1}
\end{figure}

%\begin{figure}[H]
%	\centering
%	\includegraphics[scale=.6]{hw8_1sol}
%	\caption{Age-Earnings Profile}
%	\label{fig1}
%\end{figure}

\part[2] Does our estimated migrant age-earnings profile show an earnings deficiency for recent migrants compared to native workers? 
\begin{solution}[.5in]
No. The earnings of recent migrants (the 2010 cohort who are 20) are greater than the earnings of equivalent 20 year old natives (\$32,000 versus \$26,000)
\end{solution}
\part[2] Does our estimated migrant age-earnings profile show a positive or negative effect of length of stay on wages? 
\begin{solution}[.5in]
It shows a positive effect of length of stay since it is upward sloping (wages increase with length of stay).
\end{solution}
\part[4] Is our estimated effect of length of stay on migrant earnings biased? If so, in which direction and why? 
\begin{solution}[1.5in]
Our estimated effect without taking into account cohort effects is biased downwards (i.e., negatively biased). The slope of the age-earnings profile of each cohort is greater than that of our estimated migrant age-earnings profile, so we are underestimating the effect of length of stay on wages. This is due to the fact that the quality of each migrant cohort is increasing, yet we are not accounting for that due to only observing cross-sectional data. Notice that we still correctly predict that length of stay has a positive effect on wages, but the magnitude of our predicted effect is smaller than the actual effect. Additionally, if we had even bigger differences in the quality of each cohort (so that earlier cohorts had even smaller skill levels and thus their age-earnings profiles were shifted down further relative to the 2010 cohort), our estimated effect of length of stay could even have been negative even though the effect of length of stay is actually positive for each cohort.
\end{solution}

\ddp{(2) biased down, (1) different slope, (1) due to increasing migrant quality}

\end{parts}

\question (Borjas 8.4) Labor demand for low-skilled workers in the United States is $w = 24 - 0.1E$ where
E is the number of workers (in millions) and $w$ is the hourly wage. There are
120 million domestic U.S. low-skilled workers who supply labor inelastically. If the
U.S. opened its borders to immigration, 20 million low-skill immigrants would enter
the U.S. and supply labor inelastically. 
\begin{parts}
	\part[2] What is the market-clearing wage if immigration is not allowed? 
\begin{solution}[.75in]
	$w^* = 24 - .1(120) = \$12$
\end{solution}
	\part[4] Draw a graph showing the effect of an open borders policy.
\begin{solution}[2.5in]
	\begin{figure}[H]
\centering
\includegraphics[scale=.5]{hw8_2}
	\end{figure}
\end{solution}
	\part[2] What is the market-clearing wage with open borders? 
\begin{solution}[.75in]
	$w^{*'} = 24 - .1(120+20) = \$10$
\end{solution}
	\part[4] How much is the immigration surplus when the U.S. opens its borders? 
\begin{solution}[1in]
	Immigration surplus = $1/2 \times (12-10) \times (140 - 120) = \$20,000,000$
\end{solution}
	\part[2] How much income is transferred from domestic workers to domestic firms?
\begin{solution}[1in]
	Income transferred from domestic workers to firms = $(\$12 - \$10) \times 120 = \$240,000,000$
\end{solution}
\end{parts}

\uplevel{Directions: Type your answers to the following questions and attach them to the back of this packet.}

\question Read the abstract, introduction, and conclusion in Tunali (2000). 

\begin{parts}
	\part[3] Briefly (3-4 sentences) describe his findings in regards to (i) the ``rationality hypothesis'' and (ii) the returns to migrants in his analysis.
	\begin{solution}
		\begin{enumerate}[i.]
			\item Tunali finds evidence in support of the ``rationality hypothesis:'' Both migrants and nonmigrants chose the option in which they had a comparative advantage
			\item The analysis also finds that upon migrating, the majority of individuals in the sample (3/4) realize negative returns from migration, while a small number of migrants realize extremely high returns
		\end{enumerate}
	\end{solution} 
	\part[1] What are the possible reasons stated in the conclusion as to why returns to migration might vary across individuals?
	\begin{solution}
	\begin{enumerate}[i.]
		\item Migration is a risky lottery in that ``lucky'' migrants realize high returns, but most migrants are ``unlucky'' and realize low or negative returns
		\item Migrants are making mistakes and moving when they should not. In the paper, Tunali finds that migrants from urban areas do better than migrants from rural areas, suggesting that they may have better information about opportunities elsewhere
	\end{enumerate}
	\end{solution} 
\end{parts}

\question Read ``The Immigrant Equation'' by Roger Lowenstein from The New York Times. 

\begin{parts}
	\part[3] Briefly summarize (1-2 sentences) Borjas' beliefs regarding the effect of immigration on poor Americans.  
	\begin{solution}
		Borjas believes that the costs of immigration to low-income Americans are large, with increases in immigration levels leading to decreased earnings and employment prospects.
	\end{solution}
	\part[3] Briefly summarize (1-2 sentences) Card's beliefs regarding the effect of immigration on poor Americans.
	\begin{solution}
	Card believes it simply isn't this big of a concern and that the costs on low-income Americans are quite small.
	\end{solution}
	\part[3] Compare the skill sets between:  (i) US natives and immigrants during the immigrant wave between (roughly) 1880 - 1920 and (ii) US natives and immigrants during the most recent immigrant wave.
	\begin{solution}
	(i)  The earlier cohort was fairly similar to US workers in terms of their skill-level  (ii) The most recent cohort of immigrants from Latin America are poor relative to the typical American and also have lower average education levels.  
	\end{solution}
	\part[3] Explain Giovanni Peri's theory regarding the effect of immigration on native, low-skill workers.
	\begin{solution}
	 Peri notes that there may be complementarities such that new immigrants are substitutes for prior immigrants but complements for many native workers.  Essentially, immigrant workers compete with each other in one market, and natives (even low-skill natives) compete in a somewhat separate market.  The harm done to native HS workers, according to Peri, can be minimal, on the order of 1\%.
	\end{solution}
\end{parts}

\end{questions}


\end{document}