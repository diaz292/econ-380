\documentclass[addpoints,11pt]{exam}

\usepackage{alltt}
\usepackage[margin=1in]{geometry}   % set up margins
\usepackage[T1]{fontenc}
\usepackage[usenames,dvipsnames]{xcolor}
\usepackage{enumerate}              % fancy enumerate
\usepackage{amsmath}                % used for \eqref{} in this document
\usepackage{amsthm}
\theoremstyle{definition}
\newtheorem{exmp}{Example}[section]
\usepackage{verbatim}               % useful for \begin{comment} and \end{comment}
\usepackage{eurosym}                % used for euro symbol
\usepackage{caption} 
\usepackage{graphicx}
\usepackage{threeparttable}
\graphicspath{{Figures/}}
\usepackage{subcaption}
\usepackage{booktabs}
\usepackage{color}
\usepackage{float}
\usepackage{amssymb}
\usepackage{sgamevar}
\usepackage{sgame}
\usepackage[colorlinks=true]{hyperref}
\hypersetup{colorlinks=true, citecolor=ForestGreen, linkcolor=BlueViolet, urlcolor=Magenta}

\usepackage{array}
\newcolumntype{H}{@{}>{\lrbox0}l<{\endlrbox}}


%Solutions or nah (blank next two lines out for no solutions, unblank #3)
%\printanswers
%\newcommand{\dd}[1]{{\textbf{\textcolor{red}{#1}}}}
%\newcommand{\ddp}[1]{\par {\textcolor{ForestGreen}{#1}}}

\newcommand{\dd}[1]{}  
\newcommand{\ddp}[1]{}

\setlength\parindent{0pt}
\unframedsolutions
\SolutionEmphasis{\color{red}}
\CorrectChoiceEmphasis{\color{red}}
\renewcommand{\choicelabel}{(\alph{choice})}
\newcommand{\blank}[0]{\underline{\hspace{3cm}}}
\pointformat{\bfseries[\thepoints]}
\pointpoints{pt}{pts}
\pointsinrightmargin

\begin{document}


\title{\textbf{Homework 5 \dd{\\Solutions}} \\ \vspace{2 mm} {\large ECON 380} \\ \large{UNC Chapel Hill}}
\date{}
\maketitle

\makebox[\textwidth]{Name:\enspace\hrulefill}
\\

\makebox[\textwidth]{ONYEN:\enspace\hrulefill}
\\

\begin{center}
	\fbox{\fbox{\parbox{5.5in}{\centering
				This homework is due on \textbf{March 27} by \textbf{12:05PM}. You must turn in your work on a printed copy of this document in order for it to be graded. Your assignment must be stapled and in the correct order. Non-stapled assignments will automatically receive a 10 point deduction. There are a total of 50 available points.}}}
\end{center}
	

\begin{questions}
	
\question (Borjas 7.4) Consider a simple economy where 90 percent of citizens report an annual income of \$10,000 while the remaining 10 percent report an annual income of \$110,000.

\begin{parts}
\part[4] What is the Gini coefficient associated with this economy?
\begin{solution}[1in]
The average individual income in the nation will be given by $.90(\$10,000) + .10(\$110,000) = \$9,000 + \$11,000 = \$20,000$. Each individual in the low-income and high-income group make the same amount, so the share of income going to each group as a whole will be 
\[\text{Share of income going to low-income group} = \frac{\$9,000}{\$20,000} = 45\%\]
\[\text{Share of income going to high-income group} = \frac{\$11,000}{\$20,000} = 55\%\]

Again, since each individual in their respective group makes the same amount, they each contribute the same share of income within their group so the Lorentz curve will be a straight line within groups:


\begin{figure}[H]
	\centering 
	\includegraphics[scale=.55]{hw5_1}
\end{figure}

The red curve is the perfect equality Lorenz curve, while the blue curve is the actual Lorenz curve in this economy. The Gini coefficient is given by Area A $\div$ (Area A + B + C + D).
\\

Area B = (1/2)(.10)(.55) = .0275, Area C = 1/2(.90)(.45) = .2025, Area D = .10(.45) = .045 $\Rightarrow$ Area A = $.5 - (.0275 + .2025 + .045) = .225$ $\Rightarrow$ Gini coefficient = .225/.5 = .45

\ddp{Points: (2) work, (2) correct coefficient}

\end{solution}

\part[4] Is the presence of an underground economy likely to result in a Gini coefficient that overstates or understates poverty? Why?
\begin{solution}[1in]
An underground economy is likely to result in a Gini coefficient that over-states poverty. The underground economy tends to employ low-skill, low-income workers, so underreporting their income will overstate poverty \& income inequality.
\end{solution}
\part[4] Suppose the poorest 90 percent of citizens actually have an income of \$15,000
because each receives \$5,000 of unreported income from the underground economy.
What is the Gini coefficient now?
\begin{solution}[1in]
The average individual income in the nation will be given by $.90(\$15,000) + .10(\$110,000) = \$13,500 + \$11,000 = \$24,500$. Each individual in the low-income and high-income group make the same amount, so the share of income going to each group as a whole will be 
\[\text{Share of income going to low-income group} = \frac{\$13,500}{\$24,500} \approx 55\%\]
\[\text{Share of income going to high-income group} = \frac{\$11,000}{\$24,500} \approx 45\%\]

Calculating the Gini coefficient is the same as above; Gini coefficient $\approx$ .35.

\ddp{Points: (2) work, (2) correct coefficient}

\end{solution}
\end{parts}
	
\question There are two people in the Springfield economy: Homer and Burns. Each individual gets utility from consumption:

\[U(C_H) = \sqrt{C_H}\]
\[U(C_B) = \sqrt{C_B}\]	

Initially, suppose that Homer produces 10 units of output (hence, with no redistribution, can consume $C_H$ = 10) while Burns produces 90 units of output.


\begin{parts}
	\part[5] Suppose a ``benevolent social planner'' is trying to determine the optimal level of redistribution (in other words, how many units of consumption Burns should give to Homer). To do so, this social planner wishes to maximize the sum of
	the utilities of Homer and Burns. Define ``social utility'' as
	
	\[U_S(C_H,C_B) = \sqrt{C_H} + \sqrt{C_B}\]
	
	For now, assume that redistribution does not affect how much each individual produces: Homer still always produces 10, Burns still always produces 90, and the social planner is simply trying to figure out the best way to distribute 100
	units of output. Fill in the empty cells in the following table. Round the ``social utility'' to the nearest hundredth.
	
		\begin{table}[H]
		\centering
		\begin{tabular}{c|c|c|c|c}
			Transfer & Total Production & $C_H$ & $C_B$ & Social Utility \\
			\hline 
			0 & 100 & 10 & 90 & 12.65 \\
			10 & 100 & 20 & 80 & \dd{13.42}\\
			20 & \dd{100} & \dd{30} & \dd{70} & \dd{13.84} \\
			30 &  \dd{100} & \dd{40} & \dd{60} & \dd{14.07} \\
			40 &  \dd{100} & \dd{50} & \dd{50} & \dd{14.14} \\
			50 &  \dd{100} & \dd{60} & \dd{40} & \dd{14.07} \\
			60 &  \dd{100} & \dd{70} & \dd{30} & \dd{13.84} \\
		\end{tabular}
	\end{table}

\ddp{(1) per correct column. (1) freebie}

\part[2] What is the optimal transfer from Burns to Homer? What are each respective agents' consumption levels after the transfer?

\begin{solution}[.75in]
Optimal transfer is to completely redistribute: send 40 from Burns to Homer such that each get 50.	
\end{solution}

\part[5] A common concern among economists is that redistribution affects incentives, so assuming that Burns and Homer still produce just as much output regardless of the transfer program is flawed: For both Burns and Homer, their respective incomes are only partially determined by their own actions, and partially determined by the others' actions, which may decrease their individual efforts to produce output. Now, suppose that transfers are costly: For each 10 units transferred, each individual produces one fewer unit of output. For example, if we transfer 10 units of output,
then Burns produces 89, Homer produces 9, and final consumption is $89 - 10 = 79$ for Burns and $9 + 10 = 19$ for Homer.
Fill out the following table, rounding social utility to the nearest hundredth.

	\begin{table}[H]
	\centering
	\begin{tabular}{c|c|c|c|c}
		Transfer & Total Production & $C_H$ & $C_B$ & Social Utility \\
		\hline 
		0 & 100 & 10 & 90 & 12.65 \\
		10 & 98 & 19 & 79 & \dd{13.25} \\
		20 & \dd{96} & \dd{28} & \dd{68} & \dd{13.54} \\
		30 & \dd{94} & \dd{37} & \dd{57} & \dd{13.63} \\
		40 & \dd{92} & \dd{46} & \dd{46} & \dd{13.56} \\
		50 & \dd{90} & \dd{55} & \dd{35} & \dd{13.33} \\
		60 & \dd{88} & \dd{64} & \dd{24} & \dd{12.90} \\
	\end{tabular}
\end{table}

\ddp{(1) per correct column. (1) freebie}

\part[2] What is the optimal transfer from Burns to Homer (of those listed on the table)? What are each respective agents' consumption levels after the transfer?

\begin{solution}[.75in]
Optimal transfer is to redistribute only 30 from Burns to Homer. Homer consumes 37, Burns consumes 57.	
\end{solution}

\end{parts}	


\question[4] (Borjas 7.11) Suppose two households earn \$40,000 and \$56,000, respectively. What is the expected percent difference in wages among the children, grandchildren, and great-grandchildren of the two households if the intergenerational correlation of earnings is 0.2, 0.4, or 0.6?

\begin{solution}[1.75in]
The percent difference in earnings between the two households is $\frac{(56,000 - 40,000)}{40,000} = 40\%$. The expected difference in wages among the children, grandchildren, and great-grandchildren is given by $40\%\times r$, $40\%\times r^2$, and $40\%\times r^3$, respectively, where $r$ is the intergenerational correlation of earnings.  

	\begin{table}[H]
	\centering
	\begin{tabular}{c|c|c|c}
		Correlation Coefficient & Children & Grandchildren & Great-Grandchildren \\
		\hline 
		.20 & 8\% & 1.6\% & .32\% \\
		.40 & 16\% & 6.4\% & 2.56\% \\
		.60 & 24\% & 14.4\% & 8.64\%  
	\end{tabular}
\end{table}

\ddp{(1) per correct row/column. (1) freebie}
	
\end{solution}

\uplevel{Directions: Type your answers to the following questions and attach them to the back of this packet.}

\question Read the following article: \hyperref{http://www.becker-posner-blog.com/2011/01/bad-and-good-inequality-becker.html}{category}{name}{Bad and Good Inequality by Gary Becker}


\begin{parts}
	
\part[4] What does Becker cite as an example of ``good'' inequality? Briefly explain why such inequality can create social value.
\begin{solution}
Becker gives the example of schooling. Investments into higher education are costly and also enhance productivity; in absence of a wage system which incentivizes investment into skills, fewer individuals would make these productivity-enhancing investments.
\end{solution}

\part[4] What does Becker cite as an example of ``bad'' inequality?
\begin{solution}
Chinese residence systems; in the past it was extraordinarily difficult for a rural individual in China to gain residence in an urban area. Opportunities in rural areas were hugely limited (and still are), so the gap between the urban rich and rural poor was largely determined at birth.	
\end{solution}
\part[4] According to Becker, what has happened to global income inequality over the past 30 years? Given that many developed countries have seen inequality become more severe, what can be inferred about the rate of growth in the developing/undeveloped world compared to the rate of growth in the developed world over this timeframe?
\begin{solution}
It's actually decreased. Within-country, income inequality has risen. However, on a global scale, inequality has fallen because the developing/undeveloped world has seen significantly higher growth rates than the developed world.
Countries like China, South Korea have seen extraordinary growth over the past 40 years, and there are even glimmers of strong development in Africa in more recent years (e.g. Ghana has seen 5-6\% growth for some time now).
\end{solution}	
\end{parts}

\question Read the following article: \hyperref{http://www.nytimes.com/2012/08/05/books/review/the-price-of-inequality-by-joseph-e-stiglitz.html?_r=1}{category}{name}{Separate and Unequal by Thomas Edsall}


\begin{parts}
	\part[4] Which of the causes of inequality, as discussed in the slides, is most consistent with Stiglitz's views on the factors
	driving inequality in recent years?
	\begin{solution}
	He chalks up most of the increase in inequality which we've seen to factors which fall under the umbrella of Cronyism. Essentially, he argues that the primary driver of inequality is the ability of wealthy individuals to use their wealth/power to gain favorable policy treatment.
	\end{solution}
	\part[4] In 3-4 sentences, explain why Stiglitz beliefs that excessive inequality leads to lower growth and less efficiency.
	\begin{solution}
	His argument is basically that policy becomes too focused on the well-being of the wealthy, who are deeply entrenched in the political system. In short, if the ultra-wealthy have the ability to influence policy, they may have a strong incentive to lobby for keeping the government small as a means to avoid future redistribution. If this smaller government comes at the cost of substantially decreased infrastructure in poor areas (bad public schools, roads, etc.) then little growth would come out of those areas.
	\end{solution}
\end{parts}

\end{questions}




\end{document}